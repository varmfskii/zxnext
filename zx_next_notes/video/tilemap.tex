\section{Tilemap Mode}

February 25, 2019  Phoebus Dokos

\subsection{General Description}
The tilemap is a hardware character oriented display that comes in two
resolutions: $40\times32$ ($320\times256$ pixels) and $80\times32$
($640\times256$ pixels).

The display area on screen is the same as the sprite layer, meaning it
overlaps the standard $256\times192$ area by 32 pixels on all
sides. Vertically this is larger than the physical HDMI display, which
will cut off the top and bottom character rows making the visible area
$40\times30$ or $80\times30$, but the full area is visible on VGA.

The obvious application for the tilemap is for a fast, clearly
readable and wide multicoloured character display. Less obvious
perhaps is that it can also be used to make fast and wide resolution
full colour backgrounds with easily animated components.

The tilemap is defined by two data structures.

\subsection{Data Structures}

\paragraph{Tilemap}

The first data structure is the tilemap itself which indicates what
characters occupy each cell on screen. Each tilemap entry is two bytes
so for $40\times32$ resolution, a full size tilemap will occupy 2560
bytes, and for $80\times32$ resolution the space taken is twice that
at 5120 bytes. The tilemap entries are stored in X-major order and
each two-byte tilemap entry is stored little endian:

Tilemap Entry
\begin{itemize}
\item[] bits 15-12 : palette offset
\item[] bit 11 : x mirror
\item[] bit 10 : y mirror
\item[] bit 9 : rotate
\item[] bit 8 : ULA over tilemap (in 512 tile mode, bit 8 of the tile number)
\item[] bits 7-0 : tile number
\end{itemize}
The character displayed is indicated by the “tile number” which can be
thought of as an ASCII code. The tile number is normally eight bits
allowing up to 256 unique tiles to be displayed but this can be
extended to nine bits for 512 unique tiles if 512 tile mode is enabled
via the Tilemap Control register.

The other bits are tile attributes that modify how the tile image is
drawn. Their function is the same as the equivalent sprite attributes
for sprites. Bits apply rotation then mirroring, and colour can be
shifted with a palette offset. If 512 tile mode is not enabled, bit 8
will determine if the tile is above or below the ULA display on a per
tile basis.

\paragraph{Tile Definitions}
The second data structure is the tile definitions themselves.

Each tile, identified by tile number, is $8\times8$ pixels in size
with each pixel four bits to select one of 16 colours. A tile
definition occupies 32 bytes and is defined in X major order with
packing in the X direction in the same way that 4-bit sprites are
defined. The 4-bit colour of each pixel is augmented by the 4-bit
palette offset from the tilemap in the most significant bits to form
an 8-bit colour index that is looked up in the tilemap palette to
determine the final 9-bit colour sent to the display.

Tiles are therefore defined using 16 colours with the tilemap palette
offset able to act as index into the tilemap palette to vary the
display colour. One of the 16 colours is defined as transparent in the
Transparency Index register.

\subsection{Memory Organization \& Display Layer}
The tilemap is a logical extension of the ULA and its data structures
are contained in the ULAís 16k bank 5. If both the ULA and tilemap are
enabled, this means that the tilemapís map and tile definitions should
be arranged within the 16k to avoid overlap with the display ram used
by the ULA.

The tilemap exists on the same display layer as the ULA. The graphics
generated by the ULA and tilemap are combined before being forwarded
to the SLU layer system as layer U.

\subsection{Combining ULA \& Tilemap}
The combination of the ULA and tilemap is done in one of two modes:
the standard mode or the stencil mode.

The standard mode uses bit 8 of a tile's tilemap entry to determine if
a tile is above or below the ULA. The source of the final pixel
generated is then the topmost non-transparent pixel. If the ULA or
tilemap is disabled then they are treated as transparent.

The stencil mode will only be applied if both the ULA and tilemap are
enabled. In the stencil mode, the final pixel will be transparent if
either the ULA or tilemap are transparent. Otherwise the final pixel
is a logical AND of the corresponding colour bits. The stencil mode
allows one layer to act as a cut-out for the other.

\subsection{Programming Tilemap mode}

\begin{verbatim}
(R/W) $6B (107) => Tilemap Control
  bit 7    = 1 to enable the tilemap
  bit 6    = 0 for 40x32, 1 for 80x32
  bit 5    = 1 to eliminate the attribute entry in the tilemap
  bit 4    = palette select
  bits 3-2 = Reserved set to 0
  bit 1    = 1 to activate 512 tile mode
  bit 0    = 1 to force tilemap on top of ULA
\end{verbatim}

Bits 7 \& 6 enable the tilemap and select resolution. Bit 4 selects one
of two tilemap palettes used for final colour lookup. Bit 5 changes
the structure of the tilemap so that it contains only 8-bit tilemap
entries instead of 16-bit tilemap entries. If 8-bit, the tilemap only
contains tile numbers and the attributes are instead taken from
nextreg \$6C.

Bit 1 activates 512 tile mode. In this mode, the “ULA over tilemap”
bit in a tile’s attribute is re-purposed as the ninth bit of the tile
number, allowing up to 512 unique tiles to be displayed. In this mode,
the ULA is always on top of the tilemap.

Bit 0 forces the tilemap to be on top of the ULA. It can be useful in
512 tile mode to change the relative display order of the ULA and
tilemap.

\begin{verbatim}
(R/W) $6C (108) => Default Tilemap Attribute
  bits 7-4 = Palette Offset
  bit 3    = X mirror
  bit 2    = Y mirror
  bit 1    = Rotate
  bit 0    = ULA over tilemap
             (bit 8 of the tile number if 512 tile mode is enabled)
\end{verbatim}

If bit 5 of nextreg \$6B is set, the tilemapís structure is modified
to contain only 8-bit tile numbers instead of the usual 16-bit tilemap
entries. In this case, the tile attributes used are taken from this
register instead.

\begin{verbatim}
(R/W) $6E (110) => Tilemap Base Address
  bits 7-6 = Read back as zero, write values ignored
  bits 5-0 = MSB of address of the tilemap in Bank 5
\end{verbatim}

This register determines the tilemapís base address in bank 5. The
base address is the MSB of an offset into the 16k bank, allowing the
tilemap to begin at any multiple of 256 bytes in the bank. Writing a
physical MSB address in \$40-\$7f or \$c0-\$ff, corresponding to
traditional ULA physical addresses, is permitted. The value read back
should be treated as a fully significant 8-bit value.

The tilemap will be $40\times32$ or $80\times32$ in size depending on
the resolution selected in nextreg \$6B. Each entry in the tilemap is
normally two bytes but can be one byte if attributes are eliminated by
setting bit 5 of nextreg \$6B.

\begin{verbatim}
(R/W) $6F (111) => Tile Definitions Base Address
  bits 7-6 = Read back as zero, write values ignored
  bits 5-0 = MSB of address of tile definitions in Bank 5
\end{verbatim}

This register determines the base address of tile definitions in bank
5. As with nextreg \$6E, the base address is the MSB of the an offset
into the 16k bank, allowing tile definitions to begin at any multiple
of 256 bytes in the bank. Writing a physical MSB address in \$40-\$7f
or \$c0-\$ff, corresponding to traditional ULA physical addresses, is
permitted. The value read back should be treated as a fully
significant 8-bit value.

Each tile definition is 32 bytes in size and is located at address:

\begin{verbatim}
Tile Def Base Addr + 32 * (Tile Number)

(R/W) $4C (76) => Transparency index for the tilemap
bits 7-4 = Reserved, must be 0
bits 3-0 = Set the index value ($F after reset)
\end{verbatim}

Defines the transparent colour index for tiles. The 4-bit pixels of a
tile definition are compared to this value to determine if they are
transparent.

\begin{verbatim}
(R/W) $43 (67) => Palette Control
  bit 7 = '1' to disable palette write auto-increment.
  bits 6-4 = Select palette for reading or writing:
     000 = ULA first palette
     100 = ULA second palette
     001 = Layer 2 first palette
     101 = Layer 2 second palette
     010 = Sprites first palette 
     110 = Sprites second palette
     011 = Tilemap first palette
     111 = Tilemap second palette
  bit 3 = Select Sprites palette (0 = first palette, 1 = second palette)
  bit 2 = Select Layer 2 palette (0 = first palette, 1 = second palette)
  bit 1 = Select ULA palette (0 = first palette, 1 = second palette)
  bit 0 = Enabe ULANext mode if 1. (0 after a reset)
\end{verbatim}

The tilemap has its own pair of palettes for looking up 9-bit
colours. Each tile definition pixel is 4-bits which is combined with
the 4-bit palette offset from the tilemap entry in the most
significant 8-bits. This 8-bit value is passed through the tilemap
palette to generate the final 9-bit colour for the pixel.

\begin{verbatim}
(R/W) $1B (27) => Clip Window Tilemap
  bits 7-0 = Coord. of the clip window
  1st write = X1 position
  2nd write = X2 position
  3rd write = Y1 position
  4rd write = Y2 position
  The values are 0,159,0,255 after a Reset
  Reads do not advance the clip position
\end{verbatim}

The values are 0,159,0,255 after a Reset

Reads do not advance the clip position

The tilemap display surface extends 32 pixels around the central
$256\times192$ display. The origin of the clip window is the top left
corner of this area 32 pixels to the left and 32 pixels above the
central $256\times192$ display. The X coordinates are internally
doubled to cover the full 320 pixel width of the surface. The clip
window indicates the portion of the tilemap display that is
non-transparent and its indicated extent is inclusive; it will extend
from X1*2 to X2*2+1 horizontally and from Y1 to Y2 vertically.

\begin{verbatim}
(R/W) $2F (47) => Tilemap Offset X MSB
bits 7-2 = Reserved, must be 0
bits 1-0 = MSB X Offset
\end{verbatim}

Meaningful Range is 0-319 in 40 char mode, 0-639 in 80 char mode

\begin{verbatim}
(R/W) $30 (48) => Tilemap Offset X LSB
  bits 7-0 = LSB X Offset
\end{verbatim}

Meaningful range is 0-319 in 40 char mode, 0-639 in 80 char mode

\begin{verbatim}
(R/W) $31 (49) => Tilemap Offset Y
  bits 7-0 = Y Offset (0-191)
\end{verbatim}

These are scroll registers for scrolling the tilemap area. As with
other layers, the scroll region wraps.

\begin{verbatim}
(R/W) $68 (104) => ULA Control
  bit 7    = 1 to disable ULA output
  bit 6    = 0 to select the ULA colour for blending in SLU modes 6 & 7
           = 1 to select the ULA/tilemap mix for blending in SLU modes 6 & 7
  bits 5-1 = Reserved must be 0
  bit 0    = 1 to enable stencil mode when both the ULA and tilemap are enabled
            (if either are transparent the result is transparent otherwise the
             result is a logical AND of both colours)
\end{verbatim}

Bit 0 can be set to choose stencil mode for the combined output of the
ULA and tilemap. Bit 6 determines what colour is used in SLU modes 6 \&
7 where the ULA is combined with Layer 2 to generate highlighting
effects.

Changes Since 2.00.26

\begin{enumerate}
\item 512 Tile Mode. In 2.00.26, the 512 tile mode was automatically
  selected when the ULA was disabled. With the ULA disabled, the
  tilemap attribute bit “ULA on top” was re-purposed to be bit 8 of
  the tile number. In 2.00.27, selection of the 512 tile mode is moved
  to bit 1 of Tilemap Control nextreg \$6B. This way 512 tile mode can
  be independently chosen without disabling the ULA. The “ULA on top”
  bit is still taken as bit 8 of the tile number and in the 512 mode,
  the tilemap is always displayed underneath the ULA.
\item Tilemap Always On Top of ULA. In 2.00.27, bit 0 of Tilemap
  Control nextreg \$6B is used to indicate that the tilemap should
  always be displayed on top of the ULA. This allows the tilemap to
  display over the ULA when in 512 mode.
\end{enumerate}

Future Direction

The following compatible changes may be applied at a later date:

\begin{enumerate}
\item Addition of a bit to Tilemap Control to select a reduced tilemap
  area of size $32\times24$ or $64\times24$ that covers the ULA
  screen.
\item Addition of a bit to Tilemap Control to select split addressing
  where the tilemap’s tiles and attributes as well as the tile
  definitions are split between the two 8k halves of the 16k ULA ram
  in the same way that the two Timex display files are split. The
  intention is to make it easier for the tilemap to co-exist with all
  the display modes of the ULA.
\end{enumerate}
