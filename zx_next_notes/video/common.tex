\section{Common Functionality}

\paragraph{Colour Palette}

The ZX Spectrum Next has 8 palettes, two each for ULA, tilemap,
layer2, and sprites.  There is a palette in use and an alternate that
can be quickly swapped.  Each palette consists of 256 entries out of a
total of 512 (RRRGGGBBB) colours.  Register \$43 (67, Palette Control)
is used to select which palette can be read/written and whether each
layer is using its primary or secondary palette and whether or not
writes to the palette value registers auto-increment. Palette values
are written to registers \$41 (65, 8-bit) and \$44 (70, 9-bit)

(R/W) \$40 (64) $\Rightarrow$ Palette Index
\begin{itemize}
\item bits 7-0 = Select the palette index to change the associated colour.
For the ULA only, INKs are mapped to indices 0-7, Bright INKS to
indices 8-15, PAPERs to indices 16-23 and Bright PAPERs to indices
24-31.  In ULANext mode, INKs come from a subset of indices 0-127 and
PAPERs come from a subset of indices 128-255.  The number of active
indices depends on the number of attribute bits assigned to INK and
PAPER out of the attribute byte.  The ULA always takes border colour
from paper.
\end{itemize}
(R/W) \$41 (65) $\Rightarrow$ Palette Value (8 bit colour)
\begin{itemize}
\item bits 7-0 = Colour for the palette index selected by the register \$40.
(Format is RRRGGGBB - the lower blue bit of the 9-bit colour will be a
logical OR of blue bits 1 and 0 of this 8-bit value.)  After the
write, the palette index is auto-incremented to the next index if the
auto-increment is enabled at reg \$43.  Reads do not auto-increment.
\end{itemize}
(R/W) \$43 (67) $\Rightarrow$ Palette Control
\begin{itemize}
\item bit 7 = '1' to disable palette write auto-increment.
\item bits 6-4 = Select palette for reading or writing:
  \begin{itemize}
  \item 000 = ULA first palette
  \item 100 = ULA second palette
  \item 001 = Layer 2 first palette
  \item 101 = Layer 2 second palette
  \item 010 = Sprites first palette 
  \item 110 = Sprites second palette
  \item 011 = Tilemap first palette
  \item 111 = Tilemap second palette
  \end{itemize}
\item bit 3 = Select Sprites palette (0 = first palette, 1 = second palette)
\item bit 2 = Select Layer 2 palette (0 = first palette, 1 = second palette)
\item bit 1 = Select ULA palette (0 = first palette, 1 = second palette)
\item bit 0 = Enable ULANext mode if 1. (0 after a reset)
\end{itemize}
(R/W) \$44 (68) $\Rightarrow$ Palette Value (9 bit colour)

Two consecutive writes are needed to write the 9 bit colour

1st write:

\begin{itemize}
\item bits 7-0 = RRRGGGBB
\end{itemize}

2nd write: 

If writing a L2 palette

\hrulefill
\begin{itemize}
\item bit 7 = 1 for L2 priority colour, 0 for normal priority colour will always be on top even on an SLU priority arrangement. 

If you need the exact same colour on priority and non priority
locations you will need to program the same colour twice changing bit
7 to 0 for the second colour bits 6-1 = Reserved, must be 0

\item bit 0 = lsb B
\end{itemize}
     
If writing another palette

\hrulefill
\begin{itemize}
\item bits 7-1 = Reserved, must be 0
\item bit 0 = lsb B
\end{itemize}           
After the two consecutive writes the palette index is auto-incremented
if the auto-increment is enabled by reg \$43.
     
Reads only return the 2nd byte and do not auto-increment.


\section{Clipping}

There are two clipping modes: normal clipping and over border
clipping. With normal clipping the clipping coordinates are relative
to the regular $256\times192$ video area, so 0,0 is the display area origin,
X range is 0-255, and Y range is 0-191. With over border clipping, the
ULA origin is now located so the clipping coordinate 32,32 is the
display area origin, the X range is 0-320, but values given for X1 and
X2 are doubled, the Y range is now 0-255.

ULA and Layer2 always use normal clipping.  Tilemaps always use over
border clipping, and Using bit 5 of register \$15 selects which
clipping mode is used for sprites (0=normal, 1=over border).

Each group as its own register to control clipping (\$18=Layer2,
\$19=Sprites, \$1A=ULA, and \$1B=Tilemap) with the clipping area
controlled by four consecutive writes in the order X1, X2, Y1,
Y2. Where you are in this sequence can be controlled by either using
register \$1C to reset the next write to control X1 (bit 3=tilemap,
2=ULA, 1=sprite, and 0=layer2) or register \$1D to set all of the
indices (bits 7-6=tilemap, 5-4=layer 2, 3-2=sprite, and 1-0=ULA).

\section{Scrolling}

On the ZX Spectrum Next, all of the screen area (less clipping) is
visible at any given time. Scrolling controls where the screen origin
is located with your image being mapped onto a toroidal coordinate
system. Each group that is scrollable is controlled its won set of
registers.

Layer2 scrolling is controlled by registers \$16 (X) and \$17 (Y). The
legal range is 0-255 for X and 0-191 for Y.

ULA is controlled by registers \$32 (X-offset) and \$33
(Y-offset). Legacy ULA modes scroll horizontally in 8-pixel increments
(ignoring the least significant 3-bits) and vertically in 1-pixel
increments while LoRes scrolls in half-pixel increments. The legal
range is 0-255 for X and 0-191 for Y.

Tilemap scrolling uses three registers. \$2F (X bit 8), \$30 (X bits
7-0), \$31 (Y). The legal range is 0-319 for X and 0-255 for Y.
