\section{Layer 1}
The Layer 1 consists of ZX Spectrum ULA video, Timex video modes, and
the Spectrum Next’s lores video modes all use 16k memory bank 5 or 7
with the data coming from some combination of addresses \$0000-\$17FF
(bitmap 1), \$1800-\$1AFF (attribute 1), \$2000-\$37FF (bitmap 2), and
\$3800-\$3AFF (attribute 2) within the selected bank.  Assuming
default memory mapping and the use of bank 5 this will be mapped as
some combination of memory \$4000-\$57FF, \$5800-\$5AFF,
\$6000-\$77FF, \$780-\$7AFF. All of the modes other than the lores
mode can either use the default ZX Spectrum colours, ULANext mode, or
an emulation of ULAplus. In the Spectrum and Timex modes all colours are
either Paper (foreground), paper (background), or border colours.

\subsection{Colour Attributes}
The ZX Spectrum Next has three major modes for colour attributes: the
ZX Spectrum attribute mapping, which is augmented by using the ZX
Spectrum Next's palette; ULANext, which allows the user to how many
foreground and how many background colous are to be selected by the
attribute bytes; and an emulation of ULAplus.

\paragraph{ULA Colour}
In ULA colour INKs are mapped to indices 0-7, Bright INKS to indices
8-15, PAPERs to indices 16-23 and Bright PAPERs to indices 24-31. This
is the default state for interpreting ULA palettes.

\begin{table}[h]\centering
  \caption{ULA Colour}
  \csvautotabular{video/flash.csv}
\end{table}

\paragraph{ULANext}
The ULANext modes use a varying number of bits from the attribute
byte to determine the ink colours as the palette index from the
appropriate bits (all others being zero) and the paper colours coming
from the indicated value+128 with palette format 255 being a special
case where all the bits determine the ink colour while the paper is
always palette index 128. The ULA always takes border colour from
paper. ULANext is enabled using bit 0 of nextreg \$43 (palette
control) and controlled with nextreg \$42 (ULA Next attribute byte
format)

\begin{table}[h]\centering
  \caption{ULANext}
  \csvautotabular{video/palfmt.csv}
\end{table}

\paragraph{ULAplus}
The ZX Next emulates ULAPlus using the last 64 (192-255) entries of
the ULA palette. ULAplus is controlled using two ports: \$BF3B (register
port) and \$FF3B (data port)

\subparagraph{I/O ports}
ULAplus is controlled by two ports.

\$BF3B is the register port (write only)

The byte output will be interpreted as follows:
\begin{itemize}
\item Bits 7-6: Select the register group. Two groups are currently available:
  \begin{itemize}
  \item 00=palette group\\
    When this group is selected, the sub-group determines the entry in the
    palette table (0-63).
  \item 01=mode group\\
    The sub-group is (optionally) used to mirror the video functionality
    of Timex port \$FF as follows:
  \end{itemize}
\item Bits 5-0: Select the register sub-group
\item[] Mode group
\item Bits 5-3: Sets the screen colour in hi-res mode.
  \begin{itemize}
  \item 000=Black on White
  \item 001=Blue on Yellow
  \item 010=Red on Cyan
  \item 011=Magenta on Green
  \item 100=Green on Magenta
  \item 101=Cyan on Red
  \item 110=Yellow on Blue
  \item 111=White on Black
  \end{itemize}
\item Bits 2-0: Screen mode.
  \begin{itemize}
  \item 000=screen 0 (bank 5)
  \item 001=screen 1 (bank 5)
  \item 010=hi-colour (bank 5)
  \item 100=screen 0 (bank 7)
  \item 101=screen 1 (bank 7)
  \item 110=hi-colour (bank 7)
  \item 110=hi-res (bank 5)
  \item 111=hi-res (bank 7)
  \end{itemize}
\end{itemize}

\$FF3B is the data port (read/write)

When the palette group is selected, the byte written will describe the
color.

When the mode group is selected, the byte output will be interpreted
as follows:
\begin{itemize}
\item Bit 0: ULAplus palette on (1) / off (0)
\item Bit 1: (optional) grayscale: on (1) / off (0) (same as turing
  the color off on the television)
\end{itemize}

Implementations that support the Timex video modes use the \$FF
register as the primary means to set the video mode, as per the Timex
machines. It is left to the individual implementations to determine if
reading the port returns the previous write or the floating bus.

\subparagraph{GRB palette entries}

G3R3B2 encoding\\
For a device using the GRB colour space the palette entry is
interpreted as follows
\begin{itemize}
\item Bits 7-5: Green intensity.
\item Bits 4-2: Red intensity.
\item Bits 1-0: Blue intensity.
\end{itemize}

This colour space uses a sub-set of 9-bit GRB. The missing lowest blue
bit is set to OR of the other two blue bits (Bb becomes 000 for 00,
and Bb1 for anything else). This gives access to a fixed half the
potential 512 colour palette. The reduces the jump in intensity in the
lower range in the earlier version of the specification. It also means
the standard palette can now be represented by the ULAplus palette.

\subparagraph{Grayscale palette entries}
In grayscale mode, each palette entry describes an intensity from zero
to 255. This can be achieved by simply removing the colour from the
output signal.

\subparagraph{Limitations}
Although in theory 64 colours can be displayed at once, in practice
this is usually not possible except when displaying colour bars,
because the four CLUTs are mutually exclusive; it is not possible to
mix colours from two CLUTs in the same cell. However, with software
palette cycling it is possible to display all 256 colours on screen at
once.

\subparagraph{Emulation}
The 64 colour mode lookup table is organized as 4 palettes of 16
colours.

Bits 7 and 6 of each Spectrum attribute byte (normally used for FLASH
and BRIGHT) will be used as an index value (0-3) to select one of the
four colour palettes.

Each colour palette has 16 entries (8 for INK, 8 for PAPER). Bits 0 to
2 (INK) and 3 to 5 (PAPER) of the attribute byte will be used as
indexes to retrieve colour data from the selected palette.

With the standard Spectrum display, the BORDER colour is the same as
the PAPER colour in the first CLUT. For example BORDER 0 would set the
border to the same colour as PAPER 0 (with the BRIGHT and FLASH bits
not set).

The complete index can be calculated as\\
ink\_colour = (FLASH * 2 + BRIGHT) * 16 + INK
paper\_colour = (FLASH * 2 + BRIGHT) * 16 + PAPER + 8

\subparagraph{Palette file format}
The palette format doubles as the BASIC patch loader. This enables you
to edit patches produced by other people.
\begin{verbatim}
; 64 colour palette file format (internal) - version 1.0
; copyright (c) 2009 Andrew Owen
;
; The palette file is stored as a BASIC program with embedded machine code

header:

db 0x00 ; program file
db 0x14, 0x01, "64colour" ; file name
dw 0x0097 ; data length
dw 0x0000 ; autostart line
dw 0x0097 ; program length

basic:

; 0 RANDOMIZE USR ((PEEK VAL "2
; 3635"+VAL "256"*PEEK VAL "23636"
; )+VAL "48"): LOAD "": REM

db 0x00, 0x00, 0x93, 0x00, 0xf9, 0xc0, 0x28, 0x28
db 0xbe, 0xb0, 0x22, 0x32, 0x33, 0x36, 0x33, 0x35
db 0x22, 0x2b, 0xb0, 0x22, 0x32, 0x35, 0x36, 0x22
db 0x2a, 0xbe, 0xb0, 0x22, 0x32, 0x33, 0x36, 0x33
db 0x36, 0x22, 0x29, 0x2b, 0xb0, 0x22, 0x34, 0x38
db 0x22, 0x29, 0x3a, 0xef, 0x22, 0x22, 0x3a, 0xea

start:

di ; disable interrupts
ld hl, 38 ; HL = length of code
add hl, bc ; BC = entry point (start) from BASIC
ld bc, 0xbf3b ; register select
ld a, 64 ; mode group
out (c), a ;
ld a, 1 ;
ld b, 0xff ; choose register port
out (c), a ; turn palette mode on
xor a ; first register

setreg:

ld b, 0xbf ; choose register port
out (c), a ; select register
ex af, af' ; save current register select
ld a, (hl) ; get data
ld b, 0xff ; choose data port
out (c), a ; set it
ex af, af' ; restore current register
inc hl ; advance pointer
inc a ; increase register
cp 64 ; are we nearly there yet?
jr nz, setreg ; repeat until all 64 have been done
ei ; enable interrupts
ret ; return

; this is where the actual data is stored. The following is an example palette.

registers:

db 0x00, 0x02, 0x18, 0x1b, 0xc0, 0xc3, 0xd8, 0xdb ; INK
db 0x00, 0x02, 0x18, 0x1b, 0xc0, 0xc3, 0xd8, 0xdb ; PAPER
db 0x00, 0x03, 0x1c, 0x1f, 0xe0, 0xe3, 0xfc, 0xff ; +BRIGHT
db 0x00, 0x03, 0x1c, 0x1f, 0xe0, 0xe3, 0xfc, 0xff ;
db 0xdb, 0xd8, 0xc3, 0xc0, 0x1b, 0x18, 0x02, 0x00 ; +FLASH
db 0xdb, 0xd8, 0xc3, 0xc0, 0x1b, 0x18, 0x02, 0x00 ;
db 0xff, 0xfc, 0xe3, 0xe0, 0x1f, 0x1c, 0x03, 0x00 ; +BRIGHT/
db 0xff, 0xfc, 0xe3, 0xe0, 0x1f, 0x1c, 0x03, 0x00 ; +FLASH

terminating_byte:

db 0x0d 
\end{verbatim}

\subsection{Layer 1 Scrolling}
Layer 1 has two sets of scrolling registers. One for the the legacy
modes (ZX Spectrum, Alternate Page, Timex Hi-Resoulution, and Timex
Hi-colour) and a second set for the two ZX Spextrum Next specific
LoRes modes.  All modes scroll as if they were $256\times192$ screens
located at global coordinates (32, 32) to (287, 223), The registers
for the legacy modes are \$26 and \$27 and the registers for the LoRes
modes are \$32 and \$33.

\register{R/W}{26}{ULA Horizontal Scroll Control}
\begin{itemize}
\item bits 7-0 = ULA X Offset (0-255) (0 on reset)
\end{itemize}


\register{R/W}{27}{ULA Vertical Scroll Control}
\begin{itemize}
\item bits 7-0 = ULA Y Offset (0-191) (0 on reset)
\end{itemize}


\register{R/W}{32}{Layer 1,0 (LoRes) Horizontal Scroll Control)}
\begin{itemize}
\item bits 7-0 = X Offset (0-255) (\$00 on reset)
\end{itemize}
Layer 1,0 (LoRes) scrolls in "half-pixels" at the same resolution and
smoothness as Layer 2.


\register{R/W}{33}{Layer 1,0 (LoRes) Vertical Scroll Control)}
\begin{itemize}
\item bits 7-0 = Y Offset (0-191) (\$00 on reset)
\end{itemize}
Layer 1,0 (LoRes) scrolls in "half-pixels" at the same resolution and
smoothness as Layer 2.



\subsection{Layer 1 Clipping}
All of the modes in the Layer 1 share a single clipping register,
\$1A. The clip index may alternately be set using register \$1C. This
is expecially useful for reading the current clipping coordinates as
reads on the clipping register do not change the index. Note that
clipping coordinates are based on a full display area for the mode of
$256\times192$ resolution even though not all modes have that
resolution.

\register{R/W}{1A}{Layer 0 (ULA/LoRes) Clip Window Definition}
\begin{itemize}
\item bits 7-0 = Coord. of the clip window
  \begin{itemize}
  \item[] 1st write = X1 position
  \item[] 2nd write = X2 position
  \item[] 3rd write = Y1 position
  \item[] 4rd write = Y2 position
  \end{itemize}
\end{itemize}
The values are 0,255,0,191 after a Reset\\
Reads do not advance the clip position


\register{R/W}{1C}{Clip Window Control}\\
Read
\begin{itemize}
\item bits 7-6 = Layer 3 Clip Index
\item bits 5-4 = Layer 0/1 Clip Index
\item bits 3-2 = Sprite clip index
\item bits 1-0 = Layer 2 Clip Index
\end{itemize}
Write
\begin{itemize}
\item bits 7-4 = Reserved, must be 0
\item bit 3 - reset Layer 3 clip index
\item bit 2 - reset Layer 0/1 clip index
\item bit 1 - reset sprite clip index.
\item bit 0 - reset Layer 2 clip index.
\end{itemize}



\subsection{ZX Spectrum Mode}

Timex mode 0

This is the default ULA mode and has its origins in the original ZX
Spectrum. It uses $256\times192$ pixels located at global coordinates
(32, 32) to (287, 223) with $8\times8$ colour attribute areas mapped
into a $32\times24$ grid. If Timex modes are not enabled, this and the
LoRes mode are the only ones available, so you would switch back to
this mode by writing 000xxxxx to Next register \$15 (21, the sprites
and layers register). If another Timex mode is enabled, then this is
mode 0 so you would write 0 to port \$ff to enable it. This is a
$256\times192$ video mode. The bitmap 1 area is used for selection
between ink and paper colours with one bit per pixel and the attribute
1 area for colour attributes.

The easiest way to visualize the mapping of this mode is to think of
the $256\times192$ area as being divided into a $32\times24$ grid of
$8\times8$ characters.  IF we consider X and Y as the position in the
grid and R to the the row within the character.  For ink/paper
selection, 0=paper, 1=ink and the entries are stored left to right as
lsb to msb within the bye.  The address for a pixel value is:
$0R_4R_3Y_2Y_1Y_0R_2R_1R_0C_4C_3C_2C_1C_0$. Each $8\times8$ cell has
its own colour attribute where the address for an attribute cell is
$0110R_4R_3R_2R_1R_0C_4C_3C_2C_1C_0$ in other words mapped lineally
column-wise starting at the beginning of the attribute 1 area.

Code:
\begin{verbatim}
  ;; from any other Timex mode:
  ld a,$00
  ld c,$ff
  out (c),a

  ;; from LoRes mode:
  ld bc,$243B ; next register select port
  ld a,$15
  out (c),a
  ld bc,$253B ; next register r/w port
  in a,(c)
  and $7f
  out (c),a
\end{verbatim}

\subsection{Alternate Page Mode}

Timex mode 1

This mode is the same as ZX Spectrum mode except it is at an alternate
addresses. Alternate page mode is selected by enabling Timex modes by
writing 00xxxx1xx to Next register \$08 (8, Peripheral 3 setting) then
writing 1 to the Timex ULA port (\$ff).  It is identical to ZX
Spectrum mode except the pixel are mapped to the bitmap 2 area giving
use pixel addresses of $1R_4R_3Y_2Y_1Y_0R_2R_1R_0C_4C_3C_2C_1C_0$ and
the attributes to the attribute 2 area with addresses of
$1110R_4R_3R_2R_1R_0C_4C_3C_2C_1C_0$.

Code:

\begin{verbatim}
;; disable LoRes mode:
ld bc,$243B ; next register select port
ld a,$15
out (c),a
ld bc,$253B ; next register r/w port
in a,(c)
and $7f
out (c),a
;; set Timex mode
ld bc,$243B ; next register select port
ld a,$08
out (c),a
ld bc,$253B ; next register r/w port
in a,(c)
or $04
out (c),a
;; set alternate page mode
ld c,$ff
ld a,$01
out (c),a
\end{verbatim}

\subsection{Timex Hi-Colour Mode}

Timex mode 2

This mode is a $256\times192$ video mode located at global coordinates
(32, 32) to (287, 223) with $8\times1$ colour attribute mapping on a
$32\times192$ grid. It is selected by writing 2 to the Timex ULA port
(\$ff).  Pixel mapping in this mode is the same as in ZX Spectrum mode
using the bitmap 1 area based on
$0R_4R_3Y_2Y_1Y_0R_2R_1R_0C_4C_3C_2C_1C_0$.  The colour attributes use
the bitmap 2 area with $8\times1$ colour attribute areas corresponding
to the addresses $1R_4R_3Y_2Y_1Y_0R_2R_1R_0C_4C_3C_2C_1C_0$.

Code:
\begin{verbatim}
;; disable LoRes mode:
ld bc,$243B ; next register select port
ld a,$15
out (c),a
ld bc,$253B ; next register r/w port
in a,(c)
and $7f
out (c),a
;; set Timex mode
ld bc,$243B ; next register select port
ld a,$08
out (c),a
ld bc,$253B ; next register r/w port
in a,(c)
or $04
out (c),a
;; set hi-colour mode
ld c,$ff
ld a,$02
out (c),a
\end{verbatim}

\subsection{Timex Hi-Resolution Mode}

Timex mode 6

This is a monochrome $512\times192$ video mode located at global
coordinates (32, 32) to (287, 223) with each pixel being half
width. It is selected by writing to the Timex ULA port (\$ff with
values that also select which two colours (or colour entries in
ULANext mode) you use.

\begin{table}[h]\centering
  \caption{Hi-Resolution Colours}
  \csvautotabular{video/hires.csv}
\end{table}
  
Pixels are mapped into both the bitmap 1 and bitmap 2 areas where
8-pixel wide character columns alternate between the two bitmap areas.
The pixels within a byte being rendered left to right lsb to msb as in
other Spectrum video modes.  The addresses for each row within a
character are based on a $64\times32$ grid of $8\times8$ characters
which using a $64\times24$ R, C, and Y scheme gives us addresses of
the form $C_0R_4R_3Y_2Y_1Y_0R_2R_1R_0C_5C_4C_3C_2C_1$.

Code:
\begin{verbatim}
;; disable LoRes mode:
ld bc,$243B ; next register select port
ld a,$15
out (c),a
ld bc,$253B ; next register r/w port
in a,(c)
and $7f
out (c),a
;; set Timex mode
ld bc,$243B ; next register select port
ld a,$08
out (c),a
ld bc,$253B ; next register r/w port
in a,(c)
or $04
out (c),a
;; set hi-res mode, black on white
ld c,$ff
ld a,$06
out (c),a
\end{verbatim}

\subsection{Lo-Resolution Mode}
This is a Spectrum Next specific video mode with a resolution of
$128\times96$ located at global coordinates (32, 32) to (287, 223)
with each pixel being double height and double width replacing the old
Radistan mode.  It can either allow for 16 colours, in which case it
uses either the bitmap 1 area or the bitmap 2 area, or 256 colours
using both bitmap 1 and bitmap 2. The colour of each pixel can be
selected independently with data ordered linearly in a row major
fashion. In the case of 16 colour mode, the nybbles describing the
colours are X major (MSN LSN). Scrolling is by half pixels and uses
different registers (\$32 and \$33) from the rest of the ULA group
modes. LoRes mode is enabled by writing $100xxxxx$ to Next register
\$15 (the sprites and layers register) with Next register \$6A used to
decide whether it is 16 or 256 colours.

\register{R/W}{15}{Sprite and Layer System Setup}
\begin{itemize}
\item bit 7 = LoRes mode (0 on reset)
\item bit 6 = Sprite priority (1 = sprite 0 on top, 0 = sprite 127 on
  top) (0 on reset)
\item bit 5 = Enable sprite clipping in over border mode (0 on reset)
\item bits 4-2 = set layers priorities (000 on reset)
  \begin{itemize}
  \item 000 - S L U
  \item 001 - L S U
  \item 010 - S U L
  \item 011 - L U S
  \item 100 - U S L
  \item 101 - U L S
  \item 110 - S(U+L) ULA and Layer 2 combined, colours clamped to 7
  \item 111 - S(U+L-5) ULA and Layer 2 combined, colours clamped to [0,7]
  \end{itemize}
\item bit 1 = Enable Sprites Over border (0 on reset)
\item bit 0 = Enable Sprites (0 on reset)
\end{itemize}


\register{R/W}{32}{Layer 1,0 (LoRes) Horizontal Scroll Control)}
\begin{itemize}
\item bits 7-0 = X Offset (0-255) (\$00 on reset)
\end{itemize}
Layer 1,0 (LoRes) scrolls in "half-pixels" at the same resolution and
smoothness as Layer 2.


\register{R/W}{33}{Layer 1,0 (LoRes) Vertical Scroll Control)}
\begin{itemize}
\item bits 7-0 = Y Offset (0-191) (\$00 on reset)
\end{itemize}
Layer 1,0 (LoRes) scrolls in "half-pixels" at the same resolution and
smoothness as Layer 2.


\register{R/W}{6A}{Layer 1,0 (LoRes) Control}
\begin{itemize}
\item bits 7-6 = reserved, must be 0
\item bit 5 = Enable Radistan (16-colour) (0 on reset)
\item bit 4 = Radistan DFILE switch (xor with bit 0 of port \$ff) (0
  on reset)
\item bits 3-0 = Radistsan palette offset (0 on reset)
\item bits 1-0 = ULAplus palette offset (0 on reset)
\end{itemize}



Code: 256 colour
\begin{verbatim}
;; enable LoRes mode:
nextreg $15,$80
;; 256-colour mode
ld bc,$243B ; next register select port
ld a,$6A
out (c),a
ld bc,$253B ; next register r/w port
in a,(c)
and $EF ; lores radistan control
out (c),a
\end{verbatim}

Code: 16 colour
\begin{verbatim}
;; enable LoRes mode:
nextreg $15,$80
;; 16-colour mode
nextreg $6A,$10
\end{verbatim}
