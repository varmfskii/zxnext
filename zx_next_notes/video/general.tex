\section{General Features}
There are a number of control features for the various video modes
that are done in a unified fashion. These features are layering and
transparency, palettes, scrolling, and clipping. For the sake of
convenience we will occasionally talk about a global coordinate system
for graphics on the ZX Next. This coordinate system has (0, 0) at the
upper left corner of the usable display area and (319, 255) at the
lower right corner. Individual pixels generally correspond to integer
locations in this grid, but some modes may either double or halve this
grid. This will be discussed in the sections for each of the video
layers.

\subsection{Video Layering and Transparency}
Video for the ZX Next is composed of a number of features and layers
each of which may have its own set of video modes. Not all of these
features are mandatory.

By composing together the border colour and transparency fallback
color, layer 1 (ULA, Timex modes, or LoRes), layer 2
($256\times192\times256$, $320\times256\times256$, or
$640\times256\times6$), layer 3 (16 or 2 colour tiles), and sprites;
we generate the full video display.

The border/transparency fallback is the bottom with the ordering of
the layers controlled by a combination of the video layering register
(Next register \$15 (21) bits 4-2), the interaction of layers 1 and 3
(Next register \$6B (107) bit 0), and whether or not a pixel in layer
2 is set as a priority colour.

\register{R/W}{15}{Sprite and Layer System Setup}
\begin{itemize}
\item bit 7 = LoRes mode (0 on reset)
\item bit 6 = Sprite priority (1 = sprite 0 on top, 0 = sprite 127 on
  top) (0 on reset)
\item bit 5 = Enable sprite clipping in over border mode (0 on reset)
\item bits 4-2 = set layers priorities (000 on reset)
  \begin{itemize}
  \item 000 - S L U
  \item 001 - L S U
  \item 010 - S U L
  \item 011 - L U S
  \item 100 - U S L
  \item 101 - U L S
  \item 110 - S(U+L) ULA and Layer 2 combined, colours clamped to 7
  \item 111 - S(U+L-5) ULA and Layer 2 combined, colours clamped to [0,7]
  \end{itemize}
\item bit 1 = Enable Sprites Over border (0 on reset)
\item bit 0 = Enable Sprites (0 on reset)
\end{itemize}



\index{transparency}
Transparency for Layer 2, Layer 1, and 1-bit Tilemaps are
controlled by Next register \$14 (20) and defaults to \$E3. Sprites
and 4-bit Tilemaps have their own registers (\$4B and \$4C
respectively) for setting their transparency index (not colour). This
colour ignores the state of the least significant blue bit, so \$E3
equates to both \$1C6 and \$1C7. For Sprites and Tilemaps transparency
is determined by colour index. For Sprites this is controlled by
register \$4B (with only the least significant 4-bits being relevant
for 16-colour Sprites). For Tilemaps, the transparency index is set by
register \$4C. If all layers are transparent, the transparency
fallback colour is displayed. This is set by register \$4A.

\register{R/W}{14}{Global transparency color}
\begin{itemize}
\item bits 7-0 = Transparency color value (\$E3 after a reset)
\end{itemize}
(Note: this value is 8-bit, so the transparency is compared against
only by the MSB bits of the final 9-bit colour)\\
(Note2: this only affects Layer 2, ULA and LoRes. Sprites use register
\$4B for transparency and tilemap uses nextreg \$4C)


\register{R/W}{4A}{Fallback Colour Value}
\begin{itemize}
\item bits 7-0 = 8-bit colour if all layers are transparent (\$E3 on
  reset)
\end{itemize}
(black on reset = 0)


\register{R/W}{4B}{Sprite Transparency Index}
\begin{itemize}
\item bits 7-0 = Index value (\$E3 if reset)
\end{itemize}
For 4-bit sprites only the bottom 4-bits are relevant.


\register{R/W}{4C}{Level 3 Transparency Index}
\begin{itemize}
\item bits 7-4 = Reserved, must be 0
\item bits 3-0 = Index value (\$0F on reset)
\end{itemize}



\subsection{Palette}
\index{palette}
\paragraph{Next Colour Palettes}
Each video mode group has a pair of palettes assigned to it a primary
and an alternate palette. Each palette entry is actually a 9-bit value
(RRRGGGBBB) and can be set by selecting a palette using nextreg \$43
(palette control), the entry using nextreg \$40 (palette index), then
writing the value into nextreg \$44 (palette value, 9-bit) using pairs
of consecutive writes for each palette value or nextreg \$41 (palette
value, 8-bit). Once a palette index has been selected writes
automatically increment the palette index number so it is possible to
efficiently write the values for a collection of palette entries.

\register{R/W}{40}{Palette Index Select}
\begin{itemize}
\item bits 7-0 = Palette Index Number
\end{itemize}
Selects the palette index to change the associated colour

For ULA only, INKs are mapped to indices 0 through 7, BRIGHT INKs to
indices 8 through 15, PAPERs to indices 16 through 23 and BRIGHT
PAPERs to indices 24 through 31.  In EnhancedULA mode, INKs come from
a subset of indices from 0 through 127 and PAPERs from a subset of
indices from 128 through 255.

The number of active indices depends on the number of attribute bits
assigned to INK and PAPER out of the attribute byte.

In ULAplus mode, the last 64 entries (indices 192 to 255) hold the
ULAplus palette.  The ULA always takes border colour from PAPER for
standard ULA and Enhanced ULA


\register{R/W}{41}{8-bit Palette Data}
\begin{itemize}
\item bits 7-0 = Colour Entry in RRRGGGBB format
\end{itemize}
The lower blue bit of the 9-bit internal colour will be the logical or
of bits 0 and 1 of the 8-bit entry. After each write, the palette
index auto-increments if aut-increment has been enabled (NextReg \$43
bit 7), Reads do not auto-increment.


\register{R/W}{43}{Palette Control}
\begin{itemize}
\item bit 7 = Disable palette write auto-increment.
\item bits 6-4 = Select palette for reading or writing:
  \begin{itemize}
  \item 000 = ULA first palette
  \item 100 = ULA second palette
  \item 001 = Layer 2 first palette
  \item 101 = Layer 2 second palette
  \item 010 = Sprite first palette
  \item 110 = Sprite second palette
  \item 011 = Layer 3 first palette
  \item 111 = Layer 3 second palette
  \end{itemize}
\item bit 3 = Select Sprite palette (0 = first palette, 1 = second
  palette)
\item bit 2 = Select Layer 2 palette (0 = first palette, 1 = second
  palette)
\item bit 1 = Select ULA palette (0 = first palette, 1 = second
  palette)
\item bit 0 = Enable EnhancedULA mode if 1. (0 after a reset)
\end{itemize}


\register{R/W}{44}{9-bit Palette Data}\\
Non Level 2
\begin{itemize}
\item[] 1st write
\item bits 7-0 = MSB (RRRGGGBB)
\item[] 2nd write
\item bits 7-1 = Reserved, must be 0
\item bit 0 = LSB (B)
\end{itemize}
Level 2
\begin{itemize}
\item[] 1st write
\item bits 7-0 = MSB (RRRGGGBB)
\item[] 2nd write
\item bit 7 = Priority
\item bits 6-1 = Reserved, must be 0
\item bit 0 = LSB (B)
\end{itemize}
9-bit Palette Data is entered in two consecutive writes; the second
write autoincrements the palette index if auto-increment is enabled in
NextREG \$43 bit 7

If writing an L2 palette, the second write's D7 holds the L2 priority
bit which if set (1) brings the colour defined at that index on top of
all other layers. If you also need the same colour in regular priority
(for example: for enviromental masking) you will have to set it up
again, this time with no priority.

Reads return the second byte and do not autoincrement. Writes to
nextreg \$40, \$41, \$41, or \$43 reset to the first write.



\subsection{Scrolling}
The ZX Spectrum Next has four sets of scrolling registers to
independently contol the display offsets of various video modes
(Layer2, ULA, Tilemap, and LoRes). When the video is offset, the
portion that is pushed off the screen (to the left and or top) then
becomes visible on the opposite side of the screen so that the video
offset values are effectively the coordinates of the origin in a
toroidal universe.

\subsection{Clipping}
The ZX Spectrum Next has four clipping registers create a window of
the layer that is visible. Clipping is managed by a set of four
successive writes to the clipping register applicable for the video
mode. If a section is masked off by clipping, it is as if the area
were the transparency colour and the video lyers behind it become
visible.
