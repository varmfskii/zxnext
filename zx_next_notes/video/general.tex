\section{General Features}
There are a number of control features for the various video modes
that are done in a unified fashion. These features are layering and
transparency, palettes, scrolling, and clipping. 

\subsection{Video Layering and Transparency}
Video is rendered as three layers which are referred to as ULA (which
includes the tilemap), layer 2, and sprites.  The ordering of the
layers is controlled by Next port \$15 (21) bits 4-2:

\begin{table}[h]\centering
  \caption{Video Layering}
  \csvautotabular{video/videolayering.csv}
\end{table}

\index{transparency}
Transparency for Layer 2, ULA, LoRes, and 1-bit Tilemaps are
controlled by Next register \$14 (20) and defaults to \$E3. Sprites
and 4-bit Tilemaps have their own registers (\$4B and \$4C
respectively) for setting their transparency index (not colour). This
colour ignores the state of the least significant blue bit, so \$E3
equates to both \$1C6 and \$1C7. For Sprites and Tilemaps transparency
is determined by colour index. For Sprites this is controlled by
register \$4B (with only the least significant 4-bits being relevant
for 16-colour Sprites). For Tilemaps, the transparency index is set by
register \$4C. If all layers are transparent, the transparency
fallback colour is displayed. This is set by register \$4A.

(R/W) \$14 (20) $\Rightarrow$ Global Transparency Colour
\begin{itemize}
\item bits 7-0 = 8-bit transparency colour (soft reset = \$E3)
\end{itemize}

(R/W) \$4A (74) $\Rightarrow$ Fallback colour
\begin{itemize}
\item bits 7-0 = 8-bit colour used if all layers are transparent (soft
  reset = 0)
\end{itemize}

(R/W) \$4B (75) $\Rightarrow$ Sprite transparency index
\begin{itemize}
\item bits 7-0 = Sprite colour index treated as transparent (soft
  reset = \$E3)
\item[] For 4-bit sprites only the bottom 4-bits are used
\end{itemize}

(R/W) \$4C (74) $\Rightarrow$ Tilemap transparency index
\begin{itemize}
\item bits 7-4 = Reserved, must be 0
\item bits 3-0 = Tilemap colour index treated as transparent (soft
  reset \$f)
\end{itemize}

\subsection{Palette}
\index{palette}
\paragraph{Next Colour Palettes}
Each video mode group has a pair of palettes assigned to it a primary
and an alternate palette. Each palette entry is actually a 9-bit value
(RRRGGGBBB) and can be set by selecting a palette using nextreg \$43
(palette control), the entry using nextreg \$40 (palette index), then
writing the value into nextreg \$44 (palette value, 9-bit) using pairs
of consecutive writes for each palette value or nextreg \$41 (palette
value, 8-bit). Once a palette index has been selected writes
automatically increment the palette index number so it is possible to
efficiently write the values for a collection of palette entries.

(R/W) \$40 (64) $\Rightarrow$ Palette Index
\begin{itemize}
\item bits 7-0 = Select the palette index to change the associated
  colour. (soft reset = 0)
\item[] For the ULA only, INKs are mapped to indices 0-7, Bright INKS
  to indices 8-15, PAPERs to indices 16-23 and Bright PAPERs to
  indices 24-31.  In ULANext mode, INKs come from a subset of indices
  0-127 and PAPERs come from a subset of indices 128-255.  The number
  of active indices depends on the number of attribute bits assigned
  to INK and PAPER out of the attribute byte.  In ULA+ mode, the top
  64 entries hold the ula+ palette.  The ULA always takes border
  colour from paper for standard ULA and ULAnext.
\end{itemize}

(R/W) \$41 (65) $\Rightarrow$ Palette Value (8 bit colour)
\begin{itemize}
\item bits 7-0 = Colour for the palette index selected by nextreg
  \$40.
\item[] The format is RRRGGGBB - the lower blue bit of the 9-bit
  colour will be the logical OR of blue bits 1 and 0 of this 8-bit
  value.
\item[] After the write, the palette index is auto-incremented to the
  next index if the auto-increment is enabled in nextreg \$43.  Reads
  do not auto-increment.  Any other bits associated with the index
  will be zeroed.
\end{itemize}

(R/W) \$43 (67) $\Rightarrow$ Palette Control
\begin{itemize}
\item bit 7 = Disable palette write auto-increment (soft reset = 0)
\item bits 6-4 = Select palette for reading or writing (soft reset =
  000)
  \begin{itemize}
  \item 000 = ULA first palette
  \item 100 = ULA second palette
  \item 001 = Layer 2 first palette
  \item 101 = Layer 2 second palette
  \item 010 = Sprites first palette 
  \item 110 = Sprites second palette
  \item 011 = Tilemap first palette
  \item 111 = Tilemap second palette
  \end{itemize}
\item bit 3 = Select Sprites palette (0 = first palette, 1 = second
  palette) (soft reset = 0)
\item bit 2 = Select Layer 2 palette (0 = first palette, 1 = second
  palette) (soft reset = 0)
\item bit 1 = Select ULA palette (0 = first palette, 1 = second
  palette) (soft reset = 0)
\item bit 0 = Enable ULANext mode (soft reset = 0)
\end{itemize}

(R/W) \$44 (68) $\Rightarrow$ Palette Value (9 bit colour)
\begin{itemize}
\item[] Two consecutive writes are needed to write the 9 bit colour
\item[] 1st write:
\item bits 7-0 = RRRGGGBB
\item[] 2nd write:
\item bits 7-1 = Reserved, must be 0
\item bit 0 = lsb B
\item[] If writing to an L2 palette
\item bit 7 = 1 for L2 priority colour, 0 for normal.
\item[] An L2 priority colour moves L2 above all layers.  If you need
  the same colour in both priority and normal modes, you will need to
  have two different entries with the same colour one with and one
  without priority.  If auto-increment is enabled in nextreg \$43, the
  palette index is auto-incremented after two consecutive writes
\item[] Reads only return the 2nd byte and do not auto-increment.
\item[] can also be read back from nextreg \$28
\end{itemize}

\subsection{Scrolling}
The ZX Spectrum Next has four sets of scrolling registers to
independently contol the display offsets of various video modes
(Layer2, ULA, Tilemap, and LoRes). When the video is offset, the
portion that is pushed off the screen (to the left and or top) then
becomes visible on the opposite side of the screen so that the video
offset values are effectively the coordinates of the origin in a
toroidal universe.

\subsection{Clipping}
The ZX Spectrum Next has four clipping registers create a window of
the layer that is visible. Clipping is managed by a set of four
successive writes to the clipping register applicable for the video
mode. If a section is masked off by clipping, it is as if the area
were the transparency colour and the video lyers behind it become
visible.
