\chapter{Interrupts}
The Z80 has three different hardware interrupt signals:
\textoverline{RESET}, \textoverline{NMI}, and \textoverline{INT}.

\textoverline{RESET} is used to return the CPU to a known state. When
the \textoverline{RESET} line is pulled low, a \textoverline{RESET} is
generated. The CPU then does several things. I, and R are set to \$00.
PC is set to \$0000. SP becomes \$FFFF. A and F are set to \$FF.  The
interrupt mode is set to 0. And (maskable) interrupts are disabled by
clearing IFF1 and IFF2.

\textoverline{NMI} is the non-maskable interrupt. Upon receiving a
non-maskable interrupt (\textoverline{NMI} being pulled low) one of
two sequences occur depending on the calue of bit 3 of the interrupt
control register (nextreg \$C0).

\register{R/W}{C0}{INT SIG 0} (3.01.08)\\
(\$00 on reset)
\begin{itemize}
\item bit 7 = Expansion bus INT
\item bits 6-2 = Reserved, must be 0
\item bit 1 = Line
\item bit 0 = ULA
\end{itemize}
* Set bits indicate the corresponding hardware generated an interrupt\\
* Set bits during writes clear corresponding bits\\
* Expansion bus INT operated independently of this register and is read only



If bit 3 is clear (0) PC is pushed on the stack, IFF1 is copied to
IFF2, IFF1 is cleared (inhibiting maskable interrupts). The
\textoverline{NMI} should end with RETN which copies the contents of
IFF2 to IFF1 (returning the interrupt state to what it was before the
\textoverline{NMI}) and PC is popped off the stack.

If bit 3 is set (1) PC is stored in the NMI return address registers
(nextregs \$C2 and \$C3), IFF1 is copied to IFF2, IFF1 is cleared
(inhibiting maskable interrupts). The \textoverline{NMI} should end
with RETN which copies the contents of IFF2 to IFF1 (returning the
interrupt state to what it was before the \textoverline{NMI}) and PC
is compied from the NMI return address registers.

\register{R/W}{C2}{INT EN 0} (3.01.08)\\
(\$83 on reset)
\begin{itemize}
\item bit 7 = Expansion bus INT
\item bits 6-2 = Reserved, must be 0
\item bit 1 = Line
\item bit 0 = ULA
\end{itemize}
* Set bits indicate the corresponding signals will generate an interrupt


\register{R/W}{C3}{NMI Return Address MSB} (3.01.09)\\
(\$00 on reset)


The interrupt generally of most interest to programmers is
\textoverline{INT}. So much so that if programmers talk about
interrupts on the Z80, they are probebly only talking about
\textoverline{INT}. The processing of \textoverline{INT} is controlled
by IFF1 and IFF2 which are set using EI to enable interrupts and reset
using DI to disable interrupts. Interrupts can happen at any time and
should preserve register contents.  If none of your code uses the
alternate registers the EXX and EX AF,AF’ instructions can make this
faster and easier. Interrupt routined should end with EI and RETI to
reenable interrupts, potentially inform the interrupting device that
its interrupt has been serviced, and return from the interrupt
routine. In general the Spectrum machines do not make any distingtion
between RET and RETI, but future developments in the ZX Spectrum Next
may make the distinction important.

The ZX Spectrum Next has 14 internal sources for \textoverline{INT}
signals. This can be enabled and disabled using nextregs \$C4 --
\$C6. Which signals have been received can be read/cleared using
nexregs \$C8 -- \$CA.

Interrupt Enable
\register{R/W}{C4}{Interrupt Enable 0} (3.01.08)\\
(\$83 on reset)
\begin{itemize}
\item bit 7 = Expansion bus \textoverline{INT}
\item bits 6-2 = Reserved must be zero
\item bit 1 = Line
\item bit 0 = ULA
\end{itemize}

\register{R/W}{C5}{INT TYP 1} (3.01.08)\\
(\$00 on reset)
\begin{itemize}
\item bit 7 = ctc channel 7 zc/to
\item bit 6 = ctc channel 6 zc/to
\item bit 5 = ctc channel 5 zc/to
\item bit 4 = ctc channel 4 zc/to
\item bit 3 = ctc channel 3 zc/to
\item bit 2 = ctc channel 2 zc/to
\item bit 1 = ctc channel 1 zc/to
\item bit 0 = ctc channel 0 zc/to
\end{itemize}
* Set bits indicate the corresponding signals are sticky\\
* Reset bits indicate the corresponding signals assert for about 32 CPU cycles

\register{R/W}{C6}{Interrupt Enable 2} (3.01.08)\\
(\$00 on reset)
\begin{itemize}
\item bit 7 = Reserved, must be 0
\item bit 6 = UART1 Tx empty
\item bit 5 = UART1 Rx half full *
\item bit 4 = UART1 Rx available *
\item bit 3 = Reserved, must be 0
\item bit 2 = UART0 Tx empty
\item bit 1 = UART0 Rx half full *
\item bit 0 = UART0 Rx available *
\end{itemize}
* For each UART, Rx half full and Rx available are shared interrupts


Interupt Status
\register{R/W}{C8}{Interrupt Status 0} (3.01.09)\\
(\$00 on reset)
\begin{itemize}
\item bits 7-2 = Reserved, must be zero
\item bit 1 = Line
\item bit 0 = ULA
\end{itemize}
* Set bits indicate the device generated an interrupt in the past
* Writes clear bits where bits are set except in IM2 mode




\register{R/W}{C9}{INT TYP 2} (3.01.08)\\
(\$00 on reset)
\begin{itemize}
\item bit 7 = Reserved, must be zero
\item bit 6 = UART1 Tx empty
\item bit 5 = UART1 Rx almost full
\item bit 4 = UART1 Rx available
\item bit 3 = Reserved must be zero
\item bit 2 = UART0 Tx empty
\item bit 1 = UART0 Rx almost full
\item bit 0 = UART0 Rx available
\end{itemize}
* Set bits indicate the corresponding signals are sticky\\
* Reset bits indicate the corresponding signals assert for about 32 CPU cycles

\register{R/W}{CA}{Interrupt Status 2 (3.01.09)}
(\$00 on reset)
\begin{itemize}
\item bit 7 = Reserved, must be zero
\item bit 6 = UART1 Tx empty
\item bit 5 = UART1 Rx almost full *
\item bit 4 = UART1 Rx available *
\item bit 3 = Reserved must be zero
\item bit 2 = UART0 Tx empty
\item bit 1 = UART0 Rx almost full *
\item bit 0 = UART0 Rx available *
\end{itemize}
* For each UART Rx half full and Rx available are shared interrupts
** Set bits indicate the device generated an interrupt in the past
** Writes clear bits where bits are set except in IM2 mode



Internal Interrupt Sources
\begin{itemize}
\item[] 0 = Line (highest priority)
\item[] 1 = UART 0 Rx
\item[] 2 = UART 1 Rx
\item[] 3-10 = CTC channels 0-7
\item[] 11 = ULA
\item[] 12 = UART 0 Tx
\item[] 13 = UART 1 Tx (lowest priority)
\end{itemize}

\paragraph{IM0}
When an interrupt is received by the CPU it disables interrupts and
executes the instruction placed on the bus by the interrupting device
and (no known use on the Next) It is enabled with the IM0 instruction
and enabling interrupts (EI).

\paragraph{IM1}
When an interrupt is received, the CPU disables interrupts and jumps
to an interrupt handler at \$0038 (normally in ROM). The ROM interrupt
handler updates the frame counter and scans the keyboard. This is the
default interrupt handling method for the ZX Spectrum and is probably
the method to use if you don’t need the ROMs for anything. It is
enabled using the IM1 instruction and enabling interrupts.

\paragraph{IM2}
The ZX Spectrum Next has both a legacy method for handling IM2 and an
updated one which makes better use of the capabilities of IM2 which
was added in Core 3.01.09.

The ZX Spectrum Next has 14 interrupt devices which can all be given
independent interrupt vectors when using IM2. These interrupts are
controlled by nextregs \$C0 -- \$CF.  The address of the vector for a
given interrupt is created by composing the I register (bits 15-0),
nextreg \$C0 bits 7-5 (bits 7-5) and the interrupt number of the
interrupt device (bits 4-1). This means that even if you use all 14
internal interrupt sources, your interrupt vector table is no more
than 28 bytes which can be at any 32 byte boundry. It also means that
far less processing has to be done on interrupts which are
received. External interrupts are a little different. If no vector is
supplied by a device the implied LSB will be \$FF.

While in IM2 mode, it is possible for interrupts to interrupt DMA
transfers. This capability is controlled by The DMA Interrupt enable
registers (nextregs \$CC -- \$CE). When DMA is interrupted, one
instruction of the main program will be processed, then the interrupt
will be taken. On return, DMA will continue.

\register{R/W}{CC}{DMA Interrupt Enable 0 (3.01.09)}
(\$00 on reset)
\begin{itemize}
\item bits 7-2 = Reserved, must be 0
\item bit 1 = Line
\item bit 0 = ULA
\end{itemize}
* Set bits indicate the specified interrupt will interrupt a DMA
  operation when in IM2 mode


\register{R/W}{CD}{DMA Interrupt Enable 1 (3.01.09)}
(\$00 on reset)
\begin{itemize}
\item bit 7 = CTC channel 7 zc/to
\item bit 6 = CTC channel 6 zc/to
\item bit 5 = CTC channel 5 zc/to
\item bit 4 = CTC channel 4 zc/to
\item bit 3 = CTC channel 3 zc/to
\item bit 2 = CTC channel 2 zc/to
\item bit 1 = CTC channel 1 zc/to
\item bit 0 = CTC channel 0 zc/to
\end{itemize}
* Set bits indicate the corresponding interrupt will interrupt a DMA
  operation when in IM2 mode


\register{R/W}{CE}{DMA Interrupt Enable 2 (3.01.09)}
(\$00 on reset)
\begin{itemize}
\item bit 7 = Reserved, must be 0
\item bit 6 = UART1 Tx empty
\item bit 5 = UART1 Rx half full
\item bit 4 = UART1 Rx available
\item bit 3 = Reserved, must be 0
\item bit 2 = UART0 Tx empty
\item bit 1 = UART0 Tx half full
\item bit 0 = UART0 Tx available
\end{itemize}
* Set bits indicate the corresponding interrupt will interrupt a DMA
  operation when in IM2 mode.
  


In legacy mode, when the CPU receives an interrupt it disables
interrupts and jumps to an interrupt routine starting at the contents
of the jump table at I. The start of the interrupt routine is the
contents of I*\$100+bus and I*\$100+bus+1. Most devices that can
supply interrupts on the ZX Spectrum leave the data bus in a floating
state.  As a result the interpreted state of the data bus while
generally \$FF is not entirely predictable.  The solution to place
your interrupt routine at an address where the MSB and LSB are the
same (\$0101, \$0202, … \$FFFF) then place 257 copies of that value in
a block starting at I*\$100 (you can set the value of the I register).

Code:
\begin{verbatim}
;; my program
org $8000
;; enable interrupt mode im2
ld i,$fe
im2
ei
;; program body
;; interrupt routine
handler:
;; preserve registers used
;; handle interrupt
;; restore registers
ei
reti
;; jump to interrupt routine
org $fdfd
jp handler
;; im2 jump table
org $fe00 ; not actually legal
defs $101,$fd
\end{verbatim}

\chapter{Z80 CTC}
(3.01.08) Untested, assuming it acts like two Z80 CTCs.

Eight independent CTC channels are available on ports 0x183B through
0x1F3B.  These perform counter/timer functions that can be used to
generate timer interrupts or to generate interrupts from physical
signals.

The CTC is a standard Zilog part.  Its datasheet can be found at
http://www.zilog.com/docs/z80/ps0181.pdf .  The Zilog documentation is
ambiguous around how soft resets are treated so the following
clarifies some points in the Next's implementation.

\begin{enumerate}
\item Hard reset requires both a control word and a time constant to
  be written to a channel even if bit 2 = 0 in the first control word.
\item Soft reset with bit 2 = 0 causes the entire control register to
  be modified.  Soft reset with bit 2 = 1 does not change the control
  register contents.  In both cases a time constant must follow to
  resume operation.
\item Changing the trigger edge selection in bit 4 while the channel
  is in operation counts as a clock edge.  A pending timer trigger
  will be fired and, in counter mode, an edge will be received.
\item ZC/TO is asserted for one clock cycle and not for the entire
  duration that the count is at zero.
\end{enumerate}
At the moment, any interrupt generated by the CTC will assert the
z80's /INT line for 32 cpu cycles.  This is the same way that the ULA
and line interrupts operate.

At the moment, the ZC/TO output of each channel is fed into the
CLK/TRG input of the succeeding channel so that time and count periods
can be cascaded.

\section{Programming}
Initial values are set by a write of a channel control word followed
by a time constant. In timer mode, the counter decrements every time
it is triggered. In counter mode it decrements every time the
prescaler counter reaches zero.

Channel Control Word
\begin{itemize}
\item[] bit 7 = Enable Interrupt
\item[] bit 6 = Mode
  \begin{itemize}
  \item 0 = Timer mode
  \item 1 = Counter mode
  \end{itemize}
\item[] bit 5 = Prescalar value (Timer mode only)
  \begin{itemize}
  \item 0 = 16
  \item 1 = 256
  \end{itemize}
\item[] bit 4 = CLK/TRG edge selection
  \begin{itemize}
  \item 0 = Falling Edge
  \item 1 = Rising Edge
  \end{itemize}
\item[] bit 3 = Timer Trigger (Timer mode only)
  \begin{itemize}
  \item 0 = Starts on loading of time constant
  \item 1 = Starts on CLK/TRG
  \end{itemize}
\item[] bit 2 = Time constant follows
\item[] bit 1 = Software reset
\item[] bit 0 = 0 (Control Word)
\end{itemize}

If we are running at 28MHz (Mode 0) and wish to trigger an interrupt
every 1 sec, that is 28 million T-States/cycles we could program CTC 5
as a counter with a prescalar of 16 and a period of 175, CTC 6 as a
counter with a prescalar of 16 and a period of 125, and CTC 7 as a
timer with a period of 5.

CTC 5 triggers ZC5 every 280 cycles or 10 usec.

CTC 6 triggers ZC6 every 560,000 cycles or 20 msec.

CTC 7 triggers ZC7 and an interrupt every 28,000,000 cycles or 1 sec.

\begin{verbatim}
di
; set up interrupt routine
im 2
ld bc,$183B     ; CTC 0
ld hl,$FFFA     ; address pointing to start of interrupt routine
ld de,interrupt ; start of interrupt routine
ld (hl),de
ld i,$FF
ld a,l          ; Vector to address at on this interrupt
out (c),a
; set up CTC 5
ld b,$1D        ; CTC 5
ld a,$87
out (c),a       ; Interrupt mode, timer mode, time constant, soft, control
ld a,$05
out (c),a       ; once every 5 times 
; set up CTC 6
inc b           ; CTC 6
ld a,$47
out (c),a       ; 16x, counter mode, time constant, soft, control
ld a,$7D        ; 125 = once every 2000 times
out (c),a
; set up CTC 7
inc b           ; CTC 7
ld a,$47
out (c),a       ; 16x, counter mode, time constant, soft, control
ld a,$AF        ; 175 = once every 2800 times
ei
\end{verbatim}
  

