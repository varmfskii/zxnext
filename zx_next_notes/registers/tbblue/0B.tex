\register{R/W}{0B}{Joystick I/O Mode}
\begin{itemize}
\item bit 7 = 1 to enable i/o mode
\item bit 6 = Reserved, must be 0
\item bits 5-4 = I/O Mode
  \begin{itemize}
  \item 00 = bit bang
  \item 01 = clock
  \item 10 = uart on left joystick port
  \item 11 = uart on right joystick port
  \end{itemize}
\item bits 3-1 = Reserved, must be 0
\item bit 0 = Parameter
  \begin{itemize}
  \item[] bit bang : copied to pin 7
  \item[] clock
    \begin{itemize}
    \item 0 = hold high when clock becomes high
    \item 1 = run *
    \end{itemize}
  \item[] uart
    \begin{itemize}
    \item 0 = redirect esp uart0 to joystick
    \item 1 = redirect pi uart1 to joystick\\
      (Tx out on pin 7, Rx in from pin 9, CTS\_n in from pin 6 **)
    \end{itemize}
  \end{itemize}
\end{itemize}
The state of output pin 7 is stored internally in a register and is
retained across changing modes and while i/o mode is disabled.  While
in i/o mode, keyboard joystick types (Sinclair, Cursor, etc) produce
no readings but the current state of pins can still be read via the
Kempston ports.  When leaving i/o mode, joystick operation resumes
after ~64 scan lines have passed.\\
* CTC channel 3 is currently used to drive pin 7 in clock mode.  Freq
= Fctc3 / 2.\\
** CTS\_n is only active if the seleced uart is in hw flow control mode.
