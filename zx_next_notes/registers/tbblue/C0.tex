\register{R/W}{C0}{Interrupt Control} (3.01.09)\\
(\$00 on reset)
\begin{itemize}
\item bits 7-5 = Programmable portion of IM2 vector *
\item bit 4 = Reserved, must be 0
\item bit 3 = Enable stackless \textoverline{NMI} response**
\item bits 2-1 = Reserved, must be 0
\item bit 0 = Maskable interrupt mode
\begin{itemize}
\item[] 0 - pulse
\item[] 1 - IM2
\end{itemize}
\end{itemize}
* In IM2 mode vector generated is:
\begin{itemize}
\item bits 7-5 = nextreg \$C0 bits 7-5
\item bits 4-1 = Interrupt source
\begin{itemize}
\item[] 0 - line interrupt (highest priority)
\item[] 1 - UART 0 Rx
\item[] 2 - UART 1 Rx
\item[] 3--10 - CTC channels 0-7
\item[] 11 - ULA
\item[] 12 - UART 0 Tx
\item[] 13 - UART 1 Tx (lowest priority)
\end{itemize}
\item bit 0 = 0
\end{itemize}

* In IM2 mode the expansion bus is the lowest priority interrupter and
  if no vector is supplied externally the \$FF is generated.

** The return address pushed during an nmi acknowledge cycle will not
   be written to memory (stack pointer will be decremented) and the
   RETN after acknowledge will take the return address from nextreg
   instead of memory (the stack pointer will be incremented). If bit 3
   = 0 and in other circumstances, RETN functions normally.


