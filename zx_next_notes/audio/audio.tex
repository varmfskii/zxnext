\chapter{Audio}
\section{ZX Spectrum 1-bit}
\begin{multicols}{2}
The baseline sound of the ZX Spectrum was produced by toggling the Ear
bit (bit 4) of \$fe (254) The ULA port to produce 1-bit audio.  It is
enabled by bit 4 of Next register \$08 (8).  While this does work on
the ZX Spectrum Next, there are other much better methods and this is
only supported for backward compatibility.

\sinset
Code:
\begin{verbatim}
;; enable internal speaker
ld bc,$243B
ld a,$08
out (c),a
ld bc,$253B
in a,(c)
or $10
out (c),a
\end{verbatim}
\einset

\section{Sampled 8-bit}

The ZX Next has four 8-bit D/A audio channels connected to provide
sampled stereo sound. Channels A and B are the left channels, while C
and D are the right channels. In order use 8-bit sound, it must first
be enabled by setting bit 3 on nextreg \$08. In order to emulate
legacy hardware there are a number of ports that can be used to
control the four channels additionally these are mirrored to three
nextregs to enable driving audio using the copper.  Channel A is
mapped to ports \$0f, \$3f, and \$f1; channel B to ports \$1f and \$f3
and nextreg \$2C; channel C to ports \$4f, and \$f9 and nextreg \$2E;
and channel D to: \$5f and \$fb; with port \$df connected to both
channel A and C and nextreg \$2D connected to both channel A and D.

\sinset
Code:
\begin{verbatim}
;; enable SpecDrum/Convox audio
ld bc,$243B
ld a,$08
out (c),a
ld bc,$253B
in a,(c)
or $08
out (c),a
\end{verbatim}
\einset

\section{Turbosound}

TurboSound consists of the implementation of three AY-3-8912 chips. To
enable TurboSound set bit 1 of Next Register \$08 (8). Once enabled
the sound chips and registers of the sound chips are selected using
port \$fffd (65533) TurboSound Next Control while the registers are
accessed using \$bffd () Sound Chip Register Access.  To enable access
to a particular chip write 111111xx to the control register where
01=AY1, 10=AY2, and 11=AY3.  Access to particular registers of the
selected chip is selected by writing the register number to the
control register. You can then access a chip register using the access
port.

\sinset
Code:
\begin{verbatim}
;; enable TurboSound audio
ld bc,$243B
ld a,$08
out (c),a
ld bc,$253B
in a,(c)
or $02
out (c),a
\end{verbatim}
\einset

Each of the three AY chips has three channels, A, B, and C whose
mapping is controlled by bit 5 of Next register 0x08 (8).

\register{R/W}{00}{Channel A fine tune}


\register{R/W}{01}{Channel A coarse tune (4 bits)}


\register{R/W}{02}{Reset}\\
Read
\begin{itemize}
\item bit 7 = Expansion bus \textoverline{RESET} Asserted
\item bits 6-4 = Reserved
\item bit 3 = Indicates multiface NMI was generated by this nextreg (3.01.09)
\item bit 2 = Indicates divmmc NMI was generated by this nextreg (3.01.09)
\item bit 1 = Last reset was Hard reset
\item bit 0 = Last reset was Soft reset
\end{itemize}
Write
\begin{itemize}
\item bit 7 = Hold Expansion bus and ESP \textoverline{RESET}
\item bits 6-4 = Reserved, must be 0
\item bit 3 = Generate multiface NMI (write zero to clear)(3.01.09)
\item bit 2 = Generate divmmc NMI (write zero to clear)(3.01.09)
\item bit 1 = generate Hard reset (reboot)
\item bit 0 = generate Soft reset
\end{itemize}


\register{R/W}{03}{Machine Type}\\
A write to this register disables the boot rom\\
bits 2-0 select machine type when in config mode
\begin{itemize}
\item bit 7 = (W) Display Timing change enable (allow changes to
  bits 6-4) (0 on hard reset)
\item bits 6-4 = Display Timing
\item bit 3 = Display Timing user lock control
  \item[] Read
  \begin{itemize}
  \item 0 = No user lock on display timing
  \item 1 = User lock on display timing
  \end{itemize}
  \item[] Write
  \begin{itemize}
  \item 1 = Apply user lock on display timing (0 on hard reset)
  \end{itemize}
\item bits 2-0 = Machine Type (config mode only)\\
determines roms loaded
\item[] Machine Types/Display Timings
  \begin{itemize}
  \item 000 or 001 = ZX 48K
  \item 010 = ZX 128K/+2 (Grey)
  \item 011 = ZX +2A-B/+3e/Next Native
  \item 100 = Pentagon 128K
  \end{itemize}
\end{itemize}


\register{W}{04}{Configuration Mapping}
\begin{itemize}
\item bits 7 = Reserved, must be 0
\item bits 6-0 = 16k SRAM bank mapping* (\$00 on hard reset)
\item[] * Maps a 16k SRAM bank over the bottom 16k. Applies only in
  config mode when the bootrom is disabled
\item[] ** Odd multiples of 256k are unreliable if storing data in sram
  for the mext core started.
\item[] *** number of useful bits changed from 5 to 7 in coure 3.01.06
\end{itemize}


\register{R/W}{05}{Peripheral 1 Settings}
\begin{itemize}
\item bits 7-6 = joystick 1 mode (MSB)
\item bits 5-4 = joystick 2 mode (MSB)
\item bit 3 = joystick 1 mode (LSB)
\item bit 2 = 50/60 Hz mode (0 = 50Hz, 1 = 60Hz)
\item bit 1 = joystick 2 mode (LSB)
\item bit 0 = Enable Scandoubler
\end{itemize}
Joystick modes
\begin{itemize}
\item 000 = Sinclair 2 (67890)
\item 001 = Kempston 2 (port \$37)
\item 010 = Kempston 1 (port \$1F)
\item 011 = Megadrive 1 (port \$1F)
\item 100 = Cursor
\item 101 = Megadrive 2 (port \$37)
\item 110 = Sinclair 1 (12345)
\item 111 = I/O Mode
Both joysticks are places in I/O Mode if either is set to I/O
Mode. The underlying joystick type is not changed and reads of this
register will continue to return the last joystick type. Ehether the
joystick is in io mode or not is invisible but this state can be
cleared either through reset or by re-writing the gegister with
joystick type not equal to 111. Recovery time for a normal joystick
read after leaving I/O Mode is at most 64 scan lines.
\end{itemize}


\register{R/W}{06}{Peripheral 2 Settings}
\begin{itemize}
\item bit 7 = F8 CPU Speed Hotkey Enable (1 on reset)
\item bit 6 = DMA mode (0 = zxnDMA, 1 = Z80DMA) (0 on hard reset)
\item bit 5 = F3 50Hz/60Hz Hotkey Enable (1 on reset)
\item bit 4 = divMMC Automap/NMI Enable (0 on hard reset)
\item bit 3 = NMI Button Enable (0 on hard reset)
\item bit 2 = PS/2 Mode (0 = keyboard, 1 = mouse)
\item bits 1-0 = PSG Mode (00 = YM, 01 = AY, 11 = hold all PSGs in Reset)
\end{itemize}


\register{R/W}{07}{Tone Enable}
\begin{itemize}
\item bit 5 = Channel C tone enable (0=enable, 1=disable)
\item bit 4 = Channel B tone enable (0=enable, 1=disable)
\item bit 3 = Channel A tone enable (0=enable, 1=disable)
\item bit 2 = Channel C noise enable (0=enable, 1=disable)
\item bit 1 = Channel B noise enable (0=enable, 1=disable)
\item bit 0 = Channel A noise enable (0=enable, 1=disable)
\end{itemize}


\register{R/W}{08}{Peripheral 3 Settings}
\begin{itemize}
\item bit 7 = 128K Banking Unlock (inverse of port \$7FFD, bit 5) (0
  on reset)
\item bit 6 = Disable RAM and Port Contention (0 on reset)
\item bit 5 = PSG Stereo Mode Control (0 = ABC, 1 = ACB) (0 on hard
  reset)
\item bit 4 = Enable internal speaker (1 on hard reset)
\item bit 3 = Enable DACs (0 on hard reset)
\item bit 2 = Enable read of port \$FF (Timex) (0 on hard reset)
\item bit 1 = Enable Multiple PSGs (0 on hard reset)
\item bit 0 = Enable Issue 2 Keyboard
\end{itemize}


\register{R/W}{09}{Peripheral 4 setting:}
\begin{itemize}
\item bit 7 = PSG 2 Mono Enable (0 on hard reset)
\item bit 6 = PSG 1 Mono Enable (0 on hard reset)
\item bit 5 = PSG 0 Mono Enable (0 on hard reset)
\item bit 4 = Sprite ID lockstep enable (1 = Nextreg \$34 and IO Port
  \$303B are in lockstep, 0 on reset)
\item bit 3 = divMMC mapRAM bit Control (reset bit 7 of port \$E3)
\item bit 2 = HDMI audio mute (0 on hard reset)
\item bits 1-0 = scanlines
  \begin{itemize}
  \item 00 = scanlines off
  \item 01 = scanlines 75\%
  \item 10 = scanlines 50\%
  \item 11 = scanlines 25\%
  \end{itemize}
\item[] In Sprite lockstep, NextREG \$34 and Port \$303B are in
  lockstep
\end{itemize}


\register{R/W}{0A}{Peripheral 5 setting:}
\begin{itemize}
\item bits 7-6 = Multiface type (00 on hard reset)
  \begin{itemize}
  \item 00 = Multiface +3 (enable port 0x3F, disable port 0xBF)
  \item 01 = Multiface 128 v87.2 (enable port 0xBF, disable port 0x3F)
  \item 10 = Multiface 128 v87.12 (enable port 0x9F, disable port 0x1F)
  \item 11 = Multiface 1 (enable port 0x9F, disable port 0x1F)
  \end{itemize}
\item bits 5-4 = Reserved, must be zero
\item bit 3 = 1 to reverse left and right mouse buttons
\item bits 1-0 = mouse dpi (00 on hard reset)
  \begin{itemize}
  \item 00 = low dpi
  \item 01 = default
  \item 10 = medium dpi
  \item 11 = high dpi
  \end{itemize}
\end{itemize}


\port{0B}{DMA Control (Z80 Mode, 3.01.02)}



\register{R/W}{0C}{Envelope period coarse}


\register{R/W}{0D}{Envelope shape}
\begin{itemize}
\item bit 3 = Continue
  \begin{itemize}
  \item 0 = drop to amplitude 0 after 1 cycle
  \item 1 = use ‘Hold’ value
  \end{itemize}
\item bit 2 = Attack
  \begin{itemize}
  \item 0 = generator counts down
  \item 1 = generator counts up
  \end{itemize}
\item bit 1 = Alternate\\
  hold = 0
  \begin{itemize}
  \item 0 = generator resets after each cycle
  \item 1=generator reverses direction each cycle
  \end{itemize}
  hold=1
  \begin{itemize}
  \item 0 = hold final value
  \item 1 = hold initial value
  \end{itemize}
\item bit 0 = Hold
  \begin{itemize}
  \item 0 = cycle continuously
  \item 1 = perform one cycle and hold
  \end{itemize}
\end{itemize}



\subsection{Pi Audio}
If connected the Pi Zero is configured to use the ZX Next as a
soundcard over an \iis interface making the Raspberry Pi a fully
configurable audio source for the ZX Spectrum Next.
\end{multicols}
