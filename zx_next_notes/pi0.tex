\chapter{Raspberry Pi0 Acceleration}
The Spectrum Next has a header (with male pins) which can be attached
to a Raspberry Pi Zero. There is a modified version of DietPi called
NextPi which is the standard distro for the Raspberry Pi0
accelerator. Software for the general public should be written
assuming that it will be interfacing with a Pi0 running this distro.

If you are more adventurous, you may choose to use another distro, or
even another accelerator that uses the Raspberry Pi style (40 pin)
expansion bus.  Chief concers when doing this is that you have a
console presented on the UART that defaults to 115,200 bps, you don't
need to login, the machine is configured with a driver to treat the
\iis interface as a sound card, and the presence of the nextpi
scripts.

The Raspberry Pi 0 has a Broadcom BCM2835 SoC with an ARMv6 core, a
Videocore 4 GPU, and its own 512 MB memory and HDMI output. It has its
own SD card from which it boots. For this application the Pi 0 ships
with a 1GB microSD card containing NextPi a customized version of
DietPi.

The Pi Zero, if installed, is a smart peripheral for the ZX Spectrum
Next. Available interfaces are: low level access to the GPIO pins,
higher level access to standardized I/O interfaces, and use of the Pi
Zero as a sound card.

When using the low level GPIO interface Pi Zero GPIO pins 2-27 can be
configured as either inputs or outputs using nextregs \$90-\$93. If
they are outputs, the output state can be set by writing to nextregs
\$98-\$9b. The current status of the GPIO pins can be read from
nextregs \$98-\$9b whether it is the state driven by the ZX Spectrum
Next or the state drive by some other peripherial attached to the bus
(normally the Raspberry Pi Zero).

\register{R/W}{90}{Pi GPIO output enable 1/4}
\begin{itemize}
\item bit 7 = Enable Pin 7 (0 on reset)
\item bit 6 = Enable Pin 6 (0 on reset)
\item bit 5 = Enable Pin 5 (0 on reset)
\item bit 4 = Enable Pin 4 (0 on reset)
\item bit 3 = Enable Pin 3 (0 on reset)
\item bit 2 = Enable Pin 2 (0 on reset)
\item bit 1 = Enable Pin 1 (cannot be enabled) (0 on reset)
\item bit 0 = Enable Pin 0 (cannot be enabled) (0 on reset)
\end{itemize}


\register{R/W}{91}{Pi GPIO output enable 2/4}
\begin{itemize}
\item bit 7 = Enable Pin 15 (0 on reset)
\item bit 6 = Enable Pin 14 (0 on reset)
\item bit 5 = Enable Pin 13 (0 on reset)
\item bit 4 = Enable Pin 12 (0 on reset)
\item bit 3 = Enable Pin 11 (0 on reset)
\item bit 2 = Enable Pin 10 (0 on reset)
\item bit 1 = Enable Pin 9 (0 on reset)
\item bit 0 = Enable Pin 8 (0 on reset)
\end{itemize}


\register{R/W}{92}{Pi GPIO output enable 3/4}
\begin{itemize}
\item bit 7 = Enable Pin 23 (0 on reset)
\item bit 6 = Enable Pin 22 (0 on reset)
\item bit 5 = Enable Pin 21 (0 on reset)
\item bit 4 = Enable Pin 20 (0 on reset)
\item bit 3 = Enable Pin 19 (0 on reset)
\item bit 2 = Enable Pin 18 (0 on reset)
\item bit 1 = Enable Pin 17 (0 on reset)
\item bit 0 = Enable Pin 16 (0 on reset)
\end{itemize}


\register{R/W}{93}{Pi GPIO output enable 4/4}
\begin{itemize}
\item bits 7-4 = Reserved
\item bit 3 = Enable Pin 27 (0 on reset)
\item bit 2 = Enable Pin 26 (0 on reset)
\item bit 1 = Enable Pin 25 (0 on reset)
\item bit 0 = Enable Pin 24 (0 on reset)
\end{itemize}


\register{R/W}{98}{Pi GPIO Pin State 1/4}
\begin{itemize}
\item bit 7 = Pin 7 Data (1 on reset)
\item bit 6 = Pin 6 Data (1 on reset)
\item bit 5 = Pin 5 Data (1 on reset)
\item bit 4 = Pin 4 Data (1 on reset)
\item bit 3 = Pin 3 Data (1 on reset)
\item bit 2 = Pin 2 Data (1 on reset)
\item bit 1 = Pin 1 Data (1 on reset)
\item bit 0 = Pin 0 Data (1 on reset)
\end{itemize}


\register{R/W}{99}{Pi GPIO Pin State 2/4}
\begin{itemize}
\item bit 7 = Pin 15 Data (1 on reset)
\item bit 6 = Pin 14 Data (1 on reset)
\item bit 5 = Pin 13 Data (1 on reset)
\item bit 4 = Pin 12 Data (1 on reset)
\item bit 3 = Pin 11 Data (1 on reset)
\item bit 2 = Pin 10 Data (1 on reset)
\item bit 1 = Pin 9 Data (1 on reset)
\item bit 0 = Pin 8 Data (1 on reset)
\end{itemize}


\register{R/W}{9A}{Pi GPIO Pin State 3/4}
\begin{itemize}
\item bit 7 = Pin 23 Data (1 on reset)
\item bit 6 = Pin 22 Data (1 on reset)
\item bit 5 = Pin 21 Data (1 on reset)
\item bit 4 = Pin 20 Data (1 on reset)
\item bit 3 = Pin 19 Data (1 on reset)
\item bit 2 = Pin 18 Data (1 on reset)
\item bit 1 = Pin 17 Data (1 on reset)
\item bit 0 = Pin 16 Data (1 on reset)
\end{itemize}


\register{R/W}{9B}{Pi GPIO Pin State 4/4}
\begin{itemize}
\item bits 7-4 = Reserved
\item bit 3 = Pin 27 Data (1 on reset)
\item bit 2 = Pin 26 Data (1 on reset)
\item bit 1 = Pin 25 Data (1 on reset)
\item bit 0 = Pin 24 Data (1 on reset)
\end{itemize}



Standardized I/O access with the Pi Zero can use the \iic, SPI, or
UART interfaces and is configured using nextreg \$a0. Any enabled port
will disable low level (write) access to the corresponding GPIO
pins.

\register{R/W}{A0}{Pi Peripheral Enable}
\begin{itemize}
\item bits 7-6 = Reserved, must be 0
\item bit 5 = Enable UART on GPIO 14, 15 (0 on reset)*
\item bit 4 = Communication Type (0 on reset)
  \begin{itemize}
  \item 0 = Rx to GPIO 15, Tx to GPIO 14 (Pi)
  \item 1 = Rx to GPIO 14, Tx to GPIO 15 (Pi Hats)
  \end{itemize}
\item bit 3 = Enable \iic on GPIO 2, 3 (0 on reset)*
\item bits 2-1 = Reserved, must be 0
\item bit 0 = Enable SPI on GPIO 7, 8, 9, 10, 11 (0 on reset)*
\end{itemize}
*Overrides GPIO Enables



The \iic interface is controlled using ports \$103b (SCL) and \$113b
(SDA). This is the same \iic interface that is used for the optional
Real Time Clock. Interfacing with the Pi Zero over \iic is
complicated by the fact that it is a master/slave interface, but both
the ZX Spectrum Next and Pi Zero are configured to be bus masters.

\port{103B}{\iic SCL (rtc, rpi)}


\port{113B}{\iic SDA (rtc, rpi)}



The SPI interface is controlled using ports \$e7 (/CS) and \$eb
(/DATA). The SPI interface is shared between the SD card(s), the flash
memory, and the Pi Zero. Interfacing with the Pi Zero over SPI is
complicated by the fact it is a master/slave interface and both the ZX
Spectrum Next and Pi Zero are configured to be bus masters.

\port{E7}{SPI \textoverline{CS} (SD card, flash, rpi)}\\
Disable with bit 2 of Nextreg \$09


\port{EB}{SPI \textoverline{DATA} (SD card, flash, rpi)}\\
Disable with bit 2 of Nextreg \$09



The default means of communication between the ZX Next and the Pi is
through the UART interface (see serial communications chapter). In
order to communicate withe the Pi the Pi UART must be connected to the
Pi by setting nextreg \$a0 bits 5 and 4 to 1, selecting the Pi UART by
setting port \$153b bit 6 to 1 and ensuring that both ends are using
matching communication protocols (by default 115,200 bps, 8N1 and no
flow control). On the Pi end the UART is connected to the serial
console.

\begin{verbatim}
;; enable UART connection with Pi Zero
   ld c,$3b
   ld b,$15 ; UART control
;; select Pi on UART control
   in a,(c)
   or $40
   out (c),a
   ld b,$24 ; Next Register Select
   ld a,$a0
   out (c),a
   inc b ; Next Register Data
;; Enable UART on GPIO and select Pi
   in a,(c)
   or $30
   out (c),a
\end{verbatim}

The \iis sound interface between the ZX Spectrum Next and the Pi Zero
is controlled by nextregs \$a2 and \$a3. Normally, one would control
the Pi through some other channel such as the UART recieve audio from
the Pi to either use as a fulloy programmable sound card or to allow
loading of tape files on the ZX Spectrum Next.

\register{R/W}{A2}{Pi \iis Audio Control}
\begin{itemize}
\item bits 7-6 = \iis State (\$00 on reset)
  \begin{itemize}
  \item 00 = \iis Disabled
  \item 01 = \iis is mono, source R
  \item 10 = \iis is mono, source L
  \item 11 = \iis is stereo
  \end{itemize}
\item bit 5 = Reserved, must be 0
\item bit 4 = Audio Flow Direction (0 on reset)
  \begin{itemize}
  \item 0 = PCM\_DOUT to Pi, PCM\_DIN from Pi (Hats)
  \item 1 = PCM\_DOUT from Pi, PCM\_DIN to Pi (Pi)
  \end{itemize}
\item bit 3 = Mute left (0 on reset)
\item bit 2 = Mute right (0 on reset)
\item bit 1 = Reserved must be 1 (3.01.05)
\item bit 0 = Direct \iis audio to EAR on port \$FE (0 on reset)
\end{itemize}


\register{R/W}{A3}{Pi \iis Clock Divide (Master Mode)}
\begin{itemize}
\item bits 7-0 = Clock divide value (\$0B on reset)
\end{itemize}
$\hbox{Divider}=\frac{538461}{\hbox{Rate}}-1$ or $\hbox{Rate}=\frac{538461}{\hbox{Divider}+1}$


