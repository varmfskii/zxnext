\section{Video}

ZX Spectrum Next video splits the display types into four categories
(ULA, tilemap, layer2, and sprites) which have their own sets of
controls for colour palettes, clipping, and scrolling. Some aspects of
ULA and tilemap are tied together, but all the rest operate in a
largely independent manner using a layering system. The ULA category
has a number of seperate video modes that it can use. One of these
(LoRes) is incompatible with using tilemaps.

\subsection{Video Layering and Transparency}
Video is rendered as three layers which are referred to as ULA (which
includes the tilemap), layer 2, and sprites.  The ordering of the
layers is controlled by Next port \$15 (21) bits 4-2:

\begin{table}[h]\centering
  \caption{Video Layering}
  \csvautotabular{videolayering.csv}
\end{table}

Transparency for Layer2, ULA, and LoRes are controlled by Next
register \$14 (20) and defaults to \$E3. This colour ignores the state
of the least significant blue bit, so \$E3 equates to both \$1C6 and
\$1C7. For Sprites and Tilemaps transparency is determined by colour
index. For Sprites this is controlled by register \$4B (with only the
leas significant 4-bits being relevant for 16-colour Sprites). For
Tilemaps, the transparency index is set by register \$4C. If all
layers are transparent, the transparency fallback colour is
displayed. This is set by register \$4A.

\subsection{Colour Palette}

The ZX Spectrum Next has 8 palettes, two each for ULA, tilemap,
layer2, and sprites.  There is a palette in use and an alternate that
can be quickly swapped.  Each palette consists of 256 entries out of a
total of 512 (RRRGGGBBB) colours.  Register \$43 (67, Palette Control)
is used to select which palette can be read/written and whether each
layer is using its primary or secondary palette and whether or not
writes to the palette value registers auto-increment. Palette values
are written to registers \$41 (65, 8-bit) and \$44 (70, 9-bit)

(R/W) \$40 (64) $\Rightarrow$ Palette Index
\begin{itemize}
\item bits 7-0 = Select the palette index to change the associated colour.
For the ULA only, INKs are mapped to indices 0-7, Bright INKS to
indices 8-15, PAPERs to indices 16-23 and Bright PAPERs to indices
24-31.  In ULANext mode, INKs come from a subset of indices 0-127 and
PAPERs come from a subset of indices 128-255.  The number of active
indices depends on the number of attribute bits assigned to INK and
PAPER out of the attribute byte.  The ULA always takes border colour
from paper.
\end{itemize}
(R/W) \$41 (65) $\Rightarrow$ Palette Value (8 bit colour)
\begin{itemize}
\item bits 7-0 = Colour for the palette index selected by the register \$40.
(Format is RRRGGGBB - the lower blue bit of the 9-bit colour will be a
logical OR of blue bits 1 and 0 of this 8-bit value.)  After the
write, the palette index is auto-incremented to the next index if the
auto-increment is enabled at reg \$43.  Reads do not auto-increment.
\end{itemize}
(R/W) \$43 (67) $\Rightarrow$ Palette Control
\begin{itemize}
\item bit 7 = '1' to disable palette write auto-increment.
\item bits 6-4 = Select palette for reading or writing:
  \begin{itemize}
  \item 000 = ULA first palette
  \item 100 = ULA second palette
  \item 001 = Layer 2 first palette
  \item 101 = Layer 2 second palette
  \item 010 = Sprites first palette 
  \item 110 = Sprites second palette
  \item 011 = Tilemap first palette
  \item 111 = Tilemap second palette
  \end{itemize}
\item bit 3 = Select Sprites palette (0 = first palette, 1 = second palette)
\item bit 2 = Select Layer 2 palette (0 = first palette, 1 = second palette)
\item bit 1 = Select ULA palette (0 = first palette, 1 = second palette)
\item bit 0 = Enable ULANext mode if 1. (0 after a reset)
\end{itemize}
(R/W) \$44 (68) $\Rightarrow$ Palette Value (9 bit colour)

Two consecutive writes are needed to write the 9 bit colour

1st write:

\begin{itemize}
\item bits 7-0 = RRRGGGBB
\end{itemize}

2nd write: 

If writing a L2 palette

\hrulefill
\begin{itemize}
\item bit 7 = 1 for L2 priority colour, 0 for normal priority colour will always be on top even on an SLU priority arrangement. 

If you need the exact same colour on priority and non priority
locations you will need to program the same colour twice changing bit
7 to 0 for the second colour bits 6-1 = Reserved, must be 0

\item bit 0 = lsb B
\end{itemize}
     
If writing another palette

\hrulefill
\begin{itemize}
\item bits 7-1 = Reserved, must be 0
\item bit 0 = lsb B
\end{itemize}           
After the two consecutive writes the palette index is auto-incremented
if the auto-increment is enabled by reg \$43.
     
Reads only return the 2nd byte and do not auto-increment.

\subsection{Clipping}

There are two clipping modes: normal clipping and over border
clipping. With normal clipping the clipping coordinates are relative
to the regular $256\times192$ video area, so 0,0 is the display area origin,
X range is 0-255, and Y range is 0-191. With over border clipping, the
ULA origin is now located so the clipping coordinate 32,32 is the
display area origin, the X range is 0-320, but values given for X1 and
X2 are doubled, the Y range is now 0-255.

ULA and Layer2 always use normal clipping.  Tilemaps always use over
border clipping, and Using bit 5 of register \$15 selects which
clipping mode is used for sprites (0=normal, 1=over border).

Each group as its own register to control clipping (\$18=Layer2,
\$19=Sprites, \$1A=ULA, and \$1B=Tilemap) with the clipping area
controlled by four consecutive writes in the order X1, X2, Y1,
Y2. Where you are in this sequence can be controlled by either using
register \$1C to reset the next write to control X1 (bit 3=tilemap,
2=ULA, 1=sprite, and 0=layer2) or register \$1D to set all of the
indices (bits 7-6=tilemap, 5-4=layer 2, 3-2=sprite, and 1-0=ULA).

\subsection{Scrolling}

On the ZX Spectrum Next, all of the screen area (less clipping) is
visible at any given time. Scrolling controls where the screen origin
is located with your image being mapped onto a toroidal coordinate
system. Each group that is scrollable is controlled its won set of
registers.

Layer2 scrolling is controlled by registers \$16 (X) and \$17 (Y). The
legal range is 0-255 for X and 0-191 for Y.

ULA is controlled by registers \$32 (X-offset) and \$33
(Y-offset). Legacy ULA modes scroll horizontally in 8-pixel increments
(ignoring the least significant 3-bits) and vertically in 1-pixel
increments while LoRes scrolls in half-pixel increments. The legal
range is 0-255 for X and 0-191 for Y.

Tilemap scrolling uses three registers. \$2F (X bit 8), \$30 (X bits
7-0), \$31 (Y). The legal range is 0-319 for X and 0-255 for Y.

\subsection{ULA group}

The ULA layer supports ZX Spectrum video, Timex video modes, and the
Spectrum Next’s lores video mode all use 16k memory bank 5 or 7 with
the data coming from some combination of addresses \$0000-\$17FF
(bitmap 1), \$1800-\$1AFF (attribute 1), \$2000-\$37FF (bitmap 2), and
\$3800-\$3AFF (attribute 2) within the selected bank.  Assuming
default memory mapping and the use of bank 5 this will be mapped as
some combination of memory \$4000-\$57FF, \$5800-\$5AFF,
\$6000-\$77FF, \$780-\$7AFF. All of the modes other than the lores
mode can either use the default ZX Spectrum colours, or ULANext mode
which uses a 256 entry palette to determine the colour. In the
Spectrum and Timex modes all colours are either Paper (foreground),
paper (background), or border colours.

\subsubsection{Colour Attributes}

The ZX Spectrum Next has two major modes for colour attributes
allowing a total of nine ways to map the palette to the original ZX
Spectrum attributes. One with flashing enabled and eight with flashing
disabled. This mapping is controlled by Next registers \$42 (66,
palette format) and \$43 (67, palette control).  Palette control
switches between flashing enabled (0, default) and flashing disabled
(1)

\begin{table}[h]\centering
  \caption{Flashing Enabled}
  \csvautotabular{flash.csv}
\end{table}

Flashing Enabled is similar to the original Spectrum colour attributes:

The ULA next modes use a varying number of bits from the attribute
byte to determine the ink colour as the palette index from the
appropriate bits (all others being zero) and the paper colours coming
from the indicated value+128 with palette format 255 being a special
case where all the bits determine the ink colour while the paper is
always palette index 128.

\begin{table}[h]\centering
  \caption{ULA Next}
  \csvautotabular{palfmt.csv}
\end{table}

\subsubsection{ZX Spectrum Mode}

Timex mode 0

This is the default ULA mode and has its origins in the original ZX
Spectrum. It uses $256\times192$ pixels with $8\times8$ colour
attribute areas mapped into a $32\times24$ grid. If Timex modes are
not enabled, this and the LoRes mode are the only ones available, so
you would switch back to this mode by writing 000xxxxx to Next
register \$15 (21, the sprites and layers register). If anothe Timex
mode is enabled, then this is mode 0 so you would write 0 to port \$ff
to enable it. This is a $256\times192$ video mode. The bitmap 1 area
is used for selection between ink and paper colours with one bit per
pixel and the attribute 1 area for colour attributes.

The easiest way to visualize the mapping of this mode is to think of
the $256\times192$ area as being divided into a $32\times24$ grid of
$8\times8$ characters.  IF we consider X and Y as the position in the
grid and R to the the row within the character.  For ink/paper
selection, 0=paper, 1=ink and the entries are stored left to right as
lsb to msb within the bye.  The address for a pixel value is:
$0R_4R_3Y_2Y_1Y_0R_2R_1R_0C_4C_3C_2C_1C_0$. Each $8\times8$ cell has
its own colour attribute where the address for an attribute cell is
$0110R_4R_3R_2R_1R_0C_4C_3C_2C_1C_0$ in other words mapped lineally
column-wise starting at the beginning of the attribute 1 area.

Code:
\begin{verbatim}
  ;; from any other Timex mode:
  ld a,$00
  ld c,$ff
  out (c),a

  ;; from LoRes mode:
  ld bc,$243B ; next register select port
  ld a,$15
  out (c),a
  ld bc,$253B ; next register r/w port
  in a,(c)
  and $7f
  out (c),a
\end{verbatim}

\subsubsection{Alternate Page Mode}

Timex mode 1

This mode is the same as ZX Spectrum mode at alternate
addresses. Alternate page mode is selected by enabling Timex modes by
writing 00xxxx1xx to Next register \$08 (8, Peripheral 3 setting) then
writing 1 to the Timex ULA port (\$ff).  It is identical to ZX
Spectrum mode except the pixel are mapped to the bitmap 2 area giving
use pixel addresses of $1R_4R_3Y_2Y_1Y_0R_2R_1R_0C_4C_3C_2C_1C_0$ and
the attributes to the attribute 2 area with addresses of
$1110R_4R_3R_2R_1R_0C_4C_3C_2C_1C_0$.

Code:

\begin{verbatim}
;; disable LoRes mode:
ld bc,$243B ; next register select port
ld a,$15
out (c),a
ld bc,$253B ; next register r/w port
in a,(c)
and $7f
out (c),a
;; set Timex mode
ld bc,$243B ; next register select port
ld a,$08
out (c),a
ld bc,$253B ; next register r/w port
in a,(c)
or $04
out (c),a
;; set alternate page mode
ld c,$ff
ld a,$01
out (c),a
\end{verbatim}

\subsubsection{Timex Hi-Colour Mode}

Timex mode 2

This mode is a $256\times192$ video mode with $8\times1$ colour
attribute mapping on a $32\times192$ grid. It is selected by writing 2
to the Timex ULA port (\$ff).  Pixel mapping in this mode is the same
as in ZX Spectrum mode using the bitmap 1 area based on
$0R_4R_3Y_2Y_1Y_0R_2R_1R_0C_4C_3C_2C_1C_0$.  The colour attributes use
the bitmap 2 area with $8\times1$ colour attribute areas corresponding
to the addresses $1R_4R_3Y_2Y_1Y_0R_2R_1R_0C_4C_3C_2C_1C_0$.

Code:
\begin{verbatim}
;; disable LoRes mode:
ld bc,$243B ; next register select port
ld a,$15
out (c),a
ld bc,$253B ; next register r/w port
in a,(c)
and $7f
out (c),a
;; set Timex mode
ld bc,$243B ; next register select port
ld a,$08
out (c),a
ld bc,$253B ; next register r/w port
in a,(c)
or $04
out (c),a
;; set hi-colour mode
ld c,$ff
ld a,$02
out (c),a
\end{verbatim}

\subsubsection{Timex Hi-Resolution Mode}

Timex mode 6

This is a monochrome $512\times192$ video mode. It is selected by
writing to the Timex ULA port (\$ff with values that also select which
two colours (or colour entries in ULANext mode) you use.

\begin{table}[h]\centering
  \caption{Hi-Resolution Colours}
  \csvautotabular{hires.csv}
\end{table}
  
Pixels are mapped into both the bitmap 1 and bitmap 2 areas where
8-pixel wide character columns alternate between the two bitmap areas.
The pixels within a byte being rendered left to right lsb to msb as in
other Spectrum video modes.  The addresses for each row within a
character are based on a $64\times32$ grid of $8\times8$ characters
which using a $64\times24$ R, C, and Y scheme gives us addresses of
the form $C_0R_4R_3Y_2Y_1Y_0R_2R_1R_0C_5C_4C_3C_2C_1$.

Code:
\begin{verbatim}
;; disable LoRes mode:
ld bc,$243B ; next register select port
ld a,$15
out (c),a
ld bc,$253B ; next register r/w port
in a,(c)
and $7f
out (c),a
;; set Timex mode
ld bc,$243B ; next register select port
ld a,$08
out (c),a
ld bc,$253B ; next register r/w port
in a,(c)
or $04
out (c),a
;; set hi-res mode, black on white
ld c,$ff
ld a,$06
out (c),a
\end{verbatim}

\subsubsection{Lo-Resolution Mode}

This is a Spectrum Next specific video mode with a resolution of
$128\times96$ replacing the old Radistan mode.  It allows for
independent selection from the 256 entries in the ULA palette on a
pixel by pixel basis. The pixel data is mapped into the bitmap 1 and
bitmap 2 areas.  It is selected by writing $100xxxxx$ to Next register
\$15 (21, the sprites and layers register).  Each byte corresponds to
a ULA palette entry and bytes are mapped linearly in a row-wise
fashion.

Code:
\begin{verbatim}
;; enable LoRes mode:
ld bc,$243B ; next register select port
ld a,$15
out (c),a
ld bc,$253B ; next register r/w port
in a,(c)
or $80
out (c),a
\end{verbatim}

\subsection{Tilemap}

The tilemap is a hardware character oriented display that comes in two
resolutions: $40\times32$ ($320\times256$ pixels) and $80\times32$
($640\times256$ pixels).  The display area on screen is the same as
the sprite layer, meaning it overlaps the standard $256\times192$ area
by 32 pixels on all sides.  Vertically this is larger than the
physical 60Hz HDMI display which will cut off the top and bottom
character rows (making the visible area $40\times30$ or $80\times30$)
but the full area is visible on VGA or on 50Hz HDMI.

The tilemap is defined by two data structures.

\paragraph{Tilemap Data Structure}

The first data structure is the tilemap itself which indicates what
characters occupy each cell on screen.  Each tilemap entry is either
one or two bytes according to the mode setting so for $40\times32$
resolution, a full size tilemap will occupy either 1280 or 2560 bytes,
and for $80\times32$ resolution the space taken is twice that at 2560
or 5120 bytes.  The tilemap entries are stored in row major order. One
byte entries are just the tile number while two byte entries have the
tile number followed by an attribute by with the format:

\begin{itemize}
\item bits 7-4 = palette offset
\item bit 3 = X mirror
\item bit 2 = Y mirror
\item bit 1 = rotate
\item bit 0 = ULA over tilemap (if the ula is disabled, bit 8 of tile number)
\end{itemize}

The character displayed is indicated by the "tile number" which can be
thought of as an ASCII code.  The tile number is normally eight bits
allowing up to 256 unique tiles to be defined but this can be extended
to nine bits for 512 unique tiles if the ULA display is disabled.

The other bits are tile attributes that modify how the tile image is
drawn.  Their function is the same as the equivalent sprite attributes
for sprites.  Bits will apply rotation and mirroring and colour can be
shifted with a palette offset.  If the ULA display is enabled, bit 8
will determine if the tile is above or below the ULA display.

\paragraph{Tile Definition Data Structure}

The second data structure is the tile definitions themselves.

Each tile, identified by tile number, is $8\times8$ pixels in size
with each pixel four bits to select one of 16 colours.  A tile
definition occupies 32 bytes and is defined in row major order with
packing in the X direction in the same way that 4-bit sprites are
defined.  The 4-bit colour of each pixel is augmented by the 4-bit
palette offset from the tilemap in the most significant bits to form
an 8-bit colour index that is looked up in the tilemap palette to
determine the final 9-bit colour sent to the display.  Tiles are
therefore defined using 16 colours with the tilemap palette offset
able to act as index into the tilemap palette to vary the display
colour.

\paragraph{Memory Organization and Display Layer}

The tilemap is a logical extension of the ULA and its data structures
are contained in the ULA's 16k bank 5.  If both the ULA and tilemap
are enabled, this means that the tilemap's map and tile definitions
should be arranged within the 16k to avoid overlap with the display
ram used by the ULA.

The tilemap exists on the same display layer as the ULA. The graphics
generated by the ULA and tilemap are combined before being forwarded
to the SLU layer system as layer U.

\paragraph{ULA and Tilemap Combination}
The combination of the ULA and tilemap output is done in one of two modes: 
the standard mode and the stencil mode.

The standard mode uses bit 8 of a tile's tilemap entry to determine if
a tile is above or below the ULA. The source of the final pixel
generated is then the topmost non-transparent pixel.  If the ULA or
tilemap is disabled then they are treated as transparent.

The stencil mode will only be applied if both the ULA and tilemap are
enabled. In the stencil mode, the final pixel will be transparent if
either the ULA or tilemap are transparent. Otherwise the final pixel
is a logical AND of the corresponding colour bits. The stencil mode
allows one layer to act as a cut-out stencil for the other.

\paragraph{Programming the Tilemap}
(R/W) \$6B (107) $Rightarrow$ Tilemap Control
\begin{itemize}
\item bit 7 = 1 to enable the tilemap
\item bit 6 = 0 for $40\times32$, 1 for $80\times32$
\item bit 5 = 1 to eliminate the attribute entry in the tilemap
\item bit 4 = palette select
\item bits 3-0 = Reserved set to 0
\end{itemize}

Bits 7 \& 6 enable the tilemap and select resolution.  Bit 4 selects
one of two tilemap palettes used for final colour lookup.  Bit 5
changes the structure of the tilemap so that it contains only 8-bit
tilemap entries instead of 16-bit tilemap entries.  If 8-bit, the
tilemap only contains tile numbers and the attributes are instead
taken from nextreg \$6C.

(R/W) \$6C (108) $\Rightarrow$ Default Tilemap Attribute
\begin{itemize}
\item bits 7-4 = Palette Offset
\item bit 3 = X mirror
\item bit 2 = Y mirror
\item bit 1 = Rotate
\item bit 0 = ULA over tilemap (bit 8 of tile id if the ULA is disabled)
\end{itemize}

If bit 5 of nextreg \$6B is set, the tilemap's structure is modified
to contain only 8-bit tile numbers instead of the usual 16-bit tilemap
entries.  In this case, the tile attributes used are taken from this
register instead.

(R/W) \$6E (110) $\Rightarrow$ Tilemap Base Address
\begin{itemize}
\item bits 7-6 = Read back as zero, write values ignored
\item bits 5-0 = MSB of address of the tilemap in Bank 5
\end{itemize}

This register determines the tilemap's base address in bank 5.  The
base address is the MSB of an offset into the 16k bank, allowing the
tilemap to begin at any multiple of 256 bytes in the bank.  Writing a
physical MSB address in \$40-\$7f or \$c0-\$ff, corresponding to
traditional ULA physical addresses, is permitted.  The value read back
should be treated as a fully significant 8-bit value.

The tilemap will be $40\times32$ or $80\times32$ in size depending on
the resolution selected in nextreg \$6B.  Each entry in the tilemap is
one or two bytes depending on the setting of bit 5 of nextreg \$6B.

(R/W) \$6F (111) $\Rightarrow$ Tile Definitions Base Address
\begin{itemize}
\item bits 7-6 = Read back as zero, write values ignored
\item bits 5-0 = MSB of address of tile definitions in Bank 5
\end{itemize}

This register determines the base address of tile definitions in bank
5.  As with nextreg \$6E, the base address is the MSB of the an offset
into the 16k bank, allowing tile definitions to begin at any multiple
of 256 bytes in the bank.  Writing a physical MSB address in \$40-\$7f
or \$c0-\$ff, corresponding to traditional ULA physical addresses, is
permitted.  The value read back should be treated as a fully
significant 8-bit value.

Each tile definition is 32 bytes in size and is located at address:

  Tile Def Base Addr + 32 * (Tile Number)

(R/W) \$4C (76) $\Rightarrow$ Transparency index for the tilemap
\begin{itemize}
\item bits 7-4 = Reserved, must be 0
\item bits 3-0 = Set the index value (\$F after reset)
\end{itemize}

Defines the transparent colour index for tiles.  The 4-bit pixels of a tile definition are compared to this value to determine if they are transparent.

(R/W) \$43 (67) $\Rightarrow$ Palette Control
\begin{itemize}
\item bit 7 = '1' to disable palette write auto-increment.
\item bits 6-4 = Select palette for reading or writing:
  \begin{itemize}
  \item 000 = ULA first palette
  \item 100 = ULA second palette
  \item 001 = Layer 2 first palette
  \item 101 = Layer 2 second palette
  \item 010 = Sprites first palette 
  \item 110 = Sprites second palette
  \item 011 = Tilemap first palette
  \item 111 = Tilemap second palette
  \end{itemize}
\item bit 3 = Select Sprites palette (0 = first palette, 1 = second palette)
\item bit 2 = Select Layer 2 palette (0 = first palette, 1 = second palette)
\item bit 1 = Select ULA palette (0 = first palette, 1 = second palette)
\item bit 0 = Enable ULANext mode if 1. (0 after a reset)
\end{itemize}

The tilemap has its own pair of palettes for looking up 9-bit colours.  Each 
tile definition pixel is 4-bits which is combined with the 4-bit palette offset
from the tilemap entry in the most significant 8-bits.  This 8-bit value is 
passed through the tilemap palette to generate the final 9-bit colour for the 
pixel.

(R/W) \$68 (104) $\Rightarrow$ ULA Control
\begin{itemize}
\item bit 7 = 1 to disable ULA output
\item bit 6 = 0 to select the ULA colour for blending in SLU modes 6 \& 7
  
  = 1 to select the ULA/tilemap mix for blending in SLU modes 6 \& 7
\item bits 5-1 = Reserved must be 0
\item bit 0 = 1 to enable stencil mode when both the ULA and tilemap
  are enabled (if either are transparent the result is transparent
  otherwise the result is a logical AND of both colours)
\end{itemize}

Bit 0 can be set to choose stencil mode for the combined output of the
ULA and tilemap.  Bit 6 determines what colour is used in SLU modes 6
\& 7 where the ULA is combined with Layer 2 to generate highlighting
effects.

\subsection{Layer2}
Layer 2 is a linearly mapped, row-wise, upper left to lower right,
$256\times192\times256$ bit-map graphics area.  Both the main version
(8k pages 16-21/16k banks 8-10) and a shadow version (8k pages
22-27/16k banks 11-13) are six contiguous 8k pages (three contiguous
16k blocks) in the extended RAM space indicated by the contents of
Next registers \$12 (18 Layer 2 RAM page, default 8) and \$13 (19,
Layer 2 shadow page, default 11).  The layer 2 pages can be accessed
either by mapping the pages into normal RAM, or write only access
using port \$123B to fix them in the same space as the ROMs at
\$0000-\$3FFF.  The colours come from the indices in the layer 2
palette.  Layer 2 is drawn according to the values in registers \$16
(22, Layer 2 Offset X, default 0) and \$17 (23, Layer 2 Offset Y,
default 0).

\subsection{Sprites}
\begin{itemize}
\item Total 128 sprites with 64 pattern slots;
\item Size 16 by 16 pixels;
\item 256 or 16 colours per pixel;
\item 256-colour palette where each sprite can define an offset from
  the palette;
\item Collision detection of sprites;
\item Rotate and X/Y mirror flag;
\item Sprite scaling 1x, 2x, 4x, 8x:
\item Possibility to display sprites over the ZX Spectrum border.
\end{itemize}
To turn on the sprites, bit 0 of register \$15 (21) must be set. If
bit 1 is on, the sprites may be displayed over the default ZX Spectrum
border.

Port \$303B, if read, returns some information:
\begin{itemize}
\item bits 7-2: Reserved, always 0.
\item bit 1: max sprites per line flag.
\item bit 0: Collision flag.
\end{itemize}

The current sprite number to be modified can be set using either port
\$303b or Next register \$34. If bit 4 of Next register \$09 is set
then the sprite number controlled by I/O port commands is synchronized
with those controlled by Next register commands otherwise they are
independent. When written, this defines the sprite slot to be
configured, and also initializes the address of the palette. If bit 7
is set on write an offset of 128 is added to the pattern index.

Port \$57 is write-only and is used to send the attributes of the
selected sprite slot, being the address is auto-incremented each
writing and after sending the 4 or 5 bytes of attributes the address
points to the next sprite.

Next registers \$35-\$39 are non-incrementing controls of the sprite
attributes 0-4 while Next registers \$75-\$79 are auto incrementins
controls of the sprite attributes.

\begin{enumerate}
\item Attribute 0 - registers \$35 and \$75

  X position (bits 7-0).
\item Attribute 1 = registers \$36 and \$76

  Y position (bits 7-0).
\item Arrtibute 2 - registers \$37 and \$77
  \begin{itemize}
  \item bits 7-4 = palette offset
  \item bit 3 = X mirror
  \item bit 2 = Y mirror
  \item bit 1 = rotate flag
  \item bit 0 = X MSB.
  \end{itemize}
\item Attribute 3 - registers \$38 and \$78
  \begin{itemize}
  \item bit 7 = visible flag (1 = displayed)
  \item bit 6 = extended attribute (use attribute 4)
  \item bits 5-0 = Name (pattern index, 0-63).
  \end{itemize}
\item Attribute 4 - registers \$39 and \$79
  \begin{itemize}
  \item bits 7-6 = bit pattern depth and slot
    \begin{itemize}
    \item 00 = 8-bit patterns, anchor sprite (absolute position)
    \item 01 = position relative to anchor sprite (same bit depth as anchor)
    \item 10 = 4-bit pattern, lower half of pattern, anchor sprite
    \item 11 = 4-bit pattern, upper half of pattern, anchor sprite
    \end{itemize}
  \item bit 5 = special
  
    for 4-bit relative sprites
    \begin{itemize}
    \item 0 = use top 128 bytes
    \item 1 = use bottom 128 bytes
    \end{itemize}
    otherwise, reserved, must be 0	
  \item bits 4-3 = X direction sprite scaling
    \begin{itemize}
    \item 00 = 1x (16 pixels)
    \item 01 = 2x (32 pixels)
    \item 10 = 4x (64 pixels)
    \item 11 = 8x (128 pixels)
    \end{itemize}
  \item bits 2-1 = Y direction sprite scaling
    \begin{itemize}
    \item 00 = 1x (16 pixels)
    \item 01 = 2x (32 pixels)
    \item 10 = 4x (64 pixels)
    \item 11 = 8x (128 pixels)
    \end{itemize}
  \item bit 0 = Y MSB
  \end{itemize}
\end{enumerate}

Port \$5B is write-only and is used to send the pattern of the
selected sprite slot, with the address being auto-incremented with
each write and after sending the 256 bytes of the pattern the address
points to the next sprite slot. Each pattern byte represents the
offset of the palette and the order is left to right and top to
bottom.

Addresses from Ports \$57 and \$5B are independent and configured
together in the write port \$303B however the increment of the address
is done separately, can be sent the pattern of 10 sprites, for
example, and then sent the attributes of the 10 sprites without having
to rewrite in the port \$303B.

Because sprites can be displayed on top of the ZX Spectrum border, the
coordinates of each sprite can range from 0 to 319 for the X axis and
0 to 255 for the Y axis. For both axes, values from 0 to 31 are
reserved for the Left or top border, for the X axis the values 288 to
319 is reserved for the right border and for the Y axis values 224 to
255 for the lower border.

If the display of the sprites on the border is disabled, the
coordinates of the sprites range from (32,32) to (287,223).

The priority of the sprites can be set using bit 6 of Next register
\$15 to either run from (top to bottom) 0 to 127 (1) or 127 to 0 (0)

Each sprite has the Name attribute that defines which pattern it will
display, and more than one sprite can display the same pattern. Each
byte of the pattern defines the index of the palette with the 4 most
significant bits being summed with bits 7-4 of the third byte of the
attributes, allowing the programmer to display the same pattern with
different colours.

As an example of the palette offset, if the byte of the pixel pattern
(0,0) is \$14, and the palette offset is \$00, the colour will be of
the palette index \$14, but if the palette offset is set to \$20, The
palette index will be \$34.

In the palette each byte represents an index into the master sprite
palette format, and sprite transparency index (default \$e3)
determines the transparency color.

Collisions of 2 or more sprites only happen if at least 2
non-transparent pixels are drawn in the same position on the screen.

To set this pattern for some sprite, we must set the slot number of
the patterns through port \$303B and then send the 256 bytes in
sequence to port \$5B.
