\subsection{Utilities}

From the Digital Research: CP/M 3 Command Reference Manual 1984

This section documents all standard CP/M+ 3 commands plus those extras
included with the ZX Spectrum Next CP/M system.

\textbf{COLOURS} (ZX Spectrum Next)

\hangindent=0.7cm Syntax:\\
COLOURS paper ink \{[SET]\}\\
COLOURS RGB papervalue inkvalue \{[SET]\}

Terminal colours utility.

Sets the screen colours using standard ZX colours or octal 9-bit RGB
numbers.

\textbf{COPYSYS}

\hangindent=0.7cm Syntax:\\
COPYSYS

COPYSYS copies the CP/M Plus system from a CP/M Plus system diskette
to another diskette.  The new diskette must have the same format as
the original system diskette.

\textbf{DATE}

\hangindent=0.7cm Syntax:\\
DATE\\
DATE C\\
DATE CONTINUOUS\\
DATE time-specification\\
DATE SET\\

The DATE command lets you display and set the date and time of day.

\textbf{DEVICE}

\hangindent=0.7cm Syntax:\\
DEVICE\\
DEVICE NAMES\\
DEVICE VALUES\\
DEVICE logical-dev \{XON\(|\)NOXON\(|\)baud-rate\},\\
DEVICE physical-dev \{XON\(|\)NOXON\(|\)baud-rate\}\\
DEVICE logical-dev=physical-dev \{option\} \{,physical-dev \{option\},...\}\\
DEVICE logical-dev = NULL\\
DEVICE CONSOLE \{PAGE\}\\
DEVICE CONSOLE \{COLUMNS=n, LINES=n\}\\

DEVICE displays current logical device assignments and physical device
names.

\textbf{DIR} (built-in)

\hangindent=0.7cm Syntax:\\
DIR\\
DIR d:\\
DIR filespec\\
DIR d: options\\
DIR filespec,... filespec options

The DIR command displays the names of files catalogued in the
directory of an online disk that belong to current user number and
have the Directory (DIR) attribute. DIR accepts the * and ? wildcards
in the file specification.

The DIR command with options displays the names of files and the
characteristics associated with the files. DIR is a built-in
utility. DIR with options is a transient utility and must be loaded
into memory from the disk.

\textbf{DIRSYS/DIRS} (built-in)

\hangindent=0.7cm Syntax:\\
DIRSYS\\
DIRSYS d:\\
DIRSYS filespec

The DIRSYS command lists the names of files in the current directory
that have the system (SYS) attribute. DIRSYS accept the * and ?
wildcards in the file specification. DIRSYS is a built-in utility.

\textbf{DUMP}

\hangindent=0.7cm Syntax:\\
DUMP filespec

DUMP displays the contents of a file in and ASCII format.

\textbf{ECHO} (ZX Spectrum Next)

\hangindent=0.7cm Syntax:\\
ECHO string

Echo characters to the terminal

The following special character sequences may be used
\begin{itemize}
\item \textbackslash a alert (bell) (ASCII 7)
\item \textbackslash b backspace (ASCII 8)
\item \textbackslash e escape (ASCII 27)
\item \textbackslash n line feed (ASCII 10)
\item \textbackslash r carriage return (ASCII 13)
\item \textbackslash l interpret further characters as lower-case
\item \textbackslash u interpret further characters as upper-case
\item \textbackslash\textbackslash backslash ('\textbackslash')
\end{itemize}

Note that CP/M converts all your typed characters to upper-case before
providing them to ECHO.COM. Therefore you will need to use
\textbackslash l and \textbackslash u to specify the case of
characters if it is important (in ESCape sequences, for example).

\textbf{ED}

\hangindent=0.7cm Syntax:\\
ED\\
ED input-filespec\\
ED input-filespec \{d: \(|\) output-filespec\}

Character file editor. To redirect or rename the new version of the
file specify the destination drive or destination filespec.

\textbf{ERASE/ERA} (built-in)

\hangindent=0.7cm Syntax:\\
ERASE\\
ERASE filespec\\
ERASE filespec [CONFIRM]

The ERASE command removes one or more files from the directory of a
disk. Wildcard characters are accepted in the filespec. Directory and
data space are automatically reclaimed for later use by another
file. The ERASE command can be abbreviated to ERA.

[CONFIRM] option informs the system to prompt for verification before
erasing each file that matches the filespec. CONFIRM can be
abbreviated to C.

\textbf{EXIT} (ZX Spectrum Next)

\hangindent=0.7cm Syntax:\\
EXIT

The EXIT command leaves CP/M (rebooting the ZX Spectrum Next)

\textbf{EXPORT} (ZX Spectrum Next)

\hangindent=0.7cm Syntax:\\
EXPORT cpm-filespec nextzxos-filespec

NextZXOS file export utility

Export file to a NextZXOS drive.

\textbf{GENCOM}

\hangindent=0.7cm Syntax:\\
GENCOM COM-Eilespec RSX-filespec... RSX-Eilespec \{[LOADER \(|\)
  SCB=(Offset,value)]\}\\
GENCOM RSX-filespec ... RSX-filespec \{[NULL \(|\)
  SCB=(Offset,value)l\}\\
GENCOM filename\\
GENCOM filename [SCB=(offset,value)]

The GENCOM command attaches RSX files to a COM file, or creates a
dummy COM file containing only RSXS. It can also restore a previously
GENCOMed file to the original COM file without the header and RSXS,
add or replace RSXs in already GENCOMed files, and attach header
records to COM files without RSXS.

\textbf{GENCPM}

\hangindent=0.7cm Syntax:\\
GENCPM \{AUTO\(|\)AUTO DISPLAY\}

GENCPM creates a memory image CPM3.SYS file, containing the CP/M 3
BDOS and customized BIOS. The GENCPM utility performs late resolution
of intermodule references between system modules. GENCPM can accept
its command input interactively from the console or from a file
GENCPM.DAT.

In the nonbanked system, GENCPM creates a CPM3.SYS file from the
BDOS3.SPR and BIOS3.SPR files. In the banked system, GENCPM creates
the CPM3.SYS file from the RESBDOS3.SPR, the BNKBDOS3.SPR and the
BNKBIOS3.SPR files. Remember to back up your CPM3.SYS file before
executing GENCPM, because GENCPM deletes any existing CPM3.SYS file
before it generates a new system.

\textbf{GET}

\hangindent=0.7cm Syntax:\\
GET \{CONSOLE INPUT FROM\} FILE filespec options\\
GET \{CONSOLE INPUT FROM\} CONSOLE

GET directs the system to take console input from a file for the next
system comand or user program entered at the console.

Console input is taken from a file until the program terminates. If
the file is exhausted before program input is terminated, the program
looks for subsequent input from the console. If the program terminates
before exhausting all its input, the system reverts back to the
console for console input.

\textbf{HELP}

\hangindent=0.7cm Syntax:\\
HELP\\
HELP topic\\
HELP topic subtopic\\
HELP topic [NOPAGE]\\
HELP topic subtopic1...subtopic8\\
HELP\(>\)topic\\
HELP\(>\).subtopic

HELP displays a list of topics and provides summarized information for
CP/M Plus commands.

Typing HELP topic displays information about that topic. Typing HELP
topic subtopic displays information about that subtopics One or two
letters is enough to identify the topics. After HELP displays
information for your topic, it displays the special prompt HELP\(>\)
on your screen, followed by a list of subtopics.

\begin{itemize}
\item Enter ? to display list of main topics.
\item Enter a period and subtopic name to access subtopics.
\item Enter a period to redisplay what you just read.
\item Press RETURN to return to the CP/M Plus system prompt.
\item {[NOPAGE]} option disables the 24 lines per page console display.
\item Press any key to exit a display and return to the HELP\(>\) prompt.
\end{itemize}

\textbf{HEXCOM}

\hangindent=0.7cm Syntax:\\
HEXCOM filename

The HEXCOM Command generates a command file (filetype COM) from a HEX
input file. it names the output tile with the same filename as the
input file but with filetype COM. HEXCOM always looks for a file with
filetype HEX.

\textbf{IMPORT} (ZX Spectrum Next)

\hangindent=0.7cm Syntax:\\
IMPORT nextzxos-filespec\\
IMPORT nextzxos-filespec cpm-filespec

NextZXOS file import utility

List or import files from a NextZXOS drive.

\textbf{INITDIR} (Not included)

\hangindent=0.7cm Syntax:\\
INITDIR d:

The INITDIR command initializes a disk directory to allow date and
time stamping of files on that disk. INITDIR can also recover
time/date directory space.

\textbf{NEXTREG} (ZX Spectrum Next)

\hangindent=0.7cm Syntax:\\
NEXTREG register \{value\}

NextReg Utility

Show or change a NextReg register (use at your own risk!)

\textbf{LIB} (Not included)

\hangindent=0.7cm Syntax:\\
LIB filespec options\\
LIB filespec options=filespec \(<\)modifier\(>\) f,filespec\(<\)modifier\(>\)

A library is a file that contains a collection of object modules.

Use the LIB utility to create libraries, and to append, replace,
select, or delete modules from an existing library. Use LIB to obtain
information about the contents of library files.  LIB creates and
maintains library files that contain object modules in Microsoft REL
file format.  These modules are produced by the Digital Research
relocatable macro-assembler program, RMAC, or other language
translator that produces modules in Microsoft REL file format.

You can use LINK-80 to link the object modules contained in a library
to other object files.  LINK-80 automatically selects from the library
only those modules needed by the program being linked, and then forms
an executable file with a filetype of Com.

\textbf{LINK} (Not included)

\hangindent=0.7cm Syntax:\\
LINK filespec [options]\\
LINK filespec [options],...filespec [options]\\
LINK filespec [options]=filespec [options],...\\

LINK combines relocatable object modules such as those produced by
RMAC and PL/I- 80 into a COM file ready for execution. Relocatable
files can contain external references and publics.  Relocatable files
can reference modules in library files. LINK searches the library
files and includes the referenced modules in the output file. See the
Programmer's Utilities Guide for the CP/M Family of Operating Systems
for a complete description of LINK-80.

Use LINK option switches to control execution parameters. Link options
follow the file specifications and are enclosed within square
brackets. Multiple switches are separated by commas.

\textbf{MAC} (Not included)

\hangindent=0.7cm Syntax:\\
MAC filename [\$options]

MAC, the CP/M Plus macro assembler, reads assembly language statements
from a file of type ASM, assembles the statements, and produces three
output files with the input filename and filetypes of HEX, PRN, and
SYM. Filename.HEX contains Intel hexadecimal format object
code. Filename.PRN contains an annotated source listing that you can
print or examine at the console. Filename.SYM contains a sorted list
of symbols defined in the program.

Use options to direct the input and output of MAC. Use a letter with
the option to indicate the source and destination drives, and console,
printer, or zero output. Valid drive names are A through 0. X, P, and
Z specify console, printer, and zero output, respectively.

\textbf{PATCH}

\hangindent=0.7cm Syntax:\\
PATCH filename.typ n

The PATCH command displays or installs patch number n to the CP/M Plus
system or command files. The patch number n must be between 1 and 32
inclusive.

\textbf{PIP}

\hangindent=0.7cm Syntax:\\
PIP Destination = Source\\
PIP d:[Gn]=filespec [options]\\
PIP filespec[Gn]=filespec [options]\\
PIP filespec[Gn]device=filespec [options] device

The file copy program PIP copies files, combines files, and transfers
files between disks, printers, consoles, or other devices attached to
your computer. The first filespec is the destination. The second
filespec is the source. Use two or more source filespecs separated by
commas to combine two or more files into one file. [options] is any
combination of the available options. The [Gn] option in the
destination filespec tells PIP to copy your file to that user
number. PIP with no command tail displays an * prompt and awaits your
series of commands, entered and processed one line at a time. The
source or destination can optionally be any CP/M Plus logical device.

\textbf{PUT}

\hangindent=0.7cm Syntax:\\
PUT CONSOLE \{OUTPUT TO\} FILE filespec \{option\}\\
PUT PRINTER \{OUTPUT TO\} FILE filespec \{option\}\\
PUT CONSOLE \{OUTPUT TO\} CONSOLE\\
PUT PRINTER \{OUTPUT TO\} PRINTER

PUT puts console or printer output to a file for the next command
entered at the console, until the program terminates. Then console
output reverts to the console. Printer output is directed to a file
until the program terminates. Then printer output is put back to the
printer.

PUT with the SYSTEM option directs all subsequent console/printer
output to the specified file.  This option terminates when you enter
the PUT CONSOLE or PUT PRINTER command.

\textbf{RENAME/REN} (built-in)

\hangindent=0.7cm Syntax:\\
RENAME\\
RENAME new-filespec=old-filespec

RENAME lets you change the name of a file in the directory of a
disk. To change several filenames in one command use the * or ?
wildcards in the file specifications. You can abbreviate the RENAME
command to REN. REN prompts you for input.

\textbf{RMAC} (Not included)

\hangindent=0.7cm Syntax:\\
RMAC filespec {options}

RMAC, a relocatable macro assembler, assembles ASM files into REL
files that you can link to create COM files.

RMAC options specify the destination of the output files. Replace d
with the destination drive letter for the output files.

\textbf{SAVE}

\hangindent=0.7cm Syntax:\\
SAVE

SAVE copies the contents of memory to a file. To use SAVE, first issue
the SAVE command, then run your program which reads a file into
memory. Your program exits to the SAVE utility which prompts you for a
filespec to which it copies the contents of memory, and the beginning
and ending address of the memory to be SAVED.

\textbf{SET}

\hangindent=0.7cm Syntax:\\
SET [options]\\
SET d: [options]\\
SET filespec [options]\\
SET [option = modifier]\\
SET filespec [option = modifier]

SET initiates password protection and time stamping of files. It also
sets the file and drive attributes Read/Write, Read/Only, DIR and
SYS. It lets you label a disk and passord protect the label. To enable
time stamping of files, you must first run INITDIR to format the disk
directory.

\textbf{SET Default password operation:}

\hangindent=0.7cm Syntax:\\
SET [DEFAULT=password]

Instructs the system to use a default password if you do not enter a
password for a password-protected file.

\textbf{SET Time-stamp operations:}

\hangindent=0.7cm Syntax:\\
SET {d:} [CREATE=ON\(|\)OFF]\\
SET {d:} [ACCESS=ON\(|\)OFF]\\
SET {d:} [UPDATE=ON\(|\)OFF]

The above set commands allw YOU to keep a record of the time and date
of file creation and update or of the last access update of your
files.

\textbf{SET Drive operations:}

\hangindent=0.7cm Syntax:\\
SET {d:} [RO]\\
SET {d:} [RW]

Adds or removes write protection from a drive.

\textbf{SETDEF}

\hangindent=0.7cm Syntax:\\
SETDEF\\
SETDEF [TEMPORARY=d:]\\
SETDEF d:i,d:i,d:i,d:i\\
SETDEF [ORDER= (typ1, typn)]\\
SETDEF [DISPLAY \(|\) NO DISPLAY)\\
SETDEF [PAGE \(|\) NOPAGE]

SETDEF allows the user to display or define up to four drives for the
program search order, the drive for temporary files, and the filetype
search order. The SETDEF definitions affect only the loading of
programs and/or execution of SUBMIT (SUB) files. SETDEF turns on/off
the system Display and Console Page modes. When on, the system
displays the location and name of programs loaded or SUBmit files
executed, and stops after displaying one full console screen of
information.

\textbf{SHOW}

\hangindent=0.7cm Syntax:\\
SHOW\\
SHOW d:\\
SHOW d: [SPACE]\\
SHOW d: [LABEL]\\
SHOW d: [USERS]\\
SHOW d: [DIR]\\
SHOW d: [DRIVE]

The SHOW command displays the following disk drive information:
\begin{itemize}
\item access mode and the amount of free disk space
\item disk label
\item current user number
\item number of files for each user number on the disk
\item number of free directory entries for the disk
\item drive characteristics
\end{itemize}

\textbf{SID}

\hangindent=0.7cm Syntax:\\
SID [pgm-filespec],\{sym-filespec\}

The SID symbolic debugger allows you to monitor and test programs
developed for the 8080 microprocessor. SID supports real-time
breakpoints, fully monitored execution, symbolic disassembly,
assembly, and memory display and fill functions. SID can dynamically
load SID utility programs to provide traceback and histogram
facilities.

\textbf{SUBMIT}

\hangindent=0.7cm Syntax:\\
SUBMIT\\
SUBMIT filespec\\
SUBMIT filespec argument ... argument

The SUBMIT command lets you execute a group (batch) of commands from a
SUBmit file (a file with filetype of SUB).

SUB files:

The SUB file can contain the following types of lines:
\begin{itemize}
\item any valid CP/M Plus command
\item any valid CP/M Plus command with SUBMIT parameters (\$0-\$9)
\item any data input line
\item any program input line with parameters (\$0 to \$9)
\end{itemize}

The command line cannot exceed 135 characters.

\textbf{TERMINFO} (ZX Spectrum Next)

\hangindent=0.7cm Syntax:\\
TERMINFO

This program provides information on the terminal facilities provided
by the BIOS on the ZX Spectrum Next.

\textbf{TERMSIZE} (ZX Spectrum Next)

\hangindent=0.7cm Syntax:\\
TERMSIZE top left height width

Terminal resize utility

Size can be up to 32x80 (defaults to 24x80, suitable for many
programs). If setting a reduced size, the top and left parameters can
be used to make the image more centered on your screen.

\textbf{TYPE/TYP} (built-in)

\hangindent=0.7cm Syntax:\\
TYPE\\
TYPE filespec\\
TYPE filespec [PAGE]\\
TYPE filespec [NOPAGE]

The TYPE command displays the contents of an ASCII character file on
your screen.

\textbf{UPGRADE} (ZX Spectrum Next)

\hangindent=0.7cm Syntax:\\
UPGRADE

UPGRADE CP/M from C:/NEXTZXOS/CPMBASE.P3D

\textbf{USER/USE} (built-in)

\hangindent=0.7cm Syntax:\\
USER\\
USER n

The USER command sets the current user number. The disk directory can
be divided into distinct groups according to a User Number. User
numbers range from 0 through 15.

\textbf{XREF} (Not included)

\hangindent=0.7cm Syntax:\\
XREF \{d:\} filename \{\$P\}

XREF provides a cross-reference summary of variable usage in a
program. XREF requires the PRN and SYM files produced by MAC or RMAC
for input to the program. The SYM and PRN files must have the same
filename as the filename in the XREF command tail. XREF outputs a file
of type XRF.
