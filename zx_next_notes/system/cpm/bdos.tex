\subsection{BDOS}
From the CP/M 3 Programmers' Guide 1984

This section documents all BDOS system calls to include the parameters
that must be passes to them and the values that are returned to the
calling program.

\index{system reset}
BDOS function 0: \textbf{SYSTEM RESET}

Entry Parameters:
\begin{itemize}
\item[] C: \$00
\end{itemize}

The System Reset function terminates the calling program and returns
control to the CCP via a warm start sequence. Calling this function
has the same effect as a jump to location \$0000 of Page Zero.

Note that the disk subsystem is not reset by System Reset under CP/M
3. The calling program can pass a return code to the CCP by calling
Function 108, Get/Set Program Return Code, prior to making a System
Reset call or jumping to location \$0000.

\index{console input}
BDOS function 1: \textbf{CONSOLE INPUT}
 
Entry Parameters:
\begin{itemize}
\item[] C: \$01
\end{itemize}
 
Returned Value:
\begin{itemize}
\item[] A: ASCII Character
\end{itemize}

The Console Input function reads the next character from the logical
console, CONIN:, to register A. Graphic characters, along with
carriage return, line-feed, and backspace, CTRL-H, are echoed to the
console. Tab characters, CTR-L-1, are expanded in columns of 8
characters. CTRL-S, CTRL-Q, and CTRL-P are normally intercepted as
described below. All other non-graphic characters are returned in
register A but are not echoed to the console.

When the Console Mode is in the default state Function 1 intercepts
the stop scroll, CTRL-S, start scroll, CTRL-Q, and start/stop printer
echo, CTRL-P, characters. Any characters that are typed following a
CTRL-S and preceding a CTRL-Q are also intercepted. However, if
start/stop scroll has been disabled by the Console Mode, the CTRL-S,
CTRL-Q, and CTRL-P characters are not intercepted. Instead, they are
returned in register A, but are not echoed to the console.

If printer echo has been invoked, all characters that are echoed to
the console are also sent to the list device, LST:.  Function 1 does
not return control to the calling program until a non-intercepted
character is typed, thus suspending execution if a character is not
ready.

\index{console output}
BDOS function 2: \textbf{CONSOLE OUTPUT}

Entry Parameters:
\begin{itemize}
\item[] C: \$02
\item[] E: ASCII Character
\end{itemize}

The Console Output function sends the ASCII character from register E
to the logical console device, CONOUT:. When the Console Mode is in
the default state (see Section 2.2.1), Function 2 expands tab
characters, CTRL-1, in columns of 8 characters, checks for stop
scroll, CTRL-S, start scroll, CTRL-Q, and echoes characters to the
logical list device, LST:, if printer echo, CTRL-P, has been invoked.

\index{auxiliary input}
BDOS function 3: \textbf{AUXILIARY INPUT}

Entry Parameters:
\begin{itemize}
\item[] C: \$03
\end{itemize}

Returned Value:
\begin{itemize}
\item[] A: ASCII Character
\end{itemize}

The Auxiliary Input function reads the next character from the logical
auxiliary input device, AUXIN:, into register A. Control does not
return to the calling program 'I the character is read.  unti

\index{auxiliary output}
BDOS function 4: \textbf{AUXILIARY OUTPUT}

Entry Parameters:
\begin{itemize}
\item[] C: \$04
\item[] E: ASCII Character
\end{itemize}

The Auxiliary Output function sends the ASCII character from register
E to the logical auxiliary output device, AUXOUT:.

\index{list output}
BDOS function 5: \textbf{LIST OUTPUT}

Entry Parameters:
\begin{itemize}
\item[] C: \$05
\item[] E: ASCII Character
\end{itemize}

The List Output function sends the ASCII character in register E to
the logical list device, LST:.

\index{direct console I/O}
BDOS function 6: \textbf{DIRECT CONSOLE I/O}

Entry Parameters:
\begin{itemize}
\item[] C: \$06
\item[] E: function/data (see description)
\end{itemize}

Returned Value:
\begin{itemize}
\item[] A: char/status/no value (see description)
\end{itemize}

CP/M 3 supports direct I/O to the logical console, CONIN:, for those
specialized applications where unadorned console input and output is
required. Use Direct Console I/O carefully because it bypasses all the
normal control character functions.  Programs that perform direct I/O
through the BIOS under previous releases of CP/M should be changed to
use direct I/O so that they can be fully supported under future
releases of MP/M and CP/M.

A program calls Function 6 by passing one of four different values in
register E.

\begin{itemize}
\item \$FF Console input/status command returns an input character; if
  no character is ready, a value of zero is returned.
\item \$FE Console status command (On return, register A contains 00
  if no character is ready; otherwise it contains \$FF.)
\item \$FD Console input command, returns an input character; this
  function will suspend the calling process until a character is
  ready.
\item ASCII Function 6 assumes that register E contains a valid ASCII
  chararacter and sends it to the console.
\end{itemize}

\index{auxiliary input status}
BDOS function 7: \textbf{AUXILIARY INPUT STATUS}

Entry Parameters:
\begin{itemize}
\item[] C: \$07
\end{itemize}

Returned Value:
\begin{itemize}
\item[] A: Auxiliary Input Status
\end{itemize}

The Auxiliary Input Status function returns the value \$FF in register
A if a character is ready for input from the logical auxiliary input
device, AUXIN:. If no character is ready for input, the value \$00 is
returned.

\index{auxiliary output status}
BDOS function 8: \textbf{AUXILIARY OUTPUT STATUS}

Entry Parameters:
\begin{itemize}
\item[] C: \$08
\end{itemize}

Returned Value:
\begin{itemize}
\item[] A: Auxiliary Output Status
\end{itemize}

The Auxiliary Output Status function returns the value \$FF in
register A if the logical auxiliary output device, AUXOUT:, is ready
to accept a character for output.  If the device is not ready for
output, the value \$00 is returned.

\index{print string}
BDOS function 9: \textbf{PRINT STRING}

Entry Parameters:
\begin{itemize}
\item[] C: \$09
\item[] DE: String Address
\end{itemize}

The Print String function sends the character string addressed by
register pair DE to the logical console, CONOUT:, until it encounters
a delimiter in the string. Usually the delimiter is a dollar sign, \$,
but it can be changed to any other value by Function 110, Get/Set
Output Delimiter. If the Console Mode is in the default state,
Function 9 expands tab characters, CTRL-I, in columns of 8
characters. It also checks for stop scroll, CTRL-S, start scroll,
CTRL-Q, and echoes to the logical list device, LST:, if printer echo,
CTRL-P, has been invoked.

\index{read console buffer}
BDOS function 10: \textbf{READ CONSOLE BUFFER}

Entry Parameters:
\begin{itemize}
\item[] C: \$0A
\item[] DE: Buffer Address
\end{itemize}

Returned Value:
\begin{itemize}
\item[] Console Characters in Buffer
\end{itemize}

The Read Console Buffer function reads a line of edited console input
from the logical console, CONIN:, to a buffer that register pair DE
addresses. It terminates input and returns to the calling program when
it encounters a return, CTRL-M, or a line feed, CTRL-J,
character. Function 10 also discards all input characters after the
input buffer is filled. In addition, it outputs a bell character,
CTRL-G, to the console when it discards a character to signal the user
that the buffer is full. The input buffer addressed by DE has the
following format:

%### Scan and insert illustration here.

where mx is the maximum number of characters which the buffer holds,
and nc is the number of characters placed in the buffer. The
characters entered by the operator follow the nc value. The value mx
must be set prior to making a Function 10 call and may range in value
from 1 to 255. Setting mx to zero is equivalent to setting mx to
one. The value nc is returned to the calling program and may range
from zero to mx. If nc < mx, then uninitialized positions follow the
last character, denoted by ??  in the figure. Note that a terminating
return or line feed character is not placed in the buffer and not
included in the count nc.

If register pair DE is set to zero, Function 10 assumes that an
initialized input buffer is located at the current DMA address (see
Function 26, Set DMA Address).  This allows a program to put a string
on the screen for the user to edit. To initialize the input buffer,
set characters c1 through cn to the initial value followed by a binary
zero terminator.

When a program calls Function 10 with an initialized buffer, Function
10 operates as if the user had typed in the string. When Function 10
encounters the binary zero terminator, it accepts input from the
console. At this point, the user can edit the initialized string or
accept it as it is by pressing the RETURN key. However, if the
initialized string contains a return, CTRL-M, or a linefeed, CTRL-J,
character, Function 10 returns to the calling program without giving
the user the opportunity to edit the string.

The level of console editing supported by Function 10 differs for the
banked and nonbanked versions of CP/M 3. Refer to the CPIM Plus (CPIM
Version 3) Operating System User's Guide for a detailed description of
console editing. In the nonbanked version, Function 10 recognizes the
following edit control characters.

Nonbanked CP/M 3
\begin{itemize}
\item rub/del Removes and echoes the last character; GENCPM can change
  this function to CTRL-H
\item CTRL-C Reboots when at the beginning of line; the Console Mode
  can disable this function
\item CTRL-E Causes physical end of line
\item CTRL-H Backspaces one character position; GENCPM can change this
  function to rub/del
\item CTRL-J (Line-feed) terminates input line
\item CTRL-M (Return) terminates input line
\item CTRL-P Echoes console output to the list device
\item CTRL-R Retypes the current line after new line
\item CTRL-U Removes current line after new line
\item CTRL-X Backspaces to beginning of current line
\end{itemize}

The banked version of CP/M 3 expands upon the editing provided in the
non-banked version. The functionality of the two versions is similar
when the cursor is positioned at the end of the line. However, in the
banked version, the user can move the cursor anywhere in the current
line, insert characters, delete characters, and perform other editing
functions. In addition, the banked version saves the previous command
line; it can be recalled when the current line is empty. In the banked
version, Function 10 recognizes the following edit control characters.

Banked CP/M 3
\begin{itemize}
\item rub/del Removes and echoes the last character if at the end of
  the line; otherwise deletes the character to the left of the current
  cursor position; GENCPM can change this function to CTRL-H.
\item CTRL-A Moves cursor one character to the left.
\item CTRL-B Moves cursor to the beginning of the line when not at the
  beginning; otherwise moves cursor to the end of the line.
\item CTRL-C Reboots when at the beginning of line; the Console Mode
  can disable this function.
\item CTRL-E Causes physical end-of-line; if the cursor is positioned
  in the middle of a line, the characters at and to the right of the
  cursor are displayed on the next line.
\item CTRL-F Moves cursor one character to the right.
\item CTRL-G Deletes the character at the current cursor position when
  in the middle of the line; has no effect when the cursor is at the
  end of the line.
\item CTRL-H Backspaces one character position when positioned at the
  end of the line,; otherwise deletes the character to the left of the
  cursor; GENCPM can change this function to rub/del.
\item CTRL-J (Line-feed) terminates input; the cursor can be
  positioned anywhere in the line; the entire input line is
  accepted; sets the previous line buffer to the input line.
\item CTRL-K Deletes all characters to the right of the cursor along
  with the character at the cursor.
\item CTRL-M (Return) terminates input; the cursor can be positioned
  anywhere in the line; the entire input line is accepted; sets the
  previous line buffer to the input line.
\item CTRL-P Echoes console output to the list device.
\item CTRL-R Retypes the characters to the left of the cursor on the
  new line.
\item CTRL-U Updates the previous line buffer to contain the
  characters to the left of the cursor; deletes current line, and
  advances to new line.
\item CTRL-W Recalls previous line if current line is empty; otherwise
  moves cursor to end-of-line.
\item CTRL-X Deletes all characters to the left of the cursor.
\end{itemize}

For banked systems, Function 10 uses the console width field defined
in the System Control Block. If the console width is exceeded when the
cursor is positioned at the end of the line, Function 10 automatically
advances to the next line. The beginning of the line can be edited by
entering a CTRL-R.

When a character is typed while the cursor is positioned in the middle
of the line, the typed character is inserted into the line. Characters
at and to the right of the cursor are shifted to the right. If the
console width is exceeded, the characters disappear off the right of
the screen. However, these characters are not lost. They reappear if
characters are deleted out of the line, or if a CTRL-E is typed.

\index{get console status}
BDOS function 11: \textbf{GET CONSOLE STATUS}

Entry Parameters:
\begin{itemize}
\item[] C: \$0B
\end{itemize}

Returned Value:
\begin{itemize}
\item[] A: Console Status
\end{itemize}

The Get Console Status function checks to see if a character has been
typed at the logical console, CONIN:. If the Console Mode is in the
default state, Function 11 returns the value \$01 in register A when a
character is ready. If a character is not ready, it returns a value of
\$00.

If the Console Mode is in CTRL-C Only Status mode, Function 11 returns
the value \$01 in register A only if a CTRL-C has been typed at the
console.

\index{return version number}
BDOS function 12: \textbf{RETURN VERSION NUMBER}

Entry Parameters:
\begin{itemize}
\item[] C: \$0C
\end{itemize}

Returned Value:
\begin{itemize}
\item[] HL: Version Number
\end{itemize}

The Return Version Number function provides information that allows
version independent programming. It returns a two-byte value in
register pair HL: H contains \$00 for CP/M and L contains \$31, the
BDOS file system version number.  Function 12 is useful for writing
applications programs that must run on multiple versions of CP/M and
MP/M.

\index{reset disk system}
BDOS function 13: \textbf{RESET DISK SYSTEM}

Entry Parameters:
\begin{itemize}
\item[] C: \$0D
\end{itemize}

The Reset Disk System function restores the file system to a reset
state where all the disk drives are set to read-write (see Functions
28 and 29), the default disk is set to drive A, and the default DMA
address is reset to \$0080. This function can be used, for example, by
an application program that requires disk changes during
operation. Function 37, Reset Drive, can also be used for this
purpose.

\index{select disk}
BDOS function 14: \textbf{SELECT DISK}

Entry Parameters:
\begin{itemize}
\item[] C: \$0E
\item[] E: Selected Disk
\end{itemize}

Returned Value:
\begin{itemize}
\item[] A: Error Flag
\item[] H: Physical Error
\end{itemize}

The Select Disk function designates the disk drive named in register E
as the default disk for subsequent BDOS file operations. Register E is
set to 0 for drive A, 1 for drive B, and so on through 15 for drive P
in a full 16-drive system. In addition, Function 14 logs in the
designated drive if it is currently in the reset state. Logging-in a
drive activates the drive's directory until the next disk system reset
or drive reset operation.

FCBs that specify drive code zero (dr = \$00) automatically reference
the currently selected default drive. FCBs with drive code values
between 1 and 16, however, gnore the selected default drive and
directly reference drives A through P.

Upon return, register A contains a zero if the select operation was
successful. If a physical error was encountered, the select function
performs different actions depending on the BDOS error mode (see
Function 45). If the BDOS error mode is in the default mode, a message
identifying the error is displayed at the console, and the calling
program is terminated. Otherwise, the select function returns to the
calling program with register A set to \$FF and register H set to one
of the following

physical error codes:
\begin{itemize}
\item 01 Disk I/O Error
\item 04 Invalid drive
\end{itemize}

\index{open file}
BDOS function 15: \textbf{OPEN FILE}

Entry Parameters:
\begin{itemize}
\item[] C: \$0F
\item[] DE: FCB Address
\end{itemize}

Returned Value:
\begin{itemize}
\item[] A: Directory Code
\item[] H: Physical or Extended Error
\end{itemize}

The Open File function activates the FCB for a file that exists in the
disk directory under the currently active user number or user
zero. The calling program passes the address of the FCB in register
pair DE, with byte 0 of the FCB specifying the drive, bytes 1 through
11 specifying the filename and filetype, and byte 12 specifying the
extent. Usually, byte 12 of the FCB is initialized to zero.

If the file is password protected in Read mode, the correct password
must be placed in the first eight bytes of the current DMA, or have
been previously established as the default password (see Function
106). If the current record field of the FCB, cr, is set to \$FF,
Function 15 returns the byte count of the last record of the file in
the cr field. You can set the last record byte count for a file with
Function 30, Set File Attributes. Note that the current record field
of the FCB, cr, must be zeroed by the calling program before beginning
read or write operations if the file is to be accessed sequentially
from the first record.

If the current user is non-zero, and the file to be opened does not
exist under the current user number, the open function searches user
zero for the file. If the file exists under user zero, and has the
system attribute, t2', set, the file is opened under user zero. Write
operations are not supported for a file that is opened under user zero
in this manner.

If the open operation is successful, the user's FCB is activated for
read and write operations. The relevant directory information is
copied from the matching directory FCB into bytes d0 through dn of the
FCB. If the file is opened under user zero when the current user
number is not zero, interface attribute f8' is set to one in the user's
FCB. In addition, if the referenced file is password protected in
Write mode, and the correct password was not passed in the DMA, or did
not match the default password, interface attribute f7' is set to
one. Write operations are not supported for an activated FCB if
interface attribute f7' or f8' is true.

When the open operation is successful, the open function also makes an
Access date and time stamp for the opened file when the following
conditions are satisfied: the referenced drive has a directory label
that requests Access date and time stamping, and the FCB extent number
field is zero.

Upon return, the Open File function returns a directory code in
register A with the value \$00 if the open was successful, or \$FF,
255 decimal, if the file was not found.  Register H is set to zero in
both of these cases. If a physical or extended error was encountered,
the Open File function performs different actions depending on the
BDOS error mode (see Function 45). If the BDOS error mode is in the
default mode, a message identifying the error is displayed at the
console and the program is terminated. Otherwise, the Open File
function returns to the calling program with register A set to \$FF,
and register H set to one of the following physical or extended error
codes:

\begin{itemize}
\item 01 : Disk I/O Error
\item 04 : Invalid drive error
\item 07 : File password error
\item 09 : ? in the FCB filename or filetype field
\end{itemize}

\index{close file}
BDOS function 16: \textbf{CLOSE FILE}

Entry Parameters:
\begin{itemize}
\item[] C: \$10
\item[] DE: FCB Address
\end{itemize}

Returned Value:
\begin{itemize}
\item[] A: Directory Code
\item[] H: Physical or Extended Error
\end{itemize}

The Close File function performs the inverse of the Open File
function. The calling program passes the address of an FCB in register
pair DE. The referenced FCB must have been previously activated by a
successful Open or Make function call (see Functions 15 and
22). Interface attribute f5' specifies how the file is to be closed as
shown below:

\begin{itemize}
\item f5' = 0 - Permanent close (default mode)
\item f5' = 1 - Partial close
\end{itemize}

A permanent close operation indicates that the program has completed
file operations on the file. A partial close operation updates the
directory, but indicates that the file is to be maintained in the open
state.

If the referenced FCB contains new information because of write
operations to the FCB, the close function permanently records the new
information in the referenced disk directory. Note that the FCB does
not contain new information, and the directory update step is
bypassed if only read or update operations have been made to the
referenced FCB.

Upon return, the close function returns a directory code in register A
with the value \$00 if the close was successful,, or \$FF, 255
Decimal, if the file was not found.  Register H is set to zero in both
of these cases. If a physical or extended error is encountered, the
close function performs different actions depending on the BDOS error
mode (see Function 45). If the BDOS error mode is in the default mode,
a message identifying the error is displayed at the console, and the
calling program is terminated. Otherwise, the close function returns
to the calling program with register A set to \$FF and register H set
to one of the following physical error codes:

\begin{itemize}
\item 01 Disk I/O error
\item 02 Read/only disk
\item 04 Invalid drive error
\end{itemize}

\index{search for first}
BDOS function 17: \textbf{SEARCH FOR FIRST}

Entry Parameters:
\begin{itemize}
\item[] C: \$11
\item[] DE: FCB Address
\end{itemize}

Returned Value:
\begin{itemize}
\item[] \item[] A: Directory Code
\item[] H: Physical Error
\end{itemize}

The Search For First function scans the directory for a match with the
FCB addressed by register pair DE. Two types of searches can be
performed. For standard searches, the calling program initializes
bytes 0 through 12 of the referenced FCB, with byte 0 specifying the
drive directory to be searched, bytes 1 through 11 specifying the file
or files to be searched for,, and byte 12 specifying the
extent. Usually byte 12 is set to zero. An ASCII question mark, 63
decimal, 3F hex, in any of the bytes 1 through 12 matches all entries
on the directory in the corresponding position. This facility, called
ambiguous reference, can be used to search for multiple files on the
directory. When called in the standard mode, the Search function scans
for the first file entry in the specified directory that matches the
FCB, and belongs to the current user number.

The Search For First function also initializes the Search For Next
function. After the Search function has located the first directory
entry matching the referenced FCB, the Search For Next function can be
called repeatedly to locate all remaining matching entries. In terms
of execution sequence, however, the Search For Next call must either
follow a Search For First or Search For Next call with no other
intervening BDOS disk-related function calls.

If byte 0 of the referenced FCB is set to a question mark, the Search
function ignores the remainder of the referenced FCB, and locates the
first directory entry residing on the current default drive. All
remaining directory entries can be located by making multiple Search
For Next calls. This type of search operation is not usually made by
application programs, but it does provide complete flexibility to scan
all current directory values. Note that this type of search operation
must be performed to access a drive's directory label.

Upon return, the Search function returns a Directory Code in register
A with the value 0 to 3 if the search is successful, or \$FF, 255
Decimal, if a matching directory entry is not found. Register H is set
to zero in both of these cases. For successful searches, the current
DMA is also filled with the directory record containing the matching
entry, and the relative starting position is A * 32 (that is, rotate
the A register left 5 bits, or ADD A five times). Although it is not
usually required for application programs, the directory information
can be extracted from the buffer at this position.

If the directory has been initialized for date and time stamping by
INITDIR, then an SFCB resides in every fourth directory entry, and
successful Directory Codes are restricted to the values 0 to 2. For
successful searches, if the matching directory record is an extent
zero entry, and if an SFCB resides at offset 96 within the current
DMA, contents of (DMA Address + 96) = \$21, the SFCB contains the date
and time stamp information, and password mode for the file. This
information is located at the relative starting position of 97 + (A *
10) within the current DMA in the following format:

\begin{itemize}
\item 0 - 3 Create or Access Date and Time Stamp Field
\item 4 - 7 Update Date and Time Stamp Field
\item 8 : Password Mode Field
\end{itemize}

If a physical error is encountered, the Search function performs
different actions depending on the BDOS error mode (see Function
45). If the BDOS error mode is in the default mode, a message
identifying the error is displayed at the console, and the calling
program is terminated. Otherwise, the Search function returns to the
calling program with register A set to \$FF, and register H set to
one of the following physical error codes:

\begin{itemize}
\item 01 Disk I/O error
\item 04 Invalid drive error
\end{itemize}

\index{search for next}
BDOS function 18: \textbf{SEARCH FOR NEXT}

Entry Parameters:
\begin{itemize}
\item[] C: \$12
\end{itemize}

Returned Value:
\begin{itemize}
\item[] A: Directory Code
\item[] H: Physical Error
\end{itemize}

The Search For Next function is identical to the Search For First
function, except that the directory scan continues from the last entry
that was matched. Function 18 returns a Directory code in register A,
analogous to Function 17.

Note: in execution sequence, a Function 18 call must follow either a
Function 17 or another Function 18 call with no other intervening BDOS
disk-related function calls.

\index{delete file}
BDOS function 19: \textbf{DELETE FILE}

Entry Parameters:
\begin{itemize}
\item[] C: \$13
\item[] DE: FCB Address
\end{itemize}

Returned Value:
\begin{itemize}
\item[] A: Directory Code
\item[] H: Extended or Physical Error
\end{itemize}

The Delete File function removes files or XFCBs that match the FCB
addressed in register pair DE. The filename and filetype can contain
ambiguous references, that is, question marks in bytes f1' through t3',
but the dr byte cannot be ambiguous, as it can in the Search and
Search Next functions. Interface attribute f5' specifies the type of
delete operation that is performed.

\begin{itemize}
\item f5' = 0 - Standard Delete (default mode)
\item f5' = 1 - Delete only XFCBs
\end{itemize}

If any of the files that the referenced FCB specify are password
protected, the correct password must be placed in the first eight
bytes of the current DMA buffer, or have been previously established
as the default password (see Function 106).

For standard delete operations, the Delete function removes all
directory entries belonging to files that match the referenced
FCB. All disk directory and data space owned by the deleted files is
returned to free space, and becomes available for allocation to other
files. Directory XFCBs that were owned by the deleted files are also
removed from the directory. If interface attribute f5' of the FCB is
set to 1, Function 19 deletes only the directory XFCBs that match the
referenced FCB.

Note: if any of the files that match the input FCB specification fall
the password check, or are Read-Only, then the Delete function does
not delete any files or XFCBS.  This applies to both types of delete
operations.

In nonbanked systems, file passwords and XFCBs are not
supported. Thus, if the Delete function is called with interface
attribute f5' set to true, the Delete function performs no action but
returns with register A set to zero.

Upon return, the Delete function returns a Directory Code in register
A with the value 0 if the delete is successful, or \$FF, 255 Decimal,
if no file that matches the referenced FCB is found. Register H is set
to zero in both of these cases. If a physical, or extended error is
encountered, the Delete function performs different actions depending
on the BDOS error mode (see Function 45). If the BDOS error mode is
the default mode, a message identifying the error is displayed at the
console and the calling program is terminated. Otherwise, the Delete
function returns to the calling program with register A set to \$FF
and register H set to one of the following physical or extended error
codes:

\begin{itemize}
\item 01 : Disk I/O error
\item 02 : Read-Only disk
\item 03 : Read-Only file
\item 04 : Invalid drive error
\item 07 : File password error
\end{itemize}

\index{read sequential}
BDOS function 20: \textbf{READ SEQUENTIAL}

Entry Parameters:
\begin{itemize}
\item[] C: \$14
\item[] DE: FCB Address
\end{itemize}

Returned Value:
\begin{itemize}
\item[] A: Error Code
\item[] H: Physical Error
\end{itemize}

The Read Sequential function reads the next 1 to 128 128-byte records
from a file into memory beginning at the current DMA address. The BDOS
Multi-Sector Count (see Function 44) determines the number of records
to be read. The default is one record. The FCB addressed by register
pair DE must have been previously activated by an Open or Make
function call.

Function 20 reads each record from byte cr of the extent, then
automatically increments the cr field to the next record position. If
the cr field overflows, then the function automatically opens the next
logical extent and resets the cr field to 0 in preparation for the
next read operation. The calling program must set the cr field to 0
following the Open call if the intent is to read sequentially from the
beginning of the file.  Upon return, the Read Sequential function sets
register A to zero if the read operation is successful. Otherwise,
register A contains an error code identifying the error as shown
below:

\begin{itemize}
\item 01 Reading unwritten data (end-of-file)
\item 09 Invalid FCB
\item 10 Media change occurred
\item 255 Physical Error; refer to register H
\end{itemize}

Error Code 01 is returned if no data exists at the next record
position of the file.  Usually, the no data situation is encountered
at the end of a file. However, it can also occur if an attempt is made
to read a data block that has not been previously written, or an
extent which has not been created. These situations are usually
restricted to files created or appended with the BDOS random write
functions (see Functions 34 and 40).

Error Code 09 is returned if the FCB is invalidated by a previous BDOS
close call that returns an error.

Error Code 10 is returned if a media change occurs on the drive after
the referenced FCB is activated by a BDOS Open, or Make Call.

Error Code 255 is returned if a physical error is encountered and the
BDOS error mode is Return Error mode, or Return and Display Error mode
(see Function 45). If the error mode is the default mode, a message
identifying the physical error is displayed at the console, and the
calling program is terminated. When a physical error is returned to
the calling program, register H contains one of the following error
codes:

\begin{itemize}
\item 01 Disk I/O error
\item 04 Invalid drive error
\end{itemize}

On all error returns except for physical error returns, A = 255,
Function 20 sets register H to the number of records successfully read
before the error is encountered.  This value can range from 0 to 127
depending on the current BDOS Multi-Sector Count. It is always set to
zero when the Multi-Sector Count is equal to one.

\index{write sequential}
BDOS function 21: \textbf{WRITE SEQUENTIAL}

Entry Parameters:
\begin{itemize}
\item[] C: \$15
\item[] DE: FCB Address
\end{itemize}

Returned Value:
\begin{itemize}
\item[] A: Error Code
\item[] H: Physical Error
\end{itemize}

The Write Sequential function writes 1 to 128 128-byte data records,
beginning at the current DMA address into the file named by the FCB
addressed in register pair DE. The BDOS Multi-Sector Count (see
Function 44) determines the number of 128 byte records that are
written. The default is one record. The referenced FCB must have been
previously activated by a BDOS Open or Make function call.

Function 21 places the record into the file at the position indicated
by the cr byte of the FCB, and then automatically increments the cr
byte to the next record position. If the cr field overflows, the
function automatically opens, or creates the next logical extent, and
resets the cr field to 0 in preparation for the next write operation.
If Function 21 is used to write to an existing file, then the newly
written records overlay those already existing in the file. The
calling program must set the cr field to 0 following an Open or Make
call if the intent is to write sequentially from the beginning of the
file.

Function 21 makes an Update date and time for the file if the
following conditions are satisfied: the referenced drive has a
directory label that requests date and time stamping, and the file has
not already been stamped for update by a previous Make or Write
function call.

Upon return, the Write Sequential function sets register A to zero if
the write operation is successful. Otherwise, register A contains an
error code identifying the error as shown below:

\begin{itemize}
\item 01 No available directory space
\item 02 No available data block
\item 09 Invalid FCB
\item 10 Media change occurred
\item 255 Physical Error : refer to register H
\end{itemize}

Error Code 01 is returned when the write function attempts to create a
new extent that requires a new directory entry, and no available
directory entries exist on the selected disk drive.

Error Code 02 is returned when the write command attempts to allocate
a new data block to the file, and no unallocated data blocks exist on
the selected disk drive.

Error Code 09 is returned if the FCB is invalidated by a previous BDOS
close call that returns an error.

Error Code 10 is returned if a media change occurs on the drive after
the referenced FCB is activated by a BDOS Open or Make call.

Error Code 255 is returned if a physical error is encountered and the
BDOS error mode is Return Error mode, or Return and Display Error mode
(see Function 45). If the error mode is the default mode, a message
identifying the physical error is displayed at the console, and the
calling program is terminated. When a physical error is returned to
the calling program, register H contains one of the following error
codes:

\begin{itemize}
\item 01 Disk I/O error
\item 02 Read-Only disk
\item 03 Read-Only file or File open from user 0 when the current user
  number is non-zero or File password protected in Write mode
\item 04 Invalid drive error
\end{itemize}

On all error returns, except for physical error returns, A = 255,
Function 21 sets register H to the number of records successfully
written before the error was encountered. This value can range from 0
to 127 depending on the current BDOS Multi-Sector Count. It is always
set to zero when the Multi-Sector Count is set to one.

\index{make file}
BDOS function 22: \textbf{MAKE FILE}

Entry Parameters:
\begin{itemize}
\item[] C: \$16
\item[] DE: FCB Address
\end{itemize}

Returned Value:
\begin{itemize}
\item[] A: Directory Code
\item[] H: Physical or Extended Error
\end{itemize}

The Make File function creates a new directory entry for a file under
the current user number. It also creates an XFCB for the file if the
referenced drive has a directory label that enables password
protection on the drive, and the calling program assigns a password to
the file.

The calling program passes the address of the FCB in register pair DE,
with byte 0 of the FCB specifying the drive, bytes 1 through 11
specifying the filename and filctype, and byte 12 set to the extent
number. Usually, byte 12 is set to zero. Byte 32 of the FCB, the cr
field, must be initialized to zero, before or after the Make call, if
the intent is to write sequentially from the beginning of the file.

Interface attribute f6' specifies whether a password is to be assigned
to the created file.

\begin{itemize}
\item f6' = 0 - Do not assign password (default)
\item f6' = 1 - Assign password to created file
\end{itemize}

When attribute f6' is set to 1, the calling program must place the
password in the first 8 bytes of the current DMA buffer, and set byte
9 of the DMA buffer to the password mode (see Function 102). Note that
the Make function only interrogates interface attribute f6' if
passwords are activated on the referenced drive. In nonbanked systems,
file passwords are not supported, and attribute f6' is never
interrogated.

The Make function returns with an error if the referenced FCB names a
file that currently exists in the directory under the current user
number.

If the Make function is successful, it activates the referenced FCB
for file operations by opening the FCB, and initializes both the
directory entry and the referenced FCB to an empty file. It also
initializes all file attributes to zero. In addition, Function 22
makes a Creation date and time stamp for the file if the following
conditions are satisfied: the referenced drive has a directory label
that requests Creation date and time stamping and the FCB extent
number field is equal to zero. Function 22 also makes an Update stamp
if the directory label requests update stamping and the FCB extent
field is equal to zero.

If the referenced drive contains a directory label that enables
password protection, and if interface attribute f6' has been set to 1,
the Make function creates an XFCB for the file. In addition, Function
22 also assigns the password, and password mode placed in the first
nine bytes of the DMA, to the XFCB.

Upon return, the Make function returns a directory code in register A
with the value 0 if the make operation is successful, or \$FF, 255
decimal, if no directory space is available. Register H is set to zero
in both of these cases. If a physical or extended error is
encountered, the Make function performs different actions depending on
the BDOS error mode (see Function 45). If the BDOS error mode is the
default mode, a message identifying the error is displayed at the
console, and the calling program is terminated. Otherwise, the Make
function returns to the calling program with register A set to \$FF,
and register H set to one of the following physical or extended error
codes:

\begin{itemize}
\item 01 : Disk I/O error
\item 02 : Read-Only disk
\item 04 : Invalid drive error
\item 08 : File already exists
\item 09 : ? in filename or filetype field
\end{itemize}

\index{rename file}
BDOS function 23: \textbf{RENAME FILE}

Entry Parameters:
\begin{itemize}
\item[] C: \$17
\item[] DE: FCB Address
\end{itemize}

Returned Value:
\begin{itemize}
\item[] A: Directory Code
\item[] H: Physical or Extended Error
\end{itemize}

The Rename function uses the FCB, addressed by register pair DE, to
change all directory entries of the file specified by the filename in
the first 16 bytes of the FCB to the filename in the second 16
bytes. If the file specified by the first filename is password
protected, the correct password must be placed in the first eight
bytes of the current DMA buffer, or have been previously established
as the default password (see Function 106). The calling program must
also ensure that the filenames specified in the FCB are valid and
unambiguous, and that the new filename does not already exist on the
drive. Function 23 uses the dr code at byte 0 of the FCB to select the
drive. The drive code at byte 16 of the FCB is ignored.

Upon return, the Rename function returns a Directory Code in register
A with the value 0 if the rename is successful, or \$0FF, 255 Decimal,
if the file named by the first filename in the FCB is not
found. Register H is set to zero in both of these cases.  If a
physical or extended error is encountered, the Rename function
performs different actions depending on the BDOS error mode (see
Function 45). If the BDOS error mode is the default mode, a message
identifying the error is displayed at the console and the program is
terminated. Otherwise, the Rename function returns to the calling
program with register A set to \$0FF and register H set to one of the
following physical or extended error codes:

\begin{itemize}
\item 01 Disk I/O error
\item 02 Read-Only disk
\item 03 Read-Only file
\item 04 Invalid drive error
\item 07 File password error
\item 08 File already exists
\item 09 ? in filename or filetype field
\end{itemize}

\index{return login vector}
BDOS function 24: \textbf{RETURN LOGIN VECTOR}

Entry Parameters:
\begin{itemize}
\item[] C: \$18
\end{itemize}

Returned Value:
\begin{itemize}
\item[] HL: Login Vector
\end{itemize}

Function 24 returns the login vector in register pair HL. The login
vector is a 16-bit value with the least significant bit of L
corresponding to drive A, and the highorder bit of H corresponding to
the 16th drive, labelled P. A 0 bit indicates that the drive is not
on-line, while a 1 bit indicates the drive is active. A drive is made
active by either an explicit BDOS Select Disk call, number 14, or an
implicit selection when a BDOS file operation specifies a non-zero dr
byte in the FCB. Function 24 maintains compatibilty with earlier
releases since registers A and L contain the same values upon return.

\index{return current disk}
BDOS function 25: \textbf{RETURN CURRENT DISK}

Entry Parameters:
\begin{itemize}
\item[] C: \$19
\end{itemize}

Returned Value:
\begin{itemize}
\item[] A: Current Disk
\end{itemize}

Function 25 returns the currently selected default disk number in
register A. The disk numbers range from 0 through 15 corresponding to
drives A through P.

\index{set dma address}
BDOS function 26: \textbf{SET DMA ADDRESS}

Entry Parameters:
\begin{itemize}
\item[] C: \$1A
\item[] DE: DMA Address
\end{itemize}

DMA is an acronym for Direct Memory Address, which is often used in
connection with disk controllers that directly access the memory of
the computer to transfer data to and from the disk subsystem. Under
CP/M 3, the current DMA is usually defined as the buffer in memory
where a record resides before a disk write, and after a disk read
operation. If the BDOS Multi-Sector Count is equal to one (see
Function 44), the size of the buffer is 128 bytes. However, if the
BDOS Multi-Sector Count is greater than one, the size of the buffer
must equal N * 128, where N equals the Multi-Sector Count.

Some BDOS functions also use the current DMA to pass parameters, and
to return values. For example, BDOS functions that check and assign
file passwords require that the password be placed in the current
DMA. As another example, Function 46, Get Disk Free Space, returns its
results in the first 3 bytes of the current DMA. When the current DMA
is used in this context, the size of the buffer in memory is
determined by the specific requirements of the called function.

When a transient program is initiated by the CCP, its DMA address is
set to \$0080. The BDOS Reset Disk System function, Function 13, also
sets the DMA address to \$0080. The Set DMA function can change this
default value to another memory address. The DMA address is set to the
value passed in the register pair DE.  The DMA address remains at this
value until it is changed by another Set DMA Address, or Reset Disk
System call.

\index{get addr(alloc)}
BDOS function 27: \textbf{GET ADDR(ALLOC)}

Entry Parameters:
\begin{itemize}
\item[] C: \$1B
\end{itemize}

Returned Value:
\begin{itemize}
\item[] HL: ALLOC Address
\end{itemize}

CP/M 3 maintains an allocation vector in main memory for each active
disk drive.  Some programs use the information provided by the
allocation vector to determine the amount of free data space on a
drive. Note, however, that the allocation information might be
inaccurate if the drive has been marked Read-Only.

Function 27 returns in register pair HL, the base address of the
allocation vector for the currently selected drive. If a physical
error is encountered when the BDOS error mode is one of the return
modes (see Function 45), Function 27 returns the value \$FFFF in the
register pair HL.

In banked CP/M 3 systems, the allocation vector can be placed in bank
zero. In this case, a transient program cannot access the allocation
vector. However, the BDOS function, Get Disk Free Space (Function 46),
can be used to directly return the number of free 128-byte records on
a drive. The CP/M 3 utilities that display a drive's free space, DIR
and SHOW, use Function 46 for that purpose.

\index{write protect disk}
BDOS function 28: \textbf{WRITE PROTECT DISK}

Entry Parameters:
\begin{itemize}
\item[] C: \$1C
\end{itemize}

The Write Protect Disk function provides temporary write protection
for the currently selected disk by marking the drive as Read-Only, No
program can write to a disk that is in the Read-Only state. A drive
reset operation must be performed for a Read-Only drive to restore it
to the Read-Write state (see Functions 13 and 37).

\index{get read-only vector}
BDOS function 29: \textbf{GET READ-ONLY VECTOR}

Entry Parameters:
\begin{itemize}
\item[] C: l\$D
\end{itemize}

Returned Value:
\begin{itemize}
\item[] HL: R/O Vector Value
\end{itemize}

Function 29 returns a bit vector in register pair HL that indicates
which drives have the temporary Read-Only bit set. The Read-Only bit
can be set only by a BDOS Write Protect Disk call.

The format of the bit vector is analogous to that of the login vector
returned by Function 24. The least significant bit corresponds to
drive A, while the most significant bit corresponds to drive P.

\index{set file attributes}
BDOS function 30: \textbf{SET FILE ATTRIBUTES}

Entry Parameters:
\begin{itemize}
\item[] C: \$1E
\item[] DE: FCB Address
\end{itemize}

Returned Value:
\begin{itemize}
\item[] A: Directory Code
\item[] H: Physical or Extended error
\end{itemize}

By calling the Set File Attributes function, a program can modify a
file's attributes and set its last record byte count. Other BDOS
functions can be called to interrogate these file parameters, but only
Function 30 can change them. The file attributes that can be set or
reset by Function 30 are fl' through f4', Read-Only, t1', System, t2',
and Archive, t3'. The register pair DE addresses an FCB containing a
filename with the appropriate attributes set or reset. The calling
program must ensure that it does not specify an ambiguous filename. In
addition, if the specified file is password totected, the correct
password must be placed in the first eight bytes of the current DMA
buffer or have been previously established as the default password
(see Function 106).

Interface attribute f6' specifies whether the last record byte count
of the specified file is to be set:

\begin{itemize}
\item f6' = 0 - Do not set byte count (default mode)
\item f6' = 1 - Set byte count
\end{itemize}

If interface attribute f6' is set, the calling program must set the cr
field of the referenced FCB to the byte count value. A program can
access a file's byte count value with the BDOS Open, Search, or Search
Next functions.

Function 30 searches the referenced directory for entries belonging to
the current user number that matches the FCB specified name and type
fields. The function then updates the directory to contain the
selected indicators, and if interface attribute f6' is set, the
specified byte count value. Note that the last record byte count is
maintained in byte 13 of a file's directory FCBS.

File attributes t1', t2', and t3' are defined by CP/M 3. (They are
described in Section 2.3.4.) Attributes fl' through f4' are not
presently used, but can be useful for application programs, because
they are not involved in the matching program used by the BDOS during
Open File and Close File operations. Indicators f5' through f8' are
reserved for use as interface attributes.

Upon return, Function 30 returns a Directory Code in register A with
the value 0 if the function is successful, or \$FF, 255 Decimal, if
the file specified by the referenced FCB is not found. Register H is
set to zero in both of these cases. If a physical or extended error is
encountered', the Set File Attributes function performs different
actions depending on the BDOS error mode (see Function 45). If the
BDOS error mode is the default mode, a message identifying the error
is displayed at the console, and the program is terminated. Otherwise,
Function 30 returns to the calling program with reg'ls-Ler A set to
\$FF, and register H set to one of the following physical or extended
error codes:

\begin{itemize}
\item 01 Disk I/O error
\item 02 Read-Only disk
\item 04 Select error
\item 07 File password error
\item 09 ? in filename or filetype field
\end{itemize}

\index{get addr(DPB parms)}
BDOS function 31: \textbf{GET ADDR(DPB PARMS)}

Entry Parameters:
\begin{itemize}
\item[] C: \$1F
\end{itemize}

Returned Value:
\begin{itemize}
\item[] HL: DPB Address
\end{itemize}

Function 31 returns in register pair HL the address of the
BIOS-resident Disk Parameter Block, DPB, for the currently selected
drive. (Refer to the CP/M Plus (CP/M Version 3) Operating System
System Guide for the format of the DPB). The calling program can use
this address to extract the disk parameter values.

If a physical error is encountered when the BDOS error mode is one of
the return modes (see Function 45), Function 31 returns the value
\$FFFF in the register pair HL.

\index{get user code}
\index{set user code}
BDOS function 32: \textbf{SET/GET USER CODE}

Entry Parameters:
\begin{itemize}
\item[] C: \$20
\end{itemize}

Returned Value:
\begin{itemize}
\item[] E: \$FF (get) or User Code (set)
\item[] A: Current Code or (no value)
\end{itemize}

A program can change, or interrogate the currently active user number
by calling Function 32. If register E = \$FF, then the value of the
current user number is returned in register A, where the value is in
the range of 0 to 15. If register E is not \$FF, then the current user
number is changed to the value of E, modulo 16.

\index{read random}
BDOS function 33: \textbf{READ RANDOM}

Entry Parameters:
\begin{itemize}
\item[] C: \$21
\item[] DE: FCB Address
\end{itemize}

Returned Value:
\begin{itemize}
\item[] A: Error Code
\item[] H: Physical Error
\end{itemize}

The Read Random function is similar to the Read Sequential function
except that the read operation takes place at a particular random
record number, selected by the 24-bit value constructed from the three
byte, r0, r1, r2, field beginning at position 33 of the FCB. Note that
the sequence of 24 bits is stored with the least significant byte
first, rO, the middle byte next, rl, and the high byte last, r2. The
random record number can range from 0 to 262,143. This corresponds to
a maximum value of 3 in byte r2.

To read a file with Function 33, the calling program must first open
the base extent, extent 0. This ensures that the FCB is properly
initialized for subsequent random access operations. The base extent
may or may not contain any allocated data. Function 33 reads the
record specified by the random record field into the current DMA
address. The function automatically sets the logical extent and
current record values, but unlike the Read Sequential function, it
does not advance the current record number. Thus, a subsequent Read
Random call rereads the same record. After a random read operation, a
file can be accessed sequentially, starting from the current randomly
accessed position. However, the last randomly accessed record is
reread or rewritten when switching from random to sequential mode.

If the BDOS Multi-Sector Count is greater than one (see Function 44),
the Read Random function reads multiple consecutive records into
memory beginning at the current DMA. The rO, rl, and r2 field of the
FCB is automatically incremented to read each record. However, the
FCBs random record number is restored to the first record's value upon
return to the calling program.

Upon return, the Read Random function sets register A to zero if the
read operation was successful. Otherwise, register A contains one of
the following error codes:

\begin{itemize}
\item 01 Reading unwritten data (end-of-file)
\item 03 Cannot close current extent
\item 04 Seek to unwritten extent
\item 06 Random record number out of range
\item 10 Media change occurred
\item 255 Physical Error : refer to register H
\end{itemize}

Error Code 01 is returned if no data exists at the next record
position of the file.  Usually, the no data situation is encountered
at the end of a file. However, it can also occur if an attempt is made
to read a data block that has not been previously written.

Error Code 03 is returned when the Read Random function cannot close
the current extent prior to moving to a new extent.

Error Code 04 is returned when a read random operation accesses an
extent that has not been created.

Error Code 06 is returned when byte 35, r2, of the referenced FCB is
greater than 3.

Error Code 10 is returned if a media change occurs on the drive after
the referenced FCB is activated by a BDOS Open or Make Call.

Error Code 255 is returned if a physical error is encountered, and the
BDOS error mode is one of the return modes (see Function 45). If the
error mode is the default mode, a message identifying the physical
error is displayed at the console, and the calling program is
terminated. When a physical error is returned to the calling program,
register H contains one of the following error codes:

\begin{itemize}
\item 01 Disk I/O error
\item 04 Invalid drive error
\end{itemize}

On all error returns except for physical errors, A = 255, the Read
Random function sets register H to the number of records successfully
read before the error is encountered. This value can range from 0 to
127 depending on the current BDOS Multi-Sector Count. It is always set
to zero when the Multi-Sector Count is equal to one.

\index{write random}
BDOS function 34: \textbf{WRITE RANDOM}

Entry Parameters:
\begin{itemize}
\item[] C: \$22
\item[] DE: FCB Address
\end{itemize}

Returned Value:
\begin{itemize}
\item[] A: Error Code
\item[] H: Physical Error
\end{itemize}

The Write Random function is analogous to the Read Random function,
except that data is written to the disk from the current DMA
address. If the disk extent or data block where the data is to be
written is not already allocated, the BDOS automatically performs the
allocation before the write operation continues.

To write to a file using the Write Random function, the calling
program must first open the base extent, extent 0. This ensures that
the FCB is properly initialized for subsequent random access
operations. If the file is empty, the calling program must create the
base extent with the Make File function before calling Function
34. The base extent might or might not contain any allocated data, but
it does record the file in the directory, so that the file can be
displayed by the DIR utility.

The Write Random function sets the logical extent and current record
positions to correspond with the random record being written, but does
not change the random record number. Thus, sequential read or write
operations can follow a random write, with the current record being
reread or rewritten as the calling program switches from random to
sequential mode.

Function 34 makes an Update date and time stamp for the file if the
following conditions are satisfied: the referenced drive has a
directory label that requests Update date and time stamping if the
file has not already been stamped for update by a previous BDOS Make
or Write call.

If the BDOS Multi-Sector Count is greater than one (see Function 44),
the Write Random function reads multiple consecutive records into
memory beginning at the current DMA. The rO, rl, and r2 field of the
FCB is automatically incremented to write each record. However, the
FCB's random record number is restored to the first record's value
when it returns to the calling program. Upon return, the Write Random
function sets register A to zero if the write operation is
successful. Otherwise, register A contains one of the following error
codes:

\begin{itemize}
\item 02 No available data block
\item 03 Cannot Close current extent
\item 05 No available directory space
\item 06 Random record number out of range
\item 10 Media change occurred
\item 255 Physical Error : refer to register H
\end{itemize}

Error Code 02 is returned when the write command attempts to allocate
a new data block to the file and no unallocated data blocks exist on
the selected disk drive.

Error Code 03 is returned when the Write Random function cannot close
the current extent prior to moving to a new extent.

Error Code 05 is returned when the write function attempts to create a
new extent that requires a new directory entry and no available
directory entries exist on the selected disk drive.

Error Code 06 is returned when byte 35, r2, of the referenced FCB is
greater than 3.

Error Code 10 is returned if a media change occurs on the drive after
the referenced FCB is activated by a BDOS Open or Make Call.

Error Code 255 is returned if a physical error is encountered and the
BDOS error mode is one of the return modes (see Function 45). If the
error mode is the default mode, a message identifying the physical
error is displayed at the console, and the calling program is
terminated. When a physical error is returned to the calling program,
it is identified by register H as shown below:

\begin{itemize}
\item 01 Disk I/O error
\item 02 Read-Only disk

\item 03 Read-Only file or File open from user 0 when the current user
  number is nonzero or File password protected in Write mode

\item 04 Invalid drive error
\end{itemize}

On all error returns, except for physical errors, A = 255, the Write
Random function sets register H to the number of records successfully
written before the error is encountered. This value can range from 0
to 127 depending on the current BDOS Multi-Sector Count. It is always
set to zero when the Multi-Sector Count is equal to one.

\index{compute file size}
BDOS function 35: \textbf{COMPUTE FILE SIZE}

Entry Parameters:
\begin{itemize}
\item[] C: \$23
\item[] DE: FCB Address
\end{itemize}

Returned Value:
\begin{itemize}
\item[] A: Error Flag
\item[] H: Physical or Extended error
\item[] Random Record Field Set
\end{itemize}

The Compute File Size function determines the virtual file size, which
is, in effect, the address of the record immediately following the end
of the file. The virtual size of a file corresponds to the physical
size if the file is written sequentially. If the file is written in
random mode, gaps might exist in the allocation, and the file might
contain fewer records than the indicated size. For example, if a
single record with record number 262,143, the CP/M 3 maximum is
written to a file using the Write Random function, then the virtual
size of the file is 262,144 records even though only 1 data block 'is
actually allocated.

To compute file size, the calling program passes in register pair DE
the address of an FCB in random mode format, bytes rO, rl and r2
present. Note that the FCB must contain an unambiguous filename and
filetype. Function 35 sets the random record field of the FCB to the
random record number + 1 of the last record in the file. If the r2
byte is set to 04, then the file contains the maximum record count
262,144.

A program can append data to the end of an existing file by calling
Function 35 to set the random record position to the end of file, and
then performing a sequence of random writes starting at the preset
record address.

Note: the BDOS does not require that the file be open to use Function
35. However, if the file has been written to, it must be closed before
calling Function 35. Otherwise, an incorrect file size might be
returned.

Upon return, Function 35 returns a zero in register A if the file
specified by the referenced FCB is found, or an \$FF in register A if
the file is not found. Register H is set to zero in both of these
cases. If a physical error is encountered, Function 35 performs
different actions depending on the BDOS error mode (see Function 45).
If the BDOS error mode is the default mode, a message identifying the
error is displayed at the console and the program is
terminated. Otherwise, Function 35 returns to the calling program with
register A set to \$FF, and register H set to one of the following
physical errors:

\begin{itemize}
\item 01 Disk I/O error
\item 04 Invalid drive error
\end{itemize}

\index{set random record}
BDOS function 36: \textbf{SET RANDOM RECORD}

Entry Parameters:
\begin{itemize}
\item[] C: \$24
\item[] DE: FCB Address
\end{itemize}

Returned Value:
\begin{itemize}
\item[] Random Record Field Set
\end{itemize} 

The Set Random Record function returns the random record number of the
next record to be accessed from a file that has been read or written
sequentially to a particular point. This value is returned in the
random record field, bytes rO, rl, and r2, of the FCB addressed by the
register pair DE. Function 36 can be useful in two ways,

First, it is often necessary to initially read and scan a sequential
file to extract the positions of various key fields. As each key is
encountered, Function 36 is called to compute the random record
position for the data corresponding to this key. If the data unit size
is 128 bytes, the resulting record number minus one is placed into a
table with the key for later retrieval. After scanning the entire file
and tabularizing the keys and their record numbers, you can move
directly to a particular record by performing a random read using the
corresponding random record number that you saved earlier. The scheme
is easily generalized when variable record lengths are involved,
because the program need only store the buffer-relative byte position
along with the key and record number to find the exact starting
position of the keyed data at a later time.

A second use of Function 36 occurs when switching from a sequential
read or write over to random read or write. A file is sequentially
accessed to a particular point in the file, then Function 36 is called
to set the record number, and subsequent random read and write
operations continue from the next record in the file.

\index{reset drive}
BDOS function 37: \textbf{RESET DRIVE}

Entry Parameters:
\begin{itemize}
\item[] C: \$25
\item[] DE: Drive Vector
\end{itemize}

Returned Value:
\begin{itemize}
\item[] A: \$00
\end{itemize}

The Reset Drive function programmatically restores specified drives to
the reset state. A reset drive is not logged-in and is in Read-Write
status. The passed parameter in register pair DE is a 16-bit vector of
drives to be reset, where the least significant bit corresponds to the
first drive A, and the high-order bit corresponds to the sixteenth
drive, labelled P. Bit values of 1 indicate that the specified drive
is to be reset.

\index{access drive}
BDOS function 38: \textbf{ACCESS DRIVE}

Entry Parameters:
\begin{itemize}
\item[] C: \$26
\end{itemize}

This is an MP/M function that is not supported under CP/M 3. If
called, the file system returns a zero In register A indicating that
the access request is successful.

\index{free drive}
BDOS function 39: \textbf{FREE DRIVE}

Entry Parameters:
\begin{itemize}
\item[] C: \$27
\end{itemize}

This is an MP/M function that is not supported under CP/M 3. If
called, the file system returns a zero In register A indicating that
the free request is successful.

\index{write random with zero fill}
BDOS function 40: \textbf{WRITE RANDOM WITH ZERO FILL}

Entry Parameters:
\begin{itemize}
\item[] C: \$28
\item[] DE: FCB address
\end{itemize}

Returned Value:
\begin{itemize}
\item[] A: Error Code
\item[] H: Physical Error
\end{itemize}

The Write Random With Zero Fill function is identical to the Write
Random function (Function 34) with the exception that a previously
unallocated data block is filled with zeros before the record is
written. If this function has been used to create a file, records
accessed by a read random operation that contain all zeros identify
unwritten random record numbers. Unwritten random records in allocated
data blocks of files created using the Write Random function (Function
34) contain uninitialized data.

\index{test and write record}
BDOS function 41: \textbf{TEST AND WRITE RECORD}

Entry Parameters:
\begin{itemize}
\item[] C: \$29
\item[] DE: FCB Address
\end{itemize}

Returned Value:
\begin{itemize}
\item[] A: Error Code
\item[] H: Physical Error
\end{itemize}

The Test and Write Record function is an MP/M function that is not
supported under CP/M 3. If called, Function 41 returns with register A
set to \$FF and register H set to zero.

\index{lock record}
BDOS function 42: \textbf{LOCK RECORD}

Entry Parameters:
\begin{itemize}
\item[] C: \$2A
\item[] DE: FCB Address
\end{itemize}

Returned Value:
\begin{itemize}
\item[] A: \$00
\end{itemize}

The Lock Record function is an MP/M II function that is supported
under CP/M 3 only to provide compatibility between CP/M 3 and MP/M. It
is intended for use in situations where more than one running program
has Read-Write access to a common file. Because CP/M 3 is a
single-user operating system in which only one program can run at a
time, this situation cannot occur. Thus, under CP/M 3, Function 42
performs no action except to return the value \$00 in register A
indicating that the record lock operation is successful.

\index{unlock record}
BDOS function 43: \textbf{UNLOCK RECORD}

Entry Parameters:
\begin{itemize}
\item[] C: \$2B
\item[] DE: FCB Address
\end{itemize}

Returned Value:
\begin{itemize}
\item[] A: \$00
\end{itemize}

The Unlock Record function is an MP/M II function that is supported
under CP/M 3 only to provide compatibility between CP/M 3 and MP/M. It
is intended for use in situations where more than one running program
has Read-Write access to a common file. Because CP/M 3 is a
single-user operating system in which only one program can run at a
time, this situation cannot occur. Thus, under CP/M 3, Function 43
performs no action except to return the value \$00 in register A
indicating that the record unlock operation is successful.

\index{set multi-sector count}
BDOS function 44: \textbf{SET MULTI-SECTOR COUNT}

Entry Parameters:
\begin{itemize}
\item[] C: \$2C
\item[] E: Number of Sectors
\end{itemize}

Returned Value:
\begin{itemize}
\item[] A: Return Code
\end{itemize}

The Set Multi-Sector Count function provides logical record blocking
under CP/M 3. It enables a program to read and write from 1 to 128
records of 128 bytes at a time during subsequent BDOS Read and Write
functions.

Function 44 sets the Multi-Sector Count value for the calling program
to the value passed in register E. Once set, the specified
Multi-Sector Count remains in effect until the calling program makes
another Set Multi-Sector Count function call and changes the
value. Note that the CCP sets the Multi-Sector Count to one when it
initiates a transient program.

The Multi-Sector Count affects BDOS error reporting for the BDOS Read
and Write functions. If an error interrupts these functions when the
Multi-Sector is greater than one, they return the number of records
successfully read or written in register H for all errors except for
physical errors (A = 255).

Upon return, register A is set to zero if the specified value is in
the range of 1 to 128. Otherwise, register A is set to \$FF.

\index{set BDOS error mode}
BDOS function 45: \textbf{SET BDOS ERROR MODE}

Entry Parameters:
\begin{itemize}
\item[] C: \$2D
\item[] E: BDOS Error Mode
\end{itemize}

Returned Value:
\begin{itemize}
\item[] None
\end{itemize}

Function 45 sets the BDOS error mode for the calling program to the
mode specified in register E. If register E is set to \$FF, 255
decimal, the error mode is set to Return Error mode. If register E is
set to \$FE, 254 decimal, the error mode is set to Return and Display
mode. If register E is set to any other value, the error mode is set
to the default mode.

The SET BDOS Error Mode function determines how physical and extended
errors (see Section 2.2.13) are handled for a program. The Error Mode
can exist in three modes: the default mode, Return Error mode, and
Return and Display Error mode.  In the default mode, the BDOS displays
a system message at the console that identifies the error and
terminates the calling program. In the return modes, the BDOS sets
register A to \$FF, 255 decimal, places an error code that identifies
the physical or extended error in register H and returns to the
calling program. In Return and Display mode, the BDOS displays the
system message before returning to the calling program. No system
messages are displayed, however, when the BDOS is in Return Error
mode.

\index{get disk free space}
BDOS function 46: \textbf{GET DISK FREE SPACE}

Entry Parameters:
\begin{itemize}
\item[] C: \$2E
\item[] E: Drive
\end{itemize}

Returned Value:
\begin{itemize}
\item[] First 3 bytes of current DMA buffer
\item[] A: Error Flag
\item[] H: Physical Error
\end{itemize}

The Get Disk Free Space function determines the number of free
sectors, 128 byte records, on the specified drive. The calling program
passes the drive number in register E, with 0 for drive A, 1 for B,
and so on, through 15 for drive P in a full 16drive system. Function
46 returns a binary number in the first 3 bytes of the current DMA
buffer. This number is returned in the following format:

fso fsl fs2

Disk Free Space Field Format

fso = low byte\\ fsl = middle byte\\ fs2 = high byte\\

Note that the returned free space value might be inaccurate if the
drive has been marked Read-Only.

Upon return, register A is set to zero if the function is
successful. However, if the BDOS Error Mode is one of the return modes
(see Function 45), and a physical error is encountered, register A is
set to \$FF, 255 decimal, and register H is set to one of the
following values:

\begin{itemize}
\item 01 - Disk I/O error
\item 04 - Invalid drive error
\end{itemize}

\index{chain to program}
BDOS function 47: \textbf{CHAIN TO PROGRAM}

Entry Parameters:
\begin{itemize}
\item[] C: \$2F
\item[] E: Chain Flag
\end{itemize}

The Chain To Program function provides a means of chaining from one
program to the next without operator intervention. The calling program
must place a command line terminated by a null byte, OOH, in the
default DMA buffer. If register E is set to \$FF, the CCP initializes
the default drive and user number to the current program values when
it passes control to the specified transient program. Otherwise, these
parameters are set to the default CCP values. Note that Function 108,
Get/Set Program Return Code, can be used to pass a two byte value to
the chained program.

Function 47 does not return any values to the calling program and any
encountered errors are handled by the CCP.

\index{flush buffers}
BDOS function 48: \textbf{FLUSH BUFFERS}

Entry Parameters:
\begin{itemize}
\item[] C: \$30
\end{itemize}

Returned Value:
\begin{itemize}
\item[] A: Error Flag
\item[] H: Physical Error
\end{itemize} E: Purge Flag

The Flush Buffers function forces the write of any write-pending
records contained in internal blocking/deblocking buffers. If register
E is set to \$FF, this function also purges all active data
buffers. Programs that provide write with read verify support need to
purge internal buffers to ensure that verifying reads actually access
the disk instead of returning data that is resident in internal data
buffers. The CP/M 3 PIP utility is an example of such a program.

Upon return, register A is set to zero if the flush operation is
successful. If a physical error is encountered, the Flush Buffers
function performs different actions depending on the BDOS error mode
(see Function 45). If the BDOS error mode is in the default mode,, a
message identifying the error is displayed at the console and the
calling program is terminated. Otherwise, the Flush Buffers function
returns to the calling program with register A set to \$FF and
register H set to the following physical error code:

\begin{itemize}
\item 0 1 Disk I/O error
\item 02 Read/only disk
\item 04 Invalid drive error
\end{itemize}

\index{get system control bock}
\index{set system control bock}
BDOS function 49: \textbf{GET/SET SYSTEM CONTROL BLOCK}

Entry Parameters:
\begin{itemize}
\item[] C: \$31
\item[] DE: SCB PB Address
\end{itemize}

Returned Value:
\begin{itemize}
\item[] A: Returned Byte
\item[] HL: Returned Word
\end{itemize}

Function 49 allows access to parameters located in the CP/M 3 System
Control Block (SCB). The SCB is a 100-byte data structure residing
within the BDOS that contains flags and data used by the BDOS, CCP and
other system components. Note that Function 49 is a CP/M 3 specific
function. Programs intended for both MP/M 11 and CP/M 3 should either
avoid the use of this function or isolate calls to this function in
CP/M 3 version-dependent sections.

To use Function 49, the calling program passes the address of a data
structure called the SCB parameter block in register pair DE. This
data structure identifies the byte or word of the SCB to be updated or
returned. The SCB parameter block is defined as:

\begin{verbatim}
SCBPB: DB OFFSET ; Offset within SCB
       DB SET ; 0FFH if setting a byte
       ; 0FEH if setting a word
       ; 001H - 0FDH are reserved
       ; 000H if a get operation
       DW VALUE ; Byte or word value to be set
\end{verbatim}

The OFFSET parameter identifies the offset of the field within the SCB
to be updated or accessed. The SET parameter determines whether
Function 49 is to set a byte or word value in the SCB or if it is to
return a byte from the SCB. The VALUE parameter is used only in set
calls. In addition, only the first byte of VALUE is referenced in set
byte calls.

Use caution when you set SCB fields. Some of these parameters reflect
the current state of the operating system. If they are set to invalid
values, software errors can result. In general, do not use Function 49
to set a system parameter if another BDOS function can achieve the
same result. For example, Function 49 can be called to update the
Current DMA Address field within the SCB. This is not equivalent to
making a Function 26, Set DMA Address call, and updating the SCB
Current DMA field in this way would result in system errors. However,
you can use Function 49 to return the Current DMA address. The System
Control Block is summarized in \ref{table:91}.

\begin{table}[h]\centering
  \caption{System Control Block}
\label{table:91}
\csvautotabular{system/cpm/scb.csv}
\end{table}

If Function 49 is called with the OFFSET parameter of the SCB
parameter block greater than \$63, the function performs no action but
returns with registers A and HL set to zero.

\index{direct BIOS calls}
BDOS function 50: \textbf{DIRECT BIOS CALLS}

Entry Parameters:
\begin{itemize}
\item[] C: \$32
\item[] DE: BIOS PB Address
\end{itemize}

Returned Value:
\begin{itemize}
\item[] BIOS RETURN
\end{itemize}

Function 50 provides a direct BIOS call through the BDOS to the
BIOS. The calling program passes the address of a data structure
called the BIOS Parameter Block (BIOSPB) in register pair DE. The
BIOSPB contains the BIOS function number and register contents as
shown below:

\begin{verbatim}
BIOSPB: db FUNC ; BIOS function no.
        db AREG ; A register contents
        dw BCREG ; BC register contents
        dw DEREG ; DE register contents
        dw HLREG ; HL register contents
\end{verbatim}

System Reset (Function 0) is equivalent to Function 50 with a BIOS
function number of 1.

Note that the register pair BIOSPB fields (BCREG, DEREG, HLREG) arc
defined in low byte, high byte order. For example, in the BCREG field,
the first byte contains the C register value, the second byte contains
the B register value.  Under CP/M 3, direct BIOS calls via the BIOS
jump vector are only supported for the BIOS Console I/O and List
functions. You must use Function 50 to call any other

BIOS functions. In addition, Function 50 intercepts BIOS Function 27
(Select Memory) calls and returns with register A set to zero. Refer
to the CPIM Plus (CP/M Version 3) Operating System System Guide for
the definition of the BIOS functions and their register passing and
return conventions.

\index{load overlay}
BDOS function 59: \textbf{LOAD OVERLAY}

Entry Parameters:
\begin{itemize}
\item[] C: \$3B
\item[] DE: FCB Address
\end{itemize}

Returned Value:
\begin{itemize}
\item[] A: Error Code
\item[] H: Physical Error
\end{itemize}

Only transient programs with an RSX header can use the Load Overlay
function because BDOS Function 59 is supported by the LOADER
module. The calling program must have a header to force the LOADER to
remain resident after the program is loaded (see Section 1.3).

Function 59 loads either an absolute or relocatable
module. Relocatable modules are identified by a filetype of
PRL. Function 59 does not call the loaded module.

The referenced FCB must be successfully opened before Function 59 is
called. The load address is specified in the first two random record
bytes of the FCB, rO and rl.  The LOADER returns an error if the load
address is less than \$100, or if performing the requested load
operation would overlay the LOADER, or any other Resident System
Extensions that have been previously loaded.

When loading relocatable files, the LOADER requires enough room at the
load address for the complete PRL file including the header and bit
map (see Appendix B).  Otherwise an error is returned. Function 59
also returns an error on PRL file load requests if the specified load
address is not on a page boundary.

Upon return, Function 59 sets register A to zero if the load operation
is successful.  If the LOADER RSX is not resident in memory because
the calling program did not have a RSX header, the BDOS returns with
register A set to \$FF and register H set to zero. If the LOADER
detects an invalid load address, or if insufficient memory is
available to load the overlay, Function 59 returns with register A set
to \$FE. All other error returns are consistent with the error codes
returned by BDOS Function 20, Read Sequential.

\index{call resident system extension}
BDOS function 60: \textbf{CALL RESIDENT SYSTEM EXTENSION}

Entry Parameters:
\begin{itemize}
\item[] C: \$3C
\item[] DE: RSX PB Address
\end{itemize}

Returned Value:
\begin{itemize}
\item[] A: Error Code
\item[] H: Physical Error
\end{itemize}

Function 60 is a special BDOS function that you use when you call
Resident System Extensions. The RSX subfunction is specified in a
structure called the RSX Parameter Block, defined as follows:

\begin{verbatim}
RSXPB: db FUNC ; RSX Function number
       db NUMPARMS ; Number of word Parameters
       dw PARMETER1 ; Parameter I
       dw PARMETER2 ; Parameter 2
       . . .
       dw PARMETERN ; Parameter n
\end{verbatim}

RSX modules filter all BDOS calls and capture RSX function calls that
they can handle. If there is no RSX module present in memory that can
handle a specific RSX function call, the call is not trapped, and the
BDOS returns \$FF in registers A and L. RSX function numbers from 0 to
127 are available for CP/M 3 compatible software use. RSX function
numbers 128 to 255 are reserved for system use.

\index{free blocks}
BDOS function 98: \textbf{FREE BLOCKS}

Entry Parameters:
\begin{itemize}
\item[] C: \$62
\end{itemize}

Returned Value:
\begin{itemize}
\item[] A: Error Flag
\item[] H: Physical Error
\end{itemize}

The Free Blocks function scans all the currently logged-in drives, and
for each drive returns to free space all temporarily-allocated data
blocks. A temporarily-allocated data block is a block that has been
allocated to a file by a BDOS write operation but has not been
permanently recorded in the directory by a BDOS close operation. The
CCP calls Function 98 when it receives control following a system warm
start. Be sure to close your file, particularly any file you have
written to, prior to calling Function 98.

In the nonbanked version of CP/M 3, Function 98 frees only temporarily
allocated blocks for systems that request double allocation vectors in
GENCPM.

Upon return, register A is set to zero if Function 98 is
successful. If a physical error is encountered, the Free Blocks
function performs different actions depending on the BDOS error mode
(see Function 45). If the BDOS error mode is in the default mode, a
message identifying the error is displayed at the console and the
calling program is terminated. Otherwise, the Free Blocks function
returns to the calling program with register A set to \$FF and
register H set to the following physical error code:

\begin{itemize}
\item 04 : Invalid drive error
\end{itemize}

\index{truncate file}
BDOS function 99: \textbf{TRUNCATE FILE}

Entry Parameters:
\begin{itemize}
\item[] C: \$63
\item[] DE: FCB Address
\end{itemize}

Returned Value:
\begin{itemize}
\item[] A: Directory Code
\item[] H: Extended or Physical Error
\end{itemize}

The Truncate File function sets the last record of a file to the
random record number contained in the referenced FCB. The calling
program passes the address of the FCB in register pair DE, with byte 0
of the FCB specifying the drive, bytes 1 through 11 specifying the
filename and filetype, and bytes 33 through 35, rO, rl, and r2,
specifying the last record number of the file. The last record number
is a 24 bit value, stored with the least significant byte first, rO,
the middle byte next, rl, and the high byte last, r2. This value can
range from 0 to 262,143, which corresponds to a maximum value of 3 in
byte r2.

If the file specified by the referenced FCB is password protected, the
correct password must be placed in the first eight bytes of the
current DMA buffer, or have been previously established as the default
password (see Function 106).

Function 99 requires that the file specified by the FCB not be open,
particularly if the file has been written to. In addition, any
activated FCBs naming the file are not valid after Function 99 is
called. Close your file before calling Function 99, and then reopen it
after the call to continue processing on the file.

Function 99 also requires that the random record number field of the
referenced FCB specify a value less than the current file size. In
addition, if the file is sparse, the random record field must specify
a record in a region of the file where data exists.

Upon return, the Truncate function returns a Directory Code in
register A with the value 0 if the Truncate function is successful, or
\$FF, 255 decimal, if the file is not found or the record number is
invalid. Register H is set to zero in both of these cases. If a
physical or extended error is encountered, the Truncate function
performs different actions depending on the BDOS error mode (see
Function 45). If the BDOS error mode is in the default mode, a message
identifying the error is displayed at the console and the program is
terminated. Otherwise, the Truncate function returns to the calling
program with register A set to \$FF and register H set to one of the
following physical or extended error codes:

\begin{itemize}
\item 01 Disk I/O error
\item 02 Read-Only disk
\item 03 Read-Only file
\item 04 Invalid drive error
\item 07 File password error
\item 09 ? in filename or filetype field
\end{itemize}

\index{set directory label}
BDOS function 100: \textbf{SET DIRECTORY LABEL}

Entry Parameters:
\begin{itemize}
\item[] C: \$64
\item[] DE: FCB Address
\end{itemize}

Returned Value:
\begin{itemize}
\item[] A: Directory Code
\item[] H: Physical or Extended Error
\end{itemize}

The Set Directory Label function creates a directory label, or updates
the existing directory label for the specified drive. The calling
program passes in register pair DE the address of an FCB containing
the name, type, and extent fields to be assigned to the directory
label. The name and type fields of the referenced FCB are not used to
locate the directory label in the directory; they are simply copied
into the updated or created directory label. The extent field of the
FCB, byte 12, contains the user's specificat-on of the directory label
data byte. The definition of the directory label data byte is:

\begin{itemize}
\item[bit] 7 - Require passwords for password-protected files (Not
  supported in nonbanked CP/M 3 systems)
\item[bit] 6 - Perform access date and time stamping
\item[bit] 5 - Perform update date and time stamping
\item[bit] 4 - Perform create date and time stamping
\item[bit] 0 - Assign a new password to the directory label
\end{itemize}

If the current directory label is password protected, the correct
password must be placed in the first eight bytes of the current DMA,
or have been previously established as the default password (see
Function 106). If bit 0, the low-order bit, of byte 12 of the FCB is
set to 1, it indicates that a new password for the directory label has
been placed in the second eight bytes of the current DMA.

Note that Function 100 is implemented as an RSX, DIRLBL.RSX, in
nonbanked CP/M 3 systems. If Function 100 is called in nonbanked
systems when the DIRLBL.RSX is not resident an error code of \$0FF is
returned.

Function 100 also requires that the referenced directory contain SFCBs
to activate date and time stamping on the drive. If an attempt is made
to activate date and time stamping when no SFCBs exist, Function 100
returns an error code of \$FF in register A and performs no
action. The CP/M 3 INITDIR utility initializes a directory for date
and time stamping by placing an SFCB record in every fourth entry of
the directory.

Function 100 returns a Directory Code in register A with the value 0
if the directory label create or update is successful, or \$FF, 255
decimal, if no space exists in the referenced directory to create a
directory label, or if date and time stamping was requested and the
referenced directory did not contain SFCBS. Register H is set to zero
in both of these cases. If a physical error or extended error is
encountered, Function 100 performs different actions depending on the
BDOS error mode (see Function 45). If the BDOS error mode is the
default mode, a message identifying the error is displayed at the
console and the calling program is terminated. Otherwise, Function 100
returns to the calling program with register A set to \$FF and
register H set to one of the following physical or extended error
codes:

\begin{itemize}
\item 01 Disk I/O error
\item 02 Read-Only disk
\item 04 Invalid drive error
\item 07 File password error
\end{itemize}

\index{return directory label data}
BDOS function 101: \textbf{RETURN DIRECTORY LABEL DATA}

Entry Parameters:
\begin{itemize}
\item[] C: \$65
\item[] E: Drive
\end{itemize}

Returned Value:
\begin{itemize}
\item[] A: Directory Label Data Byte
\item[] H: Physical Error
\end{itemize}

The Return Directory Label Data function returns the data byte of the
directory label for the specified drive. The calling program passes
the drive number in register E with 0 for drive A, 1 for drive B, and
so on through 15 for drive P in a full sixteen drive system. The
format of the directory label data byte is shown below:

\begin{itemize}
\item[bit] 7 - Require passwords for password protected files
\item[bit] 6 - Perform access date and time stamping
\item[bit] 5 - Perform update date and time stamping
\item[bit] 4 - Perform create date and time stamping
\item[bit] 0 - Directory label exists on drive
\end{itemize}

Function 101 returns the directory label data byte to the calling
program in register A. Register A equal to zero indicates that no
directory label exists on the specified drive. If a physical error is
encountered by Function 101 when the BDOS Error mode is in one of the
return modes (see Function 45), this function returns with register A
set to \$FF, 25S decimal, and register H set to one of the following:

\begin{itemize}
\item 01 Disk I/O error
\item 04 Invalid drive error
\end{itemize}

\index{read file date stamps and password mode}
BDOS function 102: \textbf{READ FILE DATE STAMPS AND PASSWORD MODE}

Entry Parameters:
\begin{itemize}
\item[] C: \$66
\item[] DE: FCB Address
\end{itemize}

Returned Value:
\begin{itemize}
\item[] A: Directory Code
\item[] H: Physical Error
\end{itemize}

Function 102 returns the date and time stamp information and password
mode for the specified file in byte 12 and bytes 24 through 32 of the
specified FCB. The calling program passes in register pair DE, the
address of an FCB in which the drive, filename, and filetype fields
have been defined.

If Function 102 is successful, it sets the following fields in the
referenced FCB:

\begin{itemize}
\item[] byte 12 : Password mode field
\begin{itemize}
\item[bit] 7 - Read mode
\item[bit] 6 - Write mode
\item[bit] 4 - Delete mode
\end{itemize}
\end{itemize}

Byte 12 equal to zero indicates the file has not been assigned a
password. In nonbanked systems, byte 12 is always set to zero.

\begin{itemize}
\item[] byte 24 - 27 Create or Access time stamp field
\item[] byte 28 - 31 Update time stamp field
\end{itemize}

The date stamp fields are set to binary zeros if a stamp has not been
made. The format of the time stamp fields is the same as the format of
the date and time structure described in Function 104.

Upon return, Function 102 returns a Directory Code in register A with
the value zero if the function is successful, or \$FF, 255 decimal, if
the specified file is not found. Register H is set to zero in both of
these cases. If a physical or extended error is encountered, Function
102 performs different actions depending on the BDOS error mode (see
Function 45). If the BDOS error mode is in the default mode, a message
identifying the error is displayed at the console and the calling
program is terminated. Otherwise, Function 102 returns to the calling
program with register A set to \$FF and register H set to one of the
following physical or extended error codes:

\begin{itemize}
\item 01 Disk I/O error
\item 04 Invalid drive error
\item 09 ? in filename or filetype field
\end{itemize}

\index{write file XFCB}
BDOS function 103: \textbf{WRITE FILE XFCB}

Entry Parameters:
\begin{itemize}
\item[] C: \$67
\item[] DE: FCB Address
\end{itemize}

Returned Value:
\begin{itemize}
\item[] A: Directory Code
\item[] H: Physical Error
\end{itemize}

The Write File XFCB function creates a new XFCB or updates the
existing XFCB for the specified file. The calling program passes in
register pair DE the address of an FCB in which the drive, name, type,
and extent fields have been defined. The extent field specifies the
password mode and whether a new password is to be assigned to the
file. The format of the extent byte is shown below:

FCB byte 12 (ex) : XFCB password mode
\begin{itemize}
\item[bit] 7 - Read mode
\item[bit] 6 - Write mode
\item[bit] 5 - Delete mode
\item[bit] 0 - Assign new password to the file
\end{itemize}

If the specified file is currently password protected, the correct
password must reside in the first eight bytes of the current DMA, or
have been previously established as the default password (see Function
106). If bit 0 is set to 1, the new password must reside in the second
eight bytes of the current DMA.

Upon return, Function 103 returns a Directory Code in register A with
the value zero if the XFCB create or update is successful, or \$FF,
255 decimal, if no directory label exists on the specified drive, or
the file named in the FCB is not found, or no space exists in the
directory to create an XFCB. Function 103 also returns with \$FF in
register A if passwords are not enabled by the referenced directory's
label. On nonbanked systems, this function always returns with
register A = \$FF because passwords are not supported. Register H is
set to zero in all of these cases. If a physical or extended error is
encountered, Function 103 performs different actions depending on the
BDOS error mode (see Function 45). If the BDOS error mode is the
default mode, a message identifying the error is displayed at the
console and the calling program is terminated. Otherwise, Function 103
returns to the calling program with register A set to \$FF and
register H set to one of the following physical or extended error
codes:

\begin{itemize}
\item 01 Disk I/O error
\item 02 Read-Only disk
\item 04 Invalid drive error
\item 07 File password error
\item 09 ? in filename or filetype field
\end{itemize}

\index{set date and time}
BDOS function 104: \textbf{SET DATE AND TIME}

Entry Parameters:
\begin{itemize}
\item[] C: \$68
\item[] DE: DAT Address
\end{itemize}

Returned Value:
\begin{itemize}
\item[] none
\end{itemize}

The Set Date and Time function sets the system internal date and
time. The calling program passes the address of a 4-byte structure
containing the date and time specification in the register pair
DE. The format of the date and time (DAT) data structure is:

\begin{itemize}
\item[byte] 0 - 1 Date field
\item[byte] 2 Hour field
\item[byte] 3 Minute field
\end{itemize}

The date is represented as a 16-bit integer with day 1 corresponding
to January 1, 1978. The time is represented as two bytes: hours and
minutes are stored as two BCD digits.

This function also sets the seconds field of the system date and time
to zero.

\index{get date and time}
BDOS function 105: \textbf{GET DATE AND TIME}

Entry Parameters:
\begin{itemize}
\item[] C: \$69
\item[] DE: DAT Address
\end{itemize}

Returned Value:
\begin{itemize}
\item[] A: seconds
\item[] DAT set
\end{itemize}

The Get Date and Time function obtains the system internal date and
time. The calling program passes in register pair DE, the address of a
4-byte data structure which receives the date and time values. The
format of the date and time, DAT, data structure is the same as the
format described in Function 104. Function 105 also returns the
seconds field of the system date and time in register A as a two digit
BCD value.

\index{set default password}
BDOS function 106: \textbf{SET DEFAULT PASSWORD}

Entry Parameters:
\begin{itemize}
\item[] C: \$6A
\item[] DE: Password Address
\end{itemize}

Returned Value:
\begin{itemize}
\item[] none
\end{itemize}

The Set Default Password function allows a program to specify a
password value before a file protected by the password is
accessed. When the file system accesses a password-protected file, it
checks the current DMA, and the default password for the correct
value. If either value matches the file's password, full access to the
file is allowed. Note that this function performs no action in
nonbanked CP/M 3 systems because file passwords are not supported.

To make a Function 106 call, the calling program sets register pair DE
to the address of an 8-byte field containing the password.

\index{return serial number}
BDOS function 107: \textbf{RETURN SERIAL NUMBER}

Entry Parameters:
\begin{itemize}
\item[] C: \$6B
\item[] DE: Serial Number Field
\end{itemize}

Returned Value:
\begin{itemize}
\item[] Serial number field set
\end{itemize}

Function 107 returns the CP/M 3 serial number to the 6-byte field
addressed by register pair DE.

\index{get program return code}
\index{set program return code}
BDOS function 108: \textbf{GET/SET PROGRAM RETURN CODE}

Entry Parameters:
\begin{itemize}
\item[] C: \$6C
\item[] DE: \$0FFFF (Get) or Program Return Code (Set)
\end{itemize}

Returned Value:
\begin{itemize}
\item[] HL: Program Return Code or (no value)
\end{itemize}

CP/M 3 allows programs to set a return code before terminating. This
provides a mechanism for programs to pass an error code or value to a
following job step in batch environments. For example, Program Return
Codes are used by the CCP in CP/M 3's conditional command line batch
facility. Conditional command lines are command lines that begin with
a colon, :. The execution of a conditional command depends on the
successful execution of the preceding command. The CCP tests the
return code of a terminating program to determine whether it
successfully completed or terminated in error. Program return codes
can also be used by programs to pass an error code or value to a
chained program (see Function 47, Chain To Program).

A program can set or interrogate the Program Return Code by calling
Function 108. If re 'ster pair DE = \$FFFF, then the current Program
Return Code is returned in register pair HL. Otherwise, Function 108
sets the Program Return Code to the 'value contained in register pair
DE. Program Return Codes are defined in \ref{table:92}

\begin{table}[h]\centering
  \caption{Program Return Codes}
  \label{table:92}
\csvautotabular{system/cpm/return_codes.csv}
\end{table}

\index{get console mode}
\index{set console mode}
BDOS function 109: \textbf{GET/SET CONSOLE MODE}

Entry Parameters:
\begin{itemize}
\item[] C: \$6D
\item[] DE: \$FFFF (Get) or Console Mode (Set)
\end{itemize}

Returned Value:
\begin{itemize}
\item[] HL: Console Mode or (no value)
\end{itemize}

A program can set or interrogate the Console Mode by calling Function
109. If register pair DE = \$FFFF, then the current Console Mode is
returned in register HL. Otherwise, Function 109 sets the Console Mode
to the value contained in register pair DE.

The Console Mode is a 16-bit system parameter that determines the
action of certain BDOS Console I/O functions. The definition of the
Console Mode is:

\begin{itemize}
\item[bit 0] = 1 - CTRL-C only status for Function 1 1.\\
  = 0 - Normal status for Function 1 1.
\item[bit 1] = 1- Disable stop scroll, CTRL-S, start scroll, CTRL-Q,
  support.\\
  = 0-Enable stop scroll, start scroll support.
\item[bit 2] = 1- Raw console output mode. Disables tab expansion for
  Functions 2, 9 and 111. Also disables printer echo, CTIRL-P,
  support.\\
  = 0 - Normal console output mode.
\item[bit 3] = 1 - Disable CTRL-C program termination\\
  = 0 - Enable CTRL-C program termination
\item[bits 8,9] -Console status mode for RSXs that perform console
  input redirection from a file. These bits determine how the RSX
  responds to console status requests.
  \begin{itemize}
  \item[] bit 8 = 0, bit 9 = 0 - conditional status
  \item[] bit 8 = 0, bit 9 = 1 - false status
  \item[] bit 8 = 1, bit 9 = 0 - true status
  \item[] bit 8 = 1, bit 9 = 1 - bypass redirection
  \end{itemize}
\end{itemize}

Note that the Console Mode bits are numbered from right to left.

The CCP initializes the Console Mode to zero when it loads a program
unless the program has an RSX that overrides the default value. Refer
to Section 2.2.1 for detailed information on Console Mode.

\index{get output delimiter}
\index{set output delimiter}
BDOS function 110: \textbf{GET/SET OUTPUT DELIMITER}

Entry Parameters:
\begin{itemize}
\item[] C: \$6E
\item[] DE: \$FFFF (Get) or
\item[] E: Output Delimiter (Set)
\end{itemize}

Returned Value:
\begin{itemize}
\item[] A: Output Delimiter or (no value)
\end{itemize}

A program can set or interrogate the current Output Delimiter by
calling Function 110. If register pair DE = \$FFFF, then the current
Output Delimiter is returned in register A. Otherwise, Function 110
sets the Output Delimiter to the value contained in register E.

Function 110 sets the string delimiter for Function 9, Print
String. The default delimiter value is a dollar sign, \$. The CCP
restores the Output Delimiter to the default value when a transient
program is loaded.

\index{print block}
BDOS function 111: \textbf{PRINT BLOCK}

Entry Parameters:
\begin{itemize}
\item[] C: \$6F
\item[] DE: CCB Address
\end{itemize}

Returned Value:
\begin{itemize}
\item[] none
\end{itemize}

The Print Block function sends the character string located by the
Character Control Block, CCB, addressed in register pair DE, to the
logical console, CONOUT:. If the Console Mode is in the default state
(see Section 2.2.1), Function 111 expands tab characters, CTRL-I, in
columns of eight characters. It also checks for stop scroll, CTRL-S,
start scroll, CTRL-Q, and echoes to the logical list device, LST:, if
printer echo, CTRL-P, has been invoked.

The CCB format is:
\begin{itemize}
\item[byte] 0 - 1 Address of character string (word value)
\item[byte] 2 - 3 Length of character string (word value)
\end{itemize}

\index{list block}
BDOS function 112: \textbf{LIST BLOCK}

Entry Parameters:
\begin{itemize}
\item[] C: \$70
\item[] DE: CCB Address
\end{itemize}

Returned Value:
\begin{itemize}
\item[] none
\end{itemize}

The List Block function sends the character string located by the
Character Control Block, CCB, addressed in register pair DE, to the
logical list device, LST:.

The CCB format is:
\begin{itemize}
\item[byte] 0 - 1 Address of character string (word value)
\item[byte] 2 - 3 Length of character string (word value)
\end{itemize}

\index{parse filename}
BDOS function 152: \textbf{PARSE FILENAME}

Entry Parameters:
\begin{itemize}
\item[] C: \$98
\item[] DE: PFCB Address
\end{itemize}

Returned Value:
\begin{itemize}
\item[] HL: Return code
\item[] Parsed file control block
\end{itemize}

The Parse Filename function parses an ASCII file specification and
prepares a File Control Block, FCB. The calling program passes the
address of a data structure called the Parse Filename Control Block,
PFCB, in register pair DE. The PFCB contains the address of the input
ASCII filename string followed by the address of the target FCB as
shown below:

\begin{verbatim}
 PFCB: DW INPUT ; Address of input ASCII string
       DW FCB ; Address of target FCB
\end{verbatim}

The maximum length of the input ASCII string to be parsed is 128
bytes. The target FCB must be 36 bytes in length.

Function 152 assumes the input string contains file specifications in
the following form:

\{d:\}filename\{.typ\}\{;password\}

where items enclosed in curly brackets are optional. Function 152 also
accepts isolated drive specifications d: in the input string. When it
encounters one, it sets the filename, filetype, and password fields in
the FCB to blank.

The Parse Filename function parses the first file specification it
finds in the input string. The function first eliminates leading
blanks and tabs. The function then assumes that the file specification
ends on the first delimiter it encounters that is out of context with
the specific field it is parsing. For instance, if it finds a colon,
and it is not the second character of the file specification, the
colon delimits the entire file specification.

Function 152 recognizes the following characters as detimiters:

\begin{itemize}
\item[] space
\item[] tab
\item[] return
\item[] nut]
\item[] ; (semicolon) - except before password field
\item[] = (equal)
\item[] \(<\) (less than)
\item[] \(>\) (greater than)
\item[] . (period) - except after filename and before filetype
\item[] : (colon) - except before filename and after drive
\item[] , (comma)
\item[] \(|\) (vertical bar)
\item[] [ (left square bracket)
\item[] ] (right square bracket)
\end{itemize}

If Function 152 encounters a non-graphic character in the range 1
through 31 not listed above, it treats the character as an error. The
Parse Filename function initializes the specified FCB shown in
\ref{table:93}.

\begin{table}[h]\centering
  \caption{FCB Format}
  \label{table:93}
\csvautotabular{system/cpm/fcb.csv}
\end{table}

If an error occurs, Function 152 returns an \$FFFF in register pair
HL.
  
On a successful parse, the Parse Filename function checks the next
item in the input string. It skips over trailing blanks and tabs and
looks at the next character. If the character is a null or carriage
return, it returns a 0 indicating the end of the input string. If the
character is a delimiter, it returns the address of the delimiter. If
the character is not a delimiter, it returns the address of the first
trailing blank or tab.

If the first non-blank or non-tab character in the input string is a
null, 0, or carriage return, the Parse Filename function returns a
zero indicating the end of string.

If the Parse Filename function is to be used to parse a subsequent
file specification in the input string, the returned address must be
advanced over the delimiter before placing it in the PFCB.
