\chapter{DMA} 
Development

[74\_New]Given such a large memory size, a fast data transfer facility
is needed.

TSconf offers us such a tool that allows you to transfer data in
memory without the participation of the processor.


Tao says: The transfer speed is 7 MHz, copying takes place two bytes
(16 bits), provided that there is no access to the CPU, video, or a
TCU during this clock cycle.

On average: 4 bytes - 2 clocks, dma refers to the RAM for 1 beat of 7
MHz, 16 bits, 2 calls are needed for transfer. We get : DMA speed 7
MB/s

DMA can copy data from the following sources:
\begin{table}
\begin{tabular}{lll}
Src & Destination & \\
RAM & RAM & RAM (Src) is copied to RAM (Dst)\\
BLT & RAM & Pixels from RAM (Src) are copied to RAM (Dst) if they non zero\\
SPI & RAM & SPI data is copied to RAM (Dst)\\
RAM & SPI & RAM (Src) is copied to SPI\\
IDE & RAM & IDE data is copied to RAM (Dst)\\
RAM & IDE & RAM (Src) is copied to IDE\\
FILL & RAM & RAM (Dst) is filled with word from RAM (Src)\\
RAM & CRAM & RAM (Src) is copied to CRAM (Dst)\\
RAM & SFILE & RAM (Src) is copied to SFILE (Dst)\\
\end{tabular}
\end{table}
What can we use?

Direct copying, copying mode for images with transparency
(transparency is not copied - overlaying data over), copying from SPI
devices (for example - sd-card), copying from the hard drive, filling
the memory with the specified 16 bits (clearing the screen - also fill
color) copying to the system’s internal memory for the palette and for
sprites.

Copying is done by filling in the ports of the DMA system:
\begin{itemize}
\item source data start address, source data page;
\item address of the beginning of the receiver, the page for receiving
  data;
\item the amount of data transmitted at one time (the length of the
  burst), the number of such transmission units (burst);
\item setting the copy mode that starts the transfer
\end{itemize}

The maximum length for one transmission is 512 bytes (256 times 2
bytes), the maximum number of transmission blocks is 256. The maximum
possible ONE data transfer is: 256 * 512 bytes = 131072 bytes for one
copy.

Naturally, the length can be set different.

Considering the fact that the transmission takes place in 2 bytes (16
bits), at a minimum we can copy 2 bytes of data, that there are 4
points in 16c mode (one byte is 2 points), or 2 points in 256 colors
mode. All addresses of operations are even.

Copy Modes:
\begin{table}
\begin{tabular}{lp{4in}}
S\_ALGN & Source Address Alignment\\
D\_ALGN & Destination Address Alignment\\
0 & After each burst address keeped as is\\
1 & After each burst lower bits of address restored to their initial
values before burst, upper bits increased by 1
\end{tabular}
\end{table}
\begin{table}
\begin{tabular}{lp{4in}}
A\_SZ & Address Alignment Size\\
0 & 256 bytes (8 lower address bits) alignment\\
1 & 512 bytes (9 lower address bits) alignment
\end{tabular}
\end{table}
\begin{table}
\begin{tabular}{lp{4in}}
A\_SZ & Blitting Bitness\\
0 & 4 bits per pixel, transparent color is 0 from 16 (4'b0)\\
1 & 8 bits per pixel, transparent color is 0 from 256 (8'b0)
\end{tabular}
\end{table}
These bits are used to set system actions after copying a single data
block (burst) of a given length:
\begin{itemize}
\item S\_ALGN, D\_ALGN: these bits are responsible for the “alignment”
  of the source / destination address. In the state 0, the address
  does not change, in the state 1- the address will be restored to the
  initial one with an increase in the high address depending on the
  A\_SZ bit. This mode is useful to us when copying graphics.
\item A\_SZ - data alignment method: 256/512 bytes after each burst,
  which is useful for copying 4-bit (16 color) or 8-bit graphics
\end{itemize}
For more information about \#27af (DMACtrl launch port):

DMACtrl ~R/W - S\_ALGN D\_ALGN A\_SZ DDEV[2:0]

Here, bit 7 is R / W, the 6th bit is not used, 5 is source alignment,
4 is receiver alignment, 3 is alignment type (512/256), then 3 bits of
the device.

Dao says: The address for DMA does not take into account the state of
the two high-order bits of the 16-bit address.  Accordingly, \#c000 =
\#0000. You can use both \#c000 and \#0000 for the address - they are
equivalent.  When copying data, the location in the memory (address)
for the source / receiver must be a multiple of 2.

So how do you manage all this goodness?

DMA ports must be programmed (\#xxAF): High byte of source port:

\#1Aaf, \#1Baf - address (low DMASAddrL, high DMASAddrH bytes)

\#1Caf - DMASAddrX start page High byte of receiver port

\#1Daf, \#1Eaf - address (low DMADAddrL, older DMADAddrH bytes)

\#1Faf - DMADAddrX start page

Port of one burst transmission length: \#26af, DMALen

Port of transmission burst number: \#28af, DMANum Transmission

control / transmission status port: \#27af, DMACtrl

Send data to this port \#27af (DMACtrl) launches DMA transfer

Examples:

Copy the image of 256x256 pixels in 16c mode to the screen.

The original image is located at \#c000 of the glasspat\_page of the
page, copied to the address of the \#c000 of the screen page.

Let's get down to business in a smarter way than simply programming
each port:
\begin{verbatim}
ld bc, #1aaf    ; DMASAddrL
xor a
out (c),a
inc b
out (c),a
\end{verbatim}
So, we copy the graphics with alignment (moving to the next line on
the screen) after each burst.

We note that A\_SZ is turned off, which is necessary for 16 colors:
\begin{verbatim}
                ld hl,glasspat_copy
                call set_ports
                ret

glasspat_page   equ #22
Vid_page        equ #80

glasspat_copy   db #1a,0
                db #1b,0
                db #1c,glasspat_page
                db #1d,0
                db #1e,0
                db #1f,Vid_page
                db #26,256/4-1
                db #28,256-1
                db #27,DMA_RAM + DMA_DALGN
                db #ff
\end{verbatim}
To do this, use the routine to issue data to the port ( set\_ports ):
\begin{verbatim}
set_ports       ld c,#AF
.m1             ld b,(hl)
                inc hl
                inc b
                jr z,dma_stats
                outi
                jr .m1

dma_stats       ld b,high DMASTATUS
                in a,(c)
                AND #80
                jr nz,$-4
                ret
\end{verbatim}
After setting the value for port 27 the transfer starts. We need to
control the employment of DMA, for this we use reading this same port
for a busy signal (DMA\_ACT bit). Setting this bit to 0 indicates that
the transfer is complete.

Cleaning (fill) screen:
\begin{verbatim}
                ld bc,PAGE3
                ld a,Vid_page
                out (c),a
                ld hl,0         ;00 - color set in the palette
                ld (#c000),hl
                ld hl,fill_screen
                jp set_ports

fill_screen     defb #1a,0      ;
                defb #1b,0      ;
                defb #1c,Vid_page       ;
                defb #1d,0      ;
                defb #1e,0      ;
                defb #1f,Vid_page       ;

                defb #28,200    ;
                defb #26,#ff    ;
                defb #27,%00000100    ; DMA_FILL
                db #ff
\end{verbatim}
In this case, we include the video page from the address \#c000 and
fill it with the color we need (in this case, color No. 00), after
which we start the fill. Copying occurs from the address 0000
(remember that the system is on the drum as the state of the older two
bits and one of the youngest) is filled with a length of 256 * 2, 200
bursts.

Cleaning the tile layer:
\begin{verbatim}
clear_tileset   ld bc,PAGE3
                ld a,Tile_page
                out (c),a
                ld hl,0
                ld (#c000),hl
                ld hl,tileset_clr
                jp set_ports
tileset_clr
                defb #1a,0      ;
                defb #1b,0      ;
                defb #1c,Tile_page      ;
                defb #1d,0      ;
                defb #1e,0      ;
                defb #1f,Tile_page      ;
                defb #28,#ff    ;
                defb #26,#3f    ;
                defb #27,%00000100
                db #ff
\end{verbatim}
Copying the palette to the palette memory:
\begin{verbatim}
                ld hl,pals
                jp set_ports

pals            db #1a,low pal
                db #1b,high pal
                db #1c,2
                db #1d,0
                db #1e,0
                db #1f,0
                db #26,#10
                db #28,0
                db #27,DMA_RAM_CRAM     ; #84 - copy from RAM to CRAM
                db #ff

                align 2
pal             incbin "palette.tga.pal"
\end{verbatim}
In this case, copy 16 colors (one palette), from the second page to
the beginning of the palette memory.

[Question-M]Questions

:? Why is there DMA\_DALGN?\\
! Since the image is 256 pixels wide, after the last 256 points we
have to place a new piece of data from a new line. This bit will help
us.

? Why is the DMANum port (\#28af) such a strange value: 256-1?\\
! Considering that the maximum number of burst is 256, and the minimum
is 1, it is assumed that the value sent to the port 0 is the number of
burst 1 piece, and the sent value 255 is respectively 256 bursts.

? Why is the DMALen port (\#26af) given such a strange value: 256 /
4-1?\\
! Since we have an image size of 256 points in 16-color mode, we get 2
points in one byte, and 4 points in two bytes.  Again, the value 0 for
this port is the minimum length of 2 bytes.

? Is it worth waiting for the end of the transfer in the read loop of
the DMASTATUS port?\\
! If you are sure that the shipment will have time to work out before
the start of a new one - it is not necessary at all. Moreover, it is
recommended that the waiting cycle be set BEFORE a new DMA
transaction, so that the z80 continues its computational work while
the DMA fulfills the task.

? I sent the palette, but it is not really that color at all!\\
! Check the location of the palette - it should be stored in memory at
an even address. ALIGN 2 to help you. Or, perhaps, not there copied?

? Picture sent to the screen, and he moved down the lines!\\
! The size of the picture should be a multiple of 2m bytes - 2 points
(256 c), 4 points (16 c)

? and DMA when copying the page itself switches?\\
! Of course! And the source and receiver.
