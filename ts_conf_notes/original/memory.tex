% This file was converted from HTML to LaTeX with
% gnuhtml2latex program
% (c) Tomasz Wegrzanowski <maniek@beer.com> 1999
% (c) Gunnar Wolf <gwolf@gwolf.org> 2005-2010
% Version : 0.4.
\documentclass{article}
\begin{document}
\section*{TSconf: Memory}

Development

\textbf{Consider the memory location in the system. }

ZX Evolution has 4MB of memory.

The organization of this memory is similar to zx spectrum 128 - paging
is used.

Transcribed into pages of memory, we have 256 pages of 16 kb each.

The above picture describes how the memory of the Spectrum is
organized. A similar system is used in TSconf, but somewhat expanded -
we are able to place pages not only from the address \#c000:

\begin{verbatim}
    Page0 (port #10af) - адрес #0000
    Page1 (port #11af) - адрес #4000
    Page2 (port #12af) - адрес #8000
    Page3 (port #13af) - адрес #c000
\end{verbatim}

Writing to the port switches the page in the window we need.

To use the paging system from address \#0000, you need to use the
MemConfig port (\#21af):

\begin{quotation}
  \textbf{Port MemConfig, \#21af:}

  MemConfig LCK128 [1: 0] - - W0\_RAM! W0\_MAP W0\_WE ROM128

  bits 7.6 - LCK128 - memory mode: 00-512k, 01 - 128k, 10 - Auto, 11 -
  1024 k. The default is data from the BIOS.

  bit 3 - W0\_RAM, 2 - W0\_MAP, 1 - W0\_WE, bit 0 - ROM128 (the same
  as bit 4 in \#7FFD)

  W0\_RAM 0 - ROM / 1 - RAM

  ! W0\_MAP 0 - mapped / 1 - Page0

  W0\_WE 0 - WP / 1 - WE (for ROM and RAM)
\end{quotation}
With its help, it is possible to switch the 0th memory window to read
/ write mode (the usual state is read only, there is the specter ROM),
and to enable mapping - the ability to place other pages of memory,
not just the standard page 0.

In window \#0000 you can enable RAM instead of ROM with optional
prohibition of recording in this window.

W0\_RAM = 0 - enabled ROM

W0\_RAM = 1 - enabled RAM

Bit W0\_WE allows writing for RAM and programming (WE signal for ROM):

W0\_WE = 0

W0\_RAM = 0 - ROM firmware is disabled

W0\_RAM = 1 - writing to RAM at \#0000-3FFF not allowed

W0\_WE = 1

W0\_RAM = 0 - ROM firmware is allowed

W0\_RAM = 1 - writing to RAM at \#0000-3FFF is allowed

\textbf{Bit W0\_MAP}sets the page selection method for window \#0000.

W0\_MAP = 0 - the page number is selected depending on the current ZX
mode (Basic 48, Basic 128, TR-DOS).

Page layout: 0 - system, 1 - TR-DOS, 2 - Basic 128, 3 - Basic 48.

However, the register Page0 selects a bank of 4 pages to be used:

W0\_RAM = 0 - for window \#0000 is selected ROM

Page0 = 0bxxx000xx - pages 0-3 are used,

Page0 = 0bxxx001xx - pages 4-7 are used,

...

Page0 = 0bxxx111xx - pages 28-31 are used.

W0\_RAM = 1 - RAM is selected for the window \#0000 - this mode can be
used to debug the firmware without reprogramming the ROM or testing
non-standard pzushek.

Page0 = 0b000000xx - pages 0-3 are used,

Page0 = 0b000001xx - pages 4-7 are used,

...

Page0 = 0b111111xx - pages 252-255 are used.

W0\_MAP = 1 - the page number is taken directly from the register
Page0 - linear mode

W0\_RAM = 0 - pages

0-31 ROM W0\_RAM = 1 - pages 0-255 RAM

\textbf{Example:}

Turn on page \#80 from address \#8000

\begin{verbatim}
		ld a,#80
		ld bc,PAGE2
		out (c),a
\end{verbatim}

Enable the ability to work in page 0

\begin{verbatim}
		ld bc,MemConfig
		ld a,%00001110
		out (c),a
\end{verbatim}

Voila, write, post with 0000.

\subsubsection*{2 comments}

Does this only concern tsconf or the whole evo as a whole?

Only tskonf. The architecture of the Eva is mostly due to the
configuration in the FPSA.

\end{document}
