% This file was converted from HTML to LaTeX with
% gnuhtml2latex program
% (c) Tomasz Wegrzanowski <maniek@beer.com> 1999
% (c) Gunnar Wolf <gwolf@gwolf.org> 2005-2010
% Version : 0.4.
\documentclass{article}
\begin{document}
\section*{TSconf: Utils}
Development

\textbf{TSconf Developer Tools}

\begin{itemize}
\item ASM includes
\item Graphic conversion, graphics editor
\item Spg builder
\end{itemize}
\section*{TSconf: Ports}
Development

In this section, I would like to describe all the port groups involved
in the management of the TSconf system.

This section describes the general purpose of ports. The complete dock
for the ports lies here .

So, the system ports can be divided into the following groups:
\begin{itemize}
\item control ports for code execution by the processor. This includes
  such ports as: SysConfig, CacheConfig
\item memory management ports: MemConfig, Page0 - Page3, FMAddr
\item TSU graphics page ports: VPage, T0GPage, T1GPage, TMPage, SGPage
\item Graphics / Color Management ports: VConfig, PalSel, Border
\item display control ports: TSConfig, GXOffs / GYOffs, T0XOffs /
  T0YOffs, T1XOffs / T1YOffs
\item DMA management ports: DMASAddr, DMADAddr, DMALen, DMANum,
  DMACtrl
\item ports of management of arrival INT: INTMask, HSINT, VSINT
\item Virtual FDD Management Port: FDDVirt
\end{itemize}
\section*{TSconf: ints}
Development

TSconf has an extended system of interrupts that can be triggered if
there are such states as: beam arrival at a given screen position,
beam arrival at the beginning of a line, line display on the screen,
completion of data transfer.

\begin{quotation}
  Tao says: The system has three types of masked interrupts that can
  be called at an address that has a high byte — the address in
  register I, and the youngest — its type:
  \begin{enumerate}
  \item \#FF - frame
  \item \#FD - lowercase (Line)
  \item \#FB - DMA.
  \end{enumerate}
  The processing of these interrupts can be switched by the INTMask
  port (\# \textbf{2A} af), changing the state of the bits:

  0 - Frame, 1 - Line, 2 - DMA, which leads to an on / off call to
  handlers. State of bits: 0 - disabled / 1 - enabled.

  In the case of the arrival of several events at the same time, the
  interrupt with the lower number will be processed first.
\end{quotation}
\section*{TSconf: Gfx layer}
Development

Sprites, tiles ... Perhaps this would be enough for us ...

But under these layers is the \textbf{base graphics layer} .

\begin{quotation}
  Tao says: The graphics layer displays data that resides in memory
  pages. The first page to display (its address is necessarily a
  multiple of 8 for 16 colors, 16 for a 256 color mode, total length -
  8/16 pages) is indicated by the port VPage (\# 01af). The way this
  memory is mapped is set by the VConfig port bits, which sets the
  resolution and color depth.

  The memory display window is a block with the dimensions of the
  specified resolution and is displayed by the positions X (0-511)
  and Y (0-511), which are indicated by the pairs of ports GXOffs
  and GYOffs. The window is looped around the edges in the display.

  The colors of the display are given by the palette, the number is
  selected by the first notebook of the PalSel register (\# 06af).
\end{quotation}

So, we have a screen with our own internal scrolling.
\section*{TSconf: Memory}
Development

\textbf{Consider the memory location in the system. }

ZX Evolution has 4MB of memory.

The organization of this memory is similar to zx spectrum 128 -
paging is used.

Transcribed into pages of memory, we have 256 pages of 16 kb each.
\section*{TSconf: DMA}
Development

Given such a large memory size, a fast data transfer facility is
needed.

TSconf offers us such a tool that allows you to transfer data in
memory without the participation of the processor.

\begin{quotation}
  \textbf{Tao says:} The transfer speed is 7 MHz, copying takes place
  two bytes (16 bits), provided that there is no access to the CPU,
  video, or a TCU during this clock cycle.

  On average: 4 bytes - 2 clocks, dma refers to the RAM for 1 beat
  of 7 MHz, 16 bits, 2 calls are needed for transfer. We get

  : DMA speed 7 MB / s
\end{quotation}
DMA can copy data from the following sources:
\section*{TSconf: Tiles}
Development

Yes, Multimatograf close, you need to push!

So, tiles.

To build tiles, we need to do the following:

- prepare graphics \emph{(place it on the page for tile graphics)}

- build a tile map \emph{(with placement on the map page)}

- program ports \emph{(specify which pages are used for graphics, for
  a map, and enable the display layer) }
\section*{Tao konfy}

Development

\textbf{Tile-sprite configuration (TSconf). Introduction}

In my opinion, TSconf is a very modern add-on over the beloved ZX
Spectrum, which brings long-awaited and necessary elements in the
form of color to a point, hardware sprites, and so on. This will be
discussed in this article.

Tile-sprite configuration (TSconf) can be divided into the following
logical groups:
\begin{enumerate}
\item Graphic accelerator
  \begin{itemize}
  \item Use of tiles
  \item Output and control of sprites
  \end{itemize}
\item Memory manager
\item Direct memory access unit (DMA)
\item Interrupt system
\item Cache
\item Ports management
\end{enumerate}
Let us gradually consider all these points in order.
\section*{TSconf: Sprites}
Development

But why not pile over the sprite graphics, eh?

They have a distort.
\begin{quotation}
  Tao says: Sprite is a graphic tile with a size from 8 to 64 points,
  which can be displayed in coordinates 0-511 in X and Y, with the
  possibility of reflecting the display vertically / horizontally, and
  having transparency.

  A total of 85 sprites are available for one sprite layer.
\end{quotation}

\emph{In general, working out sprites is as follows:}
\begin{itemize}
\item The sprite needs to be placed in memory in exactly the same way
  as for tiles (or on the screen in 16 color mode).
\item The sprite palette must also be loaded into the overall palette
  of the system.
\item After that - we load the description of the sprite into the
  system and turn on the display.
\end{itemize}
Sprite becomes visible.
\end{document}
