Tile-sprite configuration (TSconf). Introduction
In my opinion, TSconf is a very modern add-on over the beloved ZX Spectrum, which brings long-awaited and necessary elements in the form of color to a point, hardware sprites, and so on. This will be discussed in this article. 

Tile-sprite configuration (TSconf) can be divided into the following logical groups: 
1. Graphic accelerator 
- Use of tiles 
- Output and control of sprites 
2. Memory manager 
3. Direct memory access unit (DMA) 
4. Interrupt system 
5. Cache 
6. Ports management 

Let us gradually consider all these points in order.

Graphic subsystem
Tao says: The screen we see is the output window of the resolution specified in the system, which displays a block of 512x512 video memory pages in the specified mode according to the given output coordinates. TSU is a device that collects data from video, tile and sprite memory, which processes and presents it at the current position of the line drawing in a given resolution / desired mode.

This unit is an extension (add-on) above the standard 6912 Spectrum screen. More precisely, the 6912 mode is part of the whole family of system permissions:
256x192
320x200
320x240
360x288
These permissions can be used in different video modes:
Zx
16c
256c
Text
To display the color for a given mode, the first block of the system's internal memory is used - the palette memory . This block represents 512 bytes and stores the full color palette of the system - two-byte data of color components for 256 colors. 
The full palette is divided into groups of 16 colors (32 bytes) for 16c mode, which gives us 16 palettes with 16 colors. 
The first palette is number 0. 
When you turn on the system, a set of 16 colors for the ZX mode is loaded into the general palette, which is located in the last, 15th palette. 

Each mode sets its own characteristics of displaying color information on the screen: 
ZX limits the output to 16 colors of the standard ZX Spectrum palette;
16c - in this mode only 16 colors are used - one of the 16 available palettes; 
256s - provides the ability to display all the loaded colors in the palette 
Text - text mode, allows you to display text in color. 

In this case, it was a question of horizontal mobility. Let's talk about vertical: 
In addition to the listed modes, the system allows the use of layered display of graphical information. The standard Spectrum screen is the base screen, but it is not the lowest. 

So, the location of the layers that form the screen: 
1. Border . It is a monochromatic fullscreen layer. Set the color of the curb. 
2. Basic (main) screen. May be included in any resolution / mode of the above. 
3. The layers of the graphic accelerator are displayed in the resolution specified for the base screen in 16 color mode: 
- sprite layer 0 
- tile layer 0 
- sprite layer 1 
- tile layer 1 
- sprite layer 2 
Thus, we get 7 layers that make up a single, visible for the user screen.
All the listed layers use colors specified in a common palette of 256 colors.

Tile and sprite accelerator layers work only in 16c mode, using the 0th color of the palette set for them as transparent. Thanks to TSU, there is no need to save video memory data under tiles and sprites, since the display on the screen is literally collected at the output of each line without changing the contents of the video memory. 
Each graphic element of these two types of layers represents a block of at least 8x8 pixels, and for each of them it is necessary to specify a palette, one of all of the available 16. 

The tile layer is a map describing the location of the graphic elements specified as an image. Thus, the position of the tile on the screen directly depends on its position in the map. 
Tile cardIt has a size of 64x64 tiles (4096 tiles in general), up to 4 palettes from 4 groups of all 16 palettes can be used for one layer. For each tile it is possible to set its own (of these 4) palette. 
In total, we have two such layers organized in the same way. 

Sprites are graphics, organized by the type of tile, but having large (multiple of 8) sizes - from 8x8 to 64x64 pixels. Sprites also have zero color transparency. 
A feature of the sprites is that for each sprite, you can specify both your own palette and the position on the screen right up to the point. 
To handle the sprites, the system uses the second block of internal memory , representing the next 512 bytes of memory, which stores " sprite handles"- data from 6 bytes, describing each sprite. The maximum number of descriptors in this memory is 85 pieces. 

In the following articles I will describe in more detail about working with the devices listed at the beginning of the article. 

PS: for those who do not want to wait and want details
